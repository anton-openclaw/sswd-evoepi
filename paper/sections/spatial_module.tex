% ══════════════════════════════════════════════════════════════════════
% SPATIAL MODULE & ENVIRONMENTAL FORCING
% Fact-checked against: sswd_evoepi/spatial.py, environment.py,
%   movement.py, config.py, scripts/sensitivity/spatial_runner.py
% ══════════════════════════════════════════════════════════════════════

\section{Spatial Module and Environmental Forcing}
\label{sec:spatial_module}

\modelname{} represents the NE Pacific range of \pyc{} as a
metapopulation network of discrete spatial nodes connected by larval
dispersal and pathogen transport. Each node carries its own
environmental forcing (sea surface temperature, salinity, flushing
rate) that modulates local disease and demographic dynamics. This
section describes the spatial architecture, connectivity matrices,
environmental time series, and agent movement model.


% ──────────────────────────────────────────────────────────────────────
\subsection{Metapopulation Network Structure}
\label{sec:network_structure}

The metapopulation is a graph $\mathcal{G} = (\mathcal{N}, \mathbf{C},
\mathbf{D})$ where each node $k \in \mathcal{N}$ represents a
geographically delineated habitat patch and $\mathbf{C}$, $\mathbf{D}$
are the larval and pathogen connectivity matrices, respectively.


\subsubsection{Node Definition}
\label{sec:node_definition}

Each node is parameterized by a \texttt{NodeDefinition} record with the
following fields:

\begin{table}[H]
\centering
\caption{Node definition fields.}
\label{tab:node_definition}
\small
\begin{tabular}{lll}
\toprule
\textbf{Field} & \textbf{Units} & \textbf{Description} \\
\midrule
\texttt{lat}, \texttt{lon}      & $^\circ$N, $^\circ$E   & Geographic coordinates \\
\texttt{carrying\_capacity}     & individuals  & Local $K$ (= habitat area $\times$ $\rho_\text{max}$) \\
\texttt{is\_fjord}              & bool         & Fjord vs.\ open coast classification \\
\texttt{sill\_depth}            & m            & Sill depth ($\infty$ for open coast) \\
\texttt{flushing\_rate}         & d$^{-1}$     & Mean annual hydrodynamic flushing $\phi_k$ \\
\texttt{mean\_sst}              & $^\circ$C    & Baseline annual mean SST \\
\texttt{sst\_amplitude}         & $^\circ$C    & Annual cycle half-range \\
\texttt{sst\_trend}             & $^\circ$C\,yr$^{-1}$ & Linear warming trend \\
\texttt{salinity}               & psu          & Effective mean salinity \\
\texttt{depth\_range}           & m            & Min--max habitat depth \\
\texttt{subregion}              & ---          & Biogeographic subregion code \\
\bottomrule
\end{tabular}
\end{table}

At runtime, each \texttt{NodeDefinition} is wrapped in a
\texttt{SpatialNode} object that holds the local population arrays
(agents and genotypes), current environmental state (SST, salinity,
flushing rate), Vibrio concentration, and diagnostic flags. The
\texttt{MetapopulationNetwork} aggregates all nodes together with the
$\mathbf{C}$, $\mathbf{D}$, and distance matrices.


\subsubsection{Internode Distance Computation}
\label{sec:distance_matrix}

Connectivity kernels require pairwise waterway distances between nodes.
Two methods are available:

\paragraph{Haversine with tortuosity.}
For small networks ($\leq 11$ nodes), geodesic great-circle distances are
computed via the Haversine formula and multiplied by a uniform tortuosity
factor $\tau = 1.5$ (intermediate between open-coast $\sim$1.2 and fjord
$\sim$2.5) to approximate along-coast path lengths:
\begin{equation}
  d_{jk}^\text{water} = \tau \times d_{jk}^\text{Haversine}.
  \label{eq:haversine_tortuosity}
\end{equation}

\paragraph{Precomputed overwater distances.}
For full-range simulations, a 489-site overwater distance matrix was
computed from GEBCO 2022 bathymetric data at 15\,arc-second resolution.
Land cells were rasterized from Natural Earth 10\,m land polygons, and
Dijkstra's algorithm on a 4-connected ocean grid yielded shortest
overwater paths. The resulting $489 \times 489$ matrix spans 2.0--7{,}187\,km,
with 98.4\% of pairs connected (1{,}946 disconnected pairs involve western
Aleutian sites near the antimeridian). Model nodes are matched to the
nearest precomputed site within a 50\,km tolerance; unmatched nodes fall
back to Haversine~$\times\,\tau$.


% ──────────────────────────────────────────────────────────────────────
\subsection{Connectivity Matrices}
\label{sec:connectivity}

Two connectivity matrices govern inter-node exchange:
$\mathbf{C}$ for annual larval dispersal and $\mathbf{D}$ for daily
pathogen transport (Errata~E5). Both use exponential distance kernels
but operate at different spatial and temporal scales.


\subsubsection{Larval Connectivity Matrix $\mathbf{C}$}
\label{sec:larval_connectivity}

$C_{jk}$ gives the probability that a competent larva produced at
node~$j$ settles at node~$k$. The matrix is constructed from an
exponential dispersal kernel with explicit self-recruitment:
\begin{equation}
  C_{jk} =
  \begin{cases}
    \alpha_j & \text{if } j = k, \\[4pt]
    (1 - \alpha_j)\;\exp\!\left(-\dfrac{d_{jk}}{D_L}\right) b_{jk}
    & \text{if } j \neq k,
  \end{cases}
  \label{eq:C_raw}
\end{equation}
where:
\begin{itemize}
  \item $D_L = 400$\,km is the larval dispersal length scale,
        reflecting the 4--8 week pelagic larval duration (PLD) of
        \pyc{} \citep{strathmann1987reproduction};
  \item $\alpha_j$ is the self-recruitment fraction: $\alpha_\text{fjord}
        = 0.30$ for fjord nodes (reflecting enhanced retention behind
        sills) and $\alpha_\text{open} = 0.10$ for open-coast nodes;
  \item $b_{jk} \in [0, 1]$ is an optional barrier attenuation factor
        for biogeographic breaks (e.g., Cape Mendocino).
\end{itemize}

Rows are then normalized so that:
\begin{equation}
  \sum_{k} C_{jk} = r_\text{total} = 0.02,
  \label{eq:C_normalize}
\end{equation}
where $r_\text{total}$ represents the total per-larva settlement success
probability, accounting for the compounding losses of pelagic mortality,
failed metamorphosis, and post-settlement predation.

The elevated self-recruitment fraction for fjord nodes ($\alpha_\text{fjord}
= 3\alpha_\text{open}$) encodes the empirical observation that fjords
act as larval retention zones \citep{swearer2002evidence}: sill-mediated
circulation traps larvae near their natal site, reducing export to the
open coast.


\subsubsection{Pathogen Dispersal Matrix $\mathbf{D}$}
\label{sec:pathogen_dispersal}

$D_{jk}$ gives the fraction of waterborne \vp{} at node~$j$ that reaches
node~$k$ per day. Pathogen dispersal operates at much shorter range than
larval dispersal:
\begin{equation}
  D_{jk} = \phi_j \, f_\text{out}
    \;\exp\!\left(-\frac{d_{jk}}{D_P}\right)\, S_{jk}
  \quad\text{for}\quad d_{jk} \leq 50\;\text{km},
  \label{eq:D_matrix}
\end{equation}
where:
\begin{itemize}
  \item $D_P = 15$\,km is the pathogen dispersal scale (reflecting
        tidal-current transport);
  \item $\phi_j$ is the source node's flushing rate (d$^{-1}$);
  \item $f_\text{out} = 0.2$ is the fraction of flushed water reaching
        neighboring sites;
  \item $S_{jk}$ is the sill attenuation factor.
\end{itemize}

Pairs beyond $d_{jk} > 50$\,km receive zero pathogen transfer. Total
export from any node is capped at its flushing rate: $\sum_k D_{jk}
\leq \phi_j$.

\paragraph{Sill attenuation.}
Fjord sills impede pathogen exchange between basins. The attenuation
factor is computed from the minimum sill depth across the pair:
\begin{equation}
  S_{jk} = \min\!\left(1,\;\left[\frac{\min(z_j^\text{sill},\,
  z_k^\text{sill})}{\max(z_j^\text{max},\, z_k^\text{max})}
  \right]^2\right),
  \label{eq:sill_attenuation}
\end{equation}
where $z^\text{sill}$ is sill depth and $z^\text{max}$ is maximum habitat
depth. For open-coast nodes ($z^\text{sill} = \infty$), $S_{jk} = 1$ (no
attenuation). For Howe Sound (sill~$= 30$\,m, max depth~$= 100$\,m),
$S \approx 0.09$, reducing pathogen exchange by $\sim$91\%.


\subsubsection{Dispersal Dynamics}
\label{sec:dispersal_dynamics}

\paragraph{Pathogen dispersal (daily).}
At each timestep, the dispersal input to node~$k$ is:
\begin{equation}
  \Delta P_k^\text{dispersal} = \sum_j D_{jk}\,P_j
  = \left(\mathbf{D}^\top \mathbf{P}\right)_k,
  \label{eq:pathogen_dispersal_step}
\end{equation}
which enters the Vibrio ODE (Eq.~\ref{eq:vibrio_ode}) as an additive
source term.

\paragraph{Larval dispersal (annual).}
At the end of each reproductive season, competent larvae from each source
node are distributed to receiving nodes via $\mathbf{C}$. For source
node~$j$ producing $n_j$ competent larvae: (i)~a binomial draw
$n_\text{settle} \sim \text{Bin}(n_j,\,\sum_k C_{jk})$ determines total
settlement; (ii)~a multinomial draw allocates settlers across destinations
proportional to the conditional probabilities $C_{jk} / \sum_k C_{jk}$;
(iii)~settler genotypes are sampled with replacement from the source
pool.


% ──────────────────────────────────────────────────────────────────────
\subsection{Environmental Forcing}
\label{sec:environmental_forcing}

Each node receives a locally parameterized environmental forcing that
drives disease and demographic rates through temperature-dependent,
salinity-dependent, and flushing-dependent mechanisms.


\subsubsection{Sea Surface Temperature}
\label{sec:sst_forcing}

The model supports two SST forcing modes, selected via the
\texttt{sst\_source} configuration parameter:

\paragraph{Satellite climatology (default for validation).}
Daily SST forcing uses climatological means derived from the NOAA
Optimum Interpolation SST v2.1 dataset \citep{oisstv21}, a
0.25\textdegree{} daily global product spanning 1981--present.
For each node, the 365-day climatology is computed as the 2002--2025
day-of-year average, extracted via ERDDAP for the node's geographic
coordinates. This approach captures real seasonal dynamics including
asymmetric warming/cooling profiles and coastal upwelling effects
that a symmetric sinusoidal function cannot represent. A configurable
linear warming trend $\gamma_k$ ($^\circ$C\,yr$^{-1}$; default 0.02)
is overlaid for future projection scenarios:
\begin{equation}
  T_k(d, y) = T_{k,\text{clim}}(d) + \gamma_k\,(y - y_\text{ref}),
  \label{eq:sst_satellite}
\end{equation}
where $T_{k,\text{clim}}(d)$ is the satellite-derived climatological
SST for day-of-year $d$ at node $k$.

\paragraph{Sinusoidal approximation (fallback).}
For nodes lacking satellite data or for rapid prototyping, SST follows
a sinusoidal annual cycle:
\begin{equation}
  T_k(d, y) = \underbrace{\bar{T}_k + \gamma_k\,(y - y_\text{ref})}_{%
    \text{trend-adjusted mean}}
  + \underbrace{A_k \cos\!\left(\frac{2\pi\,(d - d_\text{peak})}{365}
    \right)}_{\text{annual cycle}},
  \label{eq:sst}
\end{equation}
where $\bar{T}_k$ is the baseline annual mean SST, $A_k$ the annual
cycle half-range, and $d_\text{peak} = 227$ (day of year $\approx$
August~15) corresponds to the late-summer SST maximum characteristic
of the NE Pacific.

For the 11-node stepping-stone network used in sensitivity analysis,
satellite climatologies produce a $\sim$6$\,^\circ$C latitudinal
gradient from Sitka ($\sim$8$\,^\circ$C annual mean) to Monterey
($\sim$13$\,^\circ$C), consistent with published SST atlases. The
satellite forcing also captures site-specific features such as the
narrow summer peak at upwelling-dominated sites (Newport, Crescent City)
versus the broader warm season at sheltered sites (Howe Sound).

SST time series are precomputed at initialization via
\texttt{generate\_satellite\_sst\_series} (satellite mode) or
\texttt{make\_sst\_timeseries} (sinusoidal mode) and stored as dense
1-D arrays of shape $(n_\text{years} \times 365)$ for efficient
daily lookup.


\subsubsection{Temperature-Dependent Rate Scaling}
\label{sec:arrhenius_spatial}

All temperature-dependent biological rates---disease progression,
pathogen shedding, Vibrio decay---are scaled via the Arrhenius function:
\begin{equation}
  k(T) = k_\text{ref}\;\exp\!\left[
    \frac{E_a}{R}\left(\frac{1}{T_\text{ref}} - \frac{1}{T}\right)
  \right],
  \label{eq:arrhenius_spatial}
\end{equation}
with $T_\text{ref} = 293.15$\,K (20$\,^\circ$C), the thermal optimum
of \vp{} \citep{lambert1998virulence}. This formulation ensures that the
latitudinal SST gradient (Eq.~\ref{eq:sst}) produces emergent
north--south gradients in disease severity, matching the observed
pattern of southward-increasing SSWD mortality during the 2013--2015
outbreak \citep{montecino2016geographic, harvell2019disease}.


\subsubsection{Salinity Modifier}
\label{sec:salinity_spatial}

Vibrio viability is suppressed at low salinities via a quadratic ramp
(Eq.~\ref{eq:salinity_mod}), reproduced here for completeness:
\begin{equation}
  S_\text{sal} = \text{clip}\!\left(
    \left[\frac{S_k - S_\text{min}}{S_\text{full} - S_\text{min}}
    \right]^2,\;0,\;1
  \right),
  \quad S_\text{min} = 10\;\text{psu},\;\;
  S_\text{full} = 28\;\text{psu}.
  \label{eq:salinity_spatial}
\end{equation}
Fjord nodes receive lower salinities (e.g., Howe Sound $S = 22$\,psu
due to freshwater runoff), yielding $S_\text{sal} = 0.44$ and reducing
effective Vibrio viability by $\sim$56\% compared to open-coast nodes
($S \geq 30$\,psu, $S_\text{sal} \geq 0.87$). This mechanism provides
a partial explanation for fjord refugia observations
\citep{montecino2016geographic}.


\subsubsection{Flushing Rate}
\label{sec:flushing}

Hydrodynamic flushing removes waterborne pathogen at rate $\phi_k$
(d$^{-1}$), entering the Vibrio ODE as the term $-\phi_k P_k$
(Eq.~\ref{eq:vibrio_ode}). Node-specific values span two orders of
magnitude:
\begin{itemize}
  \item Open coast: $\phi_k = 0.5$--$1.0$\,d$^{-1}$ (strong tidal and
        current flushing);
  \item Semi-enclosed bays: $\phi_k = 0.3$\,d$^{-1}$ (San Juan Islands);
  \item Fjords: $\phi_k = 0.007$--$0.05$\,d$^{-1}$ (Errata~E3; sill
        restricts water exchange). Howe Sound is assigned $\phi_k =
        0.03$\,d$^{-1}$.
\end{itemize}

Low flushing in fjords acts as a double-edged mechanism: it reduces the
rate of pathogen removal (potentially increasing local Vibrio
concentrations) but also reduces pathogen \emph{export} to neighboring
nodes via $\mathbf{D}$ (Eq.~\ref{eq:D_matrix}), effectively isolating
the fjord from regional epidemic dynamics.

Flushing rates are optionally modulated seasonally:
\begin{equation}
  \phi_k(m) = \bar{\phi}_k \left[1 + A_\phi\,\cos\!\left(
    \frac{2\pi\,(m - 5)}{12}\right)\right],
  \label{eq:seasonal_flushing}
\end{equation}
where $m$ is the 0-indexed month, $A_\phi = 0.3$ for fjord nodes and
$A_\phi = 0.2$ for open coast, with peak flushing in June ($m = 5$)
corresponding to freshwater-driven estuarine circulation maxima.


% ──────────────────────────────────────────────────────────────────────
\subsection{Agent Movement}
\label{sec:movement}

Within each node, agents move via a correlated random walk (CRW) that
produces realistic small-scale spatial structure:
\begin{align}
  \theta(t + \Delta t) &= \theta(t) + \mathcal{N}(0,\,\sigma_\theta^2),
    \label{eq:crw_heading} \\
  x(t + \Delta t) &= x(t) + v_i \cos\theta \;\Delta t,
    \label{eq:crw_x} \\
  y(t + \Delta t) &= y(t) + v_i \sin\theta \;\Delta t,
    \label{eq:crw_y}
\end{align}
where $\sigma_\theta = 0.6$\,rad is the turning-angle standard deviation,
$v_i = v_\text{base} \times m_\text{state}$ is the disease-modified speed,
and $\Delta t = 60$\,min (hourly substeps, 24 per day). The base speed
$v_\text{base} = 0.5$\,m\,min$^{-1}$ is consistent with undisturbed
\pyc{} locomotion rates \citep{kay2002emlet}. Disease state modifies
speed: $m_S = m_E = 1.0$ (healthy), $m_{I_1} = 0.5$ (mild impairment),
$m_{I_2} = 0.1$ (severe wasting), $m_D = 0.0$ (stationary carcass),
$m_R = 1.0$ (recovered).

Agents are confined to a square habitat of side length
$\sqrt{\text{habitat\_area}}$ with elastic boundary reflection.

\paragraph{Spatial transmission.}
When spatial transmission is enabled, each node's habitat is discretized
into a grid with cell size $\Delta x = 20$\,m. Infected agents deposit
pathogen exposure proportional to their shedding rate into their grid
cell, and two Gaussian diffusion passes (3$\times$3 averaging)
smooth the resulting density field. Susceptible agents then experience
locally elevated or reduced force of infection depending on their
proximity to infected individuals, creating emergent disease clustering
without modifying the node-level Vibrio ODE.

\paragraph{Sensitivity analysis substeps.}
The full 24 hourly substeps per day incur $\sim$20$\times$ computational
overhead. For sensitivity analysis runs (Section~\ref{sec:sensitivity}),
movement is reduced to 1~substep per day, which captures spatial mixing
and aggregation effects at acceptable cost.


% ──────────────────────────────────────────────────────────────────────
\subsection{Network Configurations}
\label{sec:network_configurations}

Three network configurations are used across model development,
validation, and sensitivity analysis.


\subsubsection{5-Node Validation Network}
\label{sec:5node}

The primary validation network spans the NE~Pacific range with five
nodes selected to represent key biogeographic contexts
(Table~\ref{tab:5node}):

\begin{table}[H]
\centering
\caption{5-node validation network configuration. SST parameters are
baseline values at reference year 2000.}
\label{tab:5node}
\small
\begin{tabular}{lcccccc}
\toprule
\textbf{Node} & \textbf{Lat} & \textbf{Lon}
  & $\bar{T}$ ($^\circ$C) & $A$ ($^\circ$C)
  & $S$ (psu) & $\phi$ (d$^{-1}$) \\
\midrule
Sitka, AK      & 57.06 & $-$135.34 & 8.0  & 3.5 & 32.0 & 0.80 \\
Howe Sound, BC & 49.52 & $-$123.25 & 10.0 & 4.0 & 22.0 & 0.03 \\
San Juan Is, WA & 48.53 & $-$123.02 & 10.0 & 4.0 & 30.0 & 0.30 \\
Newport, OR    & 44.63 & $-$124.05 & 12.0 & 3.0 & 33.0 & 1.00 \\
Monterey, CA   & 36.62 & $-$121.90 & 14.0 & 2.5 & 33.5 & 0.80 \\
\bottomrule
\end{tabular}
\end{table}

Howe Sound is the sole fjord node (sill depth~$= 30$\,m, $\alpha_\text{self}
= 0.30$); all others are open coast ($\alpha_\text{self} = 0.10$). Node
carrying capacities range from 400 (Howe Sound) to 1{,}000 (Sitka). This
network reproduces three key empirical patterns: the north--south SSWD
mortality gradient, fjord protection, and the absence of recovery in
southern populations (Section~\ref{sec:validation}).


\subsubsection{11-Node Sensitivity Analysis Network}
\label{sec:11node}

Sensitivity analysis Rounds~1--3 used a minimal 3-node network (Sitka,
Howe Sound, Monterey) with inter-node distances of 1{,}700+\,km---far
exceeding the larval dispersal scale $D_L = 400$\,km. Consequently,
the spatial connectivity parameters ($D_L$, $\alpha_\text{self}$) were
effectively untestable, as the exponential kernel
$\exp(-1700/400) < 10^{-2}$ produced negligible inter-node exchange
regardless of $D_L$ values within the SA range.

Round~4 introduced an 11-node stepping-stone chain with six additional
intermediate nodes (Table~\ref{tab:11node}), reducing maximum inter-node
spacing to $\sim$452\,km and ensuring that $D_L$ values within the SA
range [100, 1{,}000]\,km produce meaningful variation in larval exchange
(32--76\% at adjacent-node distances of 110--452\,km with the default
$D_L = 400$\,km).

\begin{table}[H]
\centering
\caption{11-node stepping-stone network for sensitivity analysis
Round~4. All nodes have $K = 5{,}000$ ($\sim$55{,}000 total agents).
SST trend = 0.02\,$^\circ$C\,yr$^{-1}$ for all nodes.}
\label{tab:11node}
\small
\begin{tabular}{lcccccc}
\toprule
\textbf{Node} & \textbf{Lat} & \textbf{Lon}
  & $\bar{T}$ ($^\circ$C) & $A$ ($^\circ$C)
  & $S$ (psu) & $\phi$ (d$^{-1}$) \\
\midrule
Sitka          & 57.06 & $-$135.34 & 8.0  & 3.5 & 32.0 & 0.80 \\
Ketchikan      & 55.34 & $-$131.64 & 8.5  & 3.5 & 31.0 & 0.50 \\
Haida Gwaii    & 53.25 & $-$132.07 & 9.0  & 3.0 & 31.5 & 0.60 \\
Bella Bella    & 52.16 & $-$128.15 & 9.5  & 3.5 & 28.0 & 0.40 \\
Howe Sound$^*$ & 49.52 & $-$123.25 & 10.0 & 4.0 & 22.0 & 0.03 \\
SJI            & 48.53 & $-$123.02 & 10.5 & 4.0 & 30.0 & 0.30 \\
Westport       & 46.89 & $-$124.10 & 11.0 & 3.5 & 32.0 & 0.50 \\
Newport        & 44.63 & $-$124.05 & 11.5 & 3.0 & 33.0 & 0.60 \\
Crescent City  & 41.76 & $-$124.20 & 12.0 & 2.5 & 33.0 & 0.50 \\
Fort Bragg     & 39.45 & $-$123.80 & 12.5 & 2.5 & 33.5 & 0.50 \\
Monterey       & 36.62 & $-$121.90 & 13.0 & 2.5 & 33.5 & 0.40 \\
\bottomrule
\multicolumn{7}{l}{\small $^*$Fjord node (sill depth = 30\,m,
  $\alpha_\text{self} = 0.30$). All other nodes open coast
  ($\alpha_\text{self} = 0.10$).}
\end{tabular}
\end{table}

This upgrade substantially altered parameter importance rankings:
$n_\text{resistance}$ rose from rank~19 to rank~5 (the three-trait
partition amplifies genetic architecture importance at finer spatial
scales), and $P_\text{env,max}$ rose from rank~11 to rank~4 (the
environmental reservoir becomes critical with more nodes seeding
independent epidemics). See Section~\ref{sec:sensitivity} for full
results.


\subsubsection{Full-Range Network (Planned)}
\label{sec:full_range}

Scaling analysis (Section~\ref{sec:validation}) demonstrated that the
model supports 150-node networks at $\sim$66\,s per 20-year run, enabling
a full NE~Pacific coastline simulation (Alaska to Baja California).
This configuration will use the precomputed 489-site overwater distance
matrix (Section~\ref{sec:distance_matrix}) and site-specific SST
forcing from satellite climatologies.


% ──────────────────────────────────────────────────────────────────────
\subsection{Network Construction}
\label{sec:build_network}

The \texttt{build\_network} function assembles the metapopulation from a
list of node definitions by: (i)~computing the pairwise distance matrix
(Haversine~$\times\,\tau$ or precomputed overwater distances);
(ii)~constructing $\mathbf{C}$ with per-node $\alpha_j$ values
($\alpha_\text{fjord}$ or $\alpha_\text{open}$), the $D_L$ kernel,
optional barrier factors, and row normalization to $r_\text{total}$;
(iii)~constructing $\mathbf{D}$ with the $D_P$ kernel, flushing-rate
modulation, sill attenuation, and the 50\,km cutoff; and (iv)~wrapping
each node definition in a \texttt{SpatialNode} with initialized
environmental state. The function accepts optional parameters for all
kernel scales, self-recruitment fractions, and barrier configurations,
allowing the same codebase to serve validation, sensitivity analysis, and
full-range simulation.
