% ══════════════════════════════════════════════════════════════════════
% METHODS — SPATIAL STRUCTURE
% Fact-checked against: sswd_evoepi/spatial.py, environment.py,
%   config.py, scripts/sensitivity/spatial_runner.py
% All parameter values verified against code defaults 2026-02-22
% ══════════════════════════════════════════════════════════════════════

\label{sec:spatial}

The model represents the NE~Pacific range of \pycshort{} as a
metapopulation network of discrete habitat nodes connected by larval
dispersal and waterborne pathogen transport. Each node maintains its own
population, disease state, and environmental forcing; inter-node coupling
occurs through annual larval exchange and daily pathogen dispersal.

\paragraph{Network topology.}
The primary network used for sensitivity analysis and validation is an
11-node stepping-stone chain spanning approximately 3{,}000\,km of
coastline from Sitka, Alaska (57.06\textdegree N) to Monterey,
California (36.62\textdegree N; Figure~\ref{fig:study_map}):
%
\begin{center}
\small
Sitka $\to$ Ketchikan $\to$ Haida Gwaii $\to$ Bella Bella $\to$
Howe Sound $\to$ San Juan Islands $\to$ Westport $\to$ Newport $\to$
Crescent City $\to$ Fort Bragg $\to$ Monterey
\end{center}
%
Adjacent nodes are separated by 111--452\,km, ensuring that the larval
dispersal kernel produces meaningful inter-node exchange across the
sensitivity analysis parameter range ($D_L \in [200, 600]$\,km). This
chain structure was adopted in sensitivity analysis Round~4 after
earlier 3-node configurations (spacing $>$1{,}700\,km) rendered
connectivity parameters effectively untestable, as
$\exp(-1700/400) < 10^{-2}$ produced negligible larval exchange.

\paragraph{Node-specific environmental forcing.}
Each node $k$ receives locally parameterized forcing along three axes.

\textit{Sea surface temperature.}\quad
Daily SST forcing uses 365-day climatological means derived from the
NOAA Optimum Interpolation SST version~2.1 dataset
\citep{oisstv21}, a 0.25\textdegree{} global daily product. For each
node, day-of-year means were computed from 24~years of monthly data
(2002--2025), accessed via NOAA PSL OPeNDAP, and interpolated to daily
resolution. The resulting climatologies capture real seasonal dynamics
---including asymmetric warming and cooling profiles, coastal upwelling
signatures at exposed sites (Newport, Crescent City), and the broad
warm season at sheltered sites (Howe Sound)---that a symmetric
sinusoidal approximation cannot represent. Annual means range from
$\sim$8.9\textdegree C at Sitka to $\sim$13.3\textdegree C at Monterey.
For projection scenarios, an optional linear warming trend $\gamma_k$
(\textdegree C\,yr$^{-1}$) is superimposed:
%
\begin{equation}
  T_k(d, y) = T_{k,\mathrm{clim}}(d)
    + \gamma_k\,(y - y_{\mathrm{ref}}),
  \label{eq:sst_satellite}
\end{equation}
%
where $T_{k,\mathrm{clim}}(d)$ is the satellite-derived climatological
SST for day-of-year~$d$ and $y_{\mathrm{ref}} = 2015$ is the reference
year.

\textit{Salinity.}\quad
Each node carries a fixed effective salinity ($S_k$, psu) that
modulates \vp{} viability through the quadratic ramp of
Equation~\ref{eq:salinity}. Open-coast nodes receive full-marine
salinities (30--33.5\,psu; $S_{\mathrm{sal}} \geq 0.87$), while
the fjord node Howe Sound has $S_k = 22$\,psu due to freshwater runoff,
yielding $S_{\mathrm{sal}} = 0.44$---a $\sim$56\% reduction in
effective Vibrio viability that provides a mechanistic basis for fjord
refugia \citep{montecino2016geographic}.

\textit{Flushing rate.}\quad
Hydrodynamic flushing removes waterborne pathogen at node-specific
rates spanning two orders of magnitude: $\phi_k = 0.5$--$0.8$\,d$^{-1}$
at open-coast sites, $\phi_k = 0.30$\,d$^{-1}$ at the semi-enclosed
San Juan Islands, and $\phi_k = 0.03$\,d$^{-1}$ at Howe Sound, where
the glacial sill restricts water exchange.

\paragraph{Temperature-dependent processes.}
SST drives three key rate processes through Arrhenius scaling
(Eq.~\ref{eq:arrhenius}): disease progression rates
(E$\to$I$_1$, I$_1$$\to$I$_2$, I$_2$$\to$D), pathogen environmental
persistence via the viable-but-not-culturable (VBNC) transition, and
spawning phenology. The $\sim$4.4\textdegree C latitudinal SST gradient
produces emergent north--south gradients in disease severity, consistent
with the observed southward-increasing SSWD mortality during the
2013--2015 outbreak \citep{harvell2019disease}.

\paragraph{Larval dispersal.}
Annual larval exchange between nodes is governed by a connectivity
matrix~$\mathbf{C}$ constructed from an exponential distance kernel
with explicit self-recruitment:
%
\begin{equation}
  C_{jk} =
  \begin{cases}
    \alpha_j & \text{if } j = k, \\[4pt]
    (1 - \alpha_j)\;
    \exp\!\left(-\dfrac{d_{jk}}{D_L}\right)
    & \text{if } j \neq k,
  \end{cases}
  \label{eq:dispersal_kernel}
\end{equation}
%
where $D_L = 400$\,km is the characteristic dispersal length scale,
$d_{jk}$ is the pairwise distance between nodes $j$ and $k$,
and $\alpha_j$ is the self-recruitment fraction. The dispersal scale
reflects a pelagic larval duration (PLD) of approximately 63~days
\citep{strathmann1987reproduction} and NE~Pacific current speeds of
5--20\,cm\,s$^{-1}$. Self-recruitment fractions differ by habitat type:
$\alpha_{\mathrm{fjord}} = 0.30$ for fjord nodes, encoding
sill-mediated circulation that traps larvae near their natal site
\citep{swearer2002evidence}, and $\alpha_{\mathrm{open}} = 0.10$ for
open-coast nodes. Rows of~$\mathbf{C}$ are normalized so that total
per-larva settlement probability equals $r_{\mathrm{total}} = 0.02$,
incorporating cumulative losses from pelagic mortality, failed
metamorphosis, and post-settlement predation. At the default
$D_L = 400$\,km, adjacent nodes (111--452\,km) exchange 32--76\% of
their non-self-recruiting larvae.

At the end of each reproductive season, competent larvae from each
source node are distributed to receiving nodes via
$\mathbf{C}$: a binomial draw determines total settlement, followed by
multinomial allocation across destinations proportional to the
conditional probabilities $C_{jk} / \sum_k C_{jk}$.

\paragraph{Pathogen dispersal.}
Daily waterborne pathogen exchange operates at much shorter range than
larval transport. The pathogen dispersal matrix~$\mathbf{D}$ uses an
exponential kernel with scale $D_P = 15$\,km (reflecting tidal-current
transport), modulated by the source node's flushing rate and a sill
attenuation factor for fjord nodes. Pairs beyond 50\,km receive zero
pathogen transfer. Low flushing in fjords thus acts as a double-edged
mechanism: it reduces pathogen \emph{removal} (increasing local
concentrations) while also reducing pathogen \emph{export} to
neighboring nodes, effectively isolating fjords from regional epidemic
dynamics.
