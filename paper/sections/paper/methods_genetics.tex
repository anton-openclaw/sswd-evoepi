% ══════════════════════════════════════════════════════════════════════
% METHODS — GENETIC ARCHITECTURE
% Fact-checked against: sswd_evoepi/genetics.py, types.py, config.py
%   (GeneticsSection), reproduction.py (mendelian_inherit_batch,
%   srs_reproductive_lottery), three-trait-genetic-architecture-spec.md
% All parameter values verified against code defaults 2026-02-22
% ══════════════════════════════════════════════════════════════════════

\label{sec:genetics}

Each individual carries a diploid genotype represented as a
$(51 \times 2)$ array of biallelic loci, where each allele is either
ancestral~(0) or derived~(1). The 51 loci are motivated by
\citet{schiebelhut2018collapse}, who identified $\sim\!51$ loci with
significant allele frequency shifts in \textit{Pisaster ochraceus}
survivors of the 2013--2015 SSWD outbreak---the closest available
genomic proxy for selection response in a wasting-affected asteroid. A
reference genome for \pycshort{} is now available
\citep{Schiebelhut2024genome}, but no species-specific GWAS data yet
distinguish among categories of immune loci.

Loci are partitioned into three contiguous blocks encoding distinct
defense traits (Table~\ref{tab:trait_partition_paper}):

\begin{table}[H]
\centering
\caption{Three-trait genetic architecture. The partition is
configurable ($n_R + n_T + n_C = 51$); the default $17/17/17$ split
is used in all analyses.}
\label{tab:trait_partition_paper}
\small
\begin{tabular}{lccl}
\toprule
\textbf{Trait} & \textbf{Symbol} & \textbf{Loci (indices)} &
  \textbf{Mechanistic role} \\
\midrule
Resistance & $r_i$ & 17 \;(0--16) &
  Immune exclusion: reduces infection probability \\
Tolerance  & $t_i$ & 17 \;(17--33) &
  Damage limitation: extends I$_2$ survival time \\
Recovery   & $c_i$ & 17 \;(34--50) &
  Pathogen clearance: daily recovery probability \\
\bottomrule
\end{tabular}
\end{table}

\noindent
These three traits represent biologically distinct immune strategies
operating at different points in the infection process
\citep{raberg2009decomposing}. Resistance prevents pathogen
establishment via receptor polymorphisms and barrier defenses, reducing
both individual risk and population-level transmission. Tolerance
extends host survival during symptomatic infection through tissue repair
and metabolic compensation, but does not reduce pathogen shedding---tolerant
individuals are epidemiological ``silent spreaders.'' Recovery enables
active pathogen clearance via coelomocyte-mediated phagocytosis,
returning the host to the susceptible pool. The equal $17/17/17$
partition is a simplifying assumption; the partition ratio is included
as a sensitivity analysis parameter
(Section~\ref{sec:sensitivity_analysis}).


\paragraph{Trait score computation.}
Each locus $\ell$ within a trait block carries a fixed effect size
$e_\ell > 0$. Effect sizes are drawn independently for each trait from
an exponential distribution, $e_\ell \sim \mathrm{Exp}(1)$, normalized
to sum to unity within each block, and sorted in descending order. This
produces an $L$-shaped effect-size distribution in which a few loci of
large effect coexist with many loci of small effect, consistent with
empirical quantitative trait locus architectures
\citep{Lotterhos2015}. A fixed random seed ensures identical effect
sizes across replicate simulations.

An individual's score for trait $\theta \in \{r,\, t,\, c\}$ is the
effect-weighted mean allele dosage across the corresponding locus set
$\mathcal{L}_\theta$:
%
\begin{equation}
  \theta_i
  = \sum_{\ell \in \mathcal{L}_\theta}
    e_\ell \;\frac{g_{\ell,0} + g_{\ell,1}}{2}\,,
  \label{eq:trait_score_paper}
\end{equation}
%
where $g_{\ell,0}, g_{\ell,1} \in \{0,1\}$ are the two allele copies.
Because $\sum_\ell e_\ell = 1$ within each block and the mean allele
dosage per locus is bounded by $[0,1]$, all trait scores satisfy
$\theta_i \in [0,1]$. Inheritance is purely additive within and across
loci; there is no dominance or epistasis.


\paragraph{Genotype initialization.}
Initial per-locus allele frequencies are drawn from a Beta distribution,
$q_\ell \sim \mathrm{Beta}(2,\, 8)$, producing a right-skewed frequency
spectrum (mean~$0.2$) in which most derived alleles are rare---consistent
with standing variation in immune genes maintained by mutation--selection
balance. The raw frequencies are rescaled per trait so that the expected
population-mean trait score matches a configurable target:
$\bar{r} = 0.15$, $\bar{t} = 0.10$, $\bar{c} = 0.02$. Recovery is
initialized with the lowest mean because active pathogen clearance is
assumed to be the rarest pre-epidemic phenotype. Per-locus frequencies
are clipped to $[0.001, 0.5]$ to prevent fixation or majority-derived
states at initialization. Each allele copy is then sampled independently
as a Bernoulli trial with success probability $q_\ell$, establishing
Hardy--Weinberg equilibrium at each locus.


\paragraph{Mendelian inheritance.}
At reproduction, offspring receive one randomly chosen allele from each
parent at every locus, with free recombination (independent assortment,
no linkage). Allele choices for all $n_\mathrm{offspring} \times 51
\times 2$ positions are drawn simultaneously and applied via vectorized
indexing into parental genotype arrays.


\paragraph{Mutation.}
Bidirectional point mutations ($0 \to 1$ or $1 \to 0$) are applied to
offspring genotypes at a rate of $\mu = 10^{-8}$ per allele per
generation \citep{Lynch2010}. The total number of mutations per cohort
is Poisson-distributed,
$n_\mathrm{mut} \sim \mathrm{Pois}(\mu \times n_\mathrm{offspring}
\times 51 \times 2)$, and each mutation is placed at a uniformly random
allele position. At this rate, mutations are negligible over the 20--100
year simulation horizon ($\sim\!10^{-6}$ expected mutations per
offspring), and evolution proceeds almost entirely through selection on
standing genetic variation.


\paragraph{Coupling to disease dynamics.}
Each trait feeds into a single mechanistic point in the disease module
(Section~\ref{sec:force_of_infection}--\ref{sec:defense_traits}):
resistance reduces the force of infection via $(1 - r_i)$ in
Eq.~\ref{eq:foi}; tolerance extends I$_2$ survival by scaling the
disease mortality rate as
$\mu_{I_2 D,i}^\mathrm{eff} = \mu_{I_2 D}(T)(1 - t_i\,\tau_\max)$
with $\tau_\max = 0.85$ and a 5\% floor
(Eq.~\ref{eq:tolerance}); and recovery determines the daily clearance
probability $p_{\mathrm{rec}} = \rho_\mathrm{rec} \times c_i$ with
$\rho_\mathrm{rec} = 0.05$\,d$^{-1}$ (Eq.~\ref{eq:recovery}). Each
trait does exactly one thing, ensuring clean separation of selective
pressures.

No cost of resistance is imposed: fecundity depends solely on body
size (Section~\ref{sec:fecundity_paper}). This decision reflects the
absence of empirical evidence for a measurable fecundity penalty
associated with disease-resistance alleles in \pycshort{}.
