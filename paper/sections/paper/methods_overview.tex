% ══════════════════════════════════════════════════════════════════════
% METHODS — MODEL OVERVIEW
% Fact-checked against: sswd_evoepi/model.py, config.py, types.py
% ══════════════════════════════════════════════════════════════════════

\modelname{} is an individual-based, spatially explicit simulation in which
each agent represents a single \pyc{} tracked through its complete life
history. Agents are characterized by continuous state variables---body size,
age, spatial position, disease compartment, and a diploid genotype at 51
loci encoding three quantitative defense traits (resistance, tolerance,
recovery)---updated at daily resolution. We chose an individual-based
approach because SSWD outcomes depend on the joint distribution of body
size, genotype, and spatial location within each host, and because
evolutionary rescue requires explicit tracking of heritable genetic
variation across generations \citep{deangelis2005individual,
clement2024coevolution}.

The model couples four mechanistic modules (Figure~\ref{fig:conceptual}):
%
\begin{enumerate}[nosep]
  \item \textbf{Disease dynamics} --- a stochastic SEIPD compartmental
    framework driven by waterborne \vp{} concentration, with
    temperature-dependent progression rates calibrated to experimental
    data \citep{prentice2025kochs};
  \item \textbf{Genetic architecture} --- 51 diploid loci partitioned
    $17/17/17$ into resistance (immune exclusion), tolerance (damage
    limitation), and recovery (pathogen clearance) traits, based on the
    number of loci under selection identified by \citet{schiebelhut2018collapse};
  \item \textbf{Population ecology} --- von~Bertalanffy growth,
    stage-structured natural mortality, multi-bout spawning with
    post-spawning immunosuppression, and Beverton--Holt
    density-dependent recruitment;
  \item \textbf{Spatial structure} --- a metapopulation network of
    habitat nodes connected by distance-dependent larval dispersal
    ($D_L = 400$\,km kernel) and waterborne pathogen exchange
    ($D_P = 15$\,km kernel).
\end{enumerate}

The simulation advances in daily time steps nested within an annual
cycle. Each day, the following operations execute sequentially at every
node: (i)~environmental forcing (sea surface temperature from NOAA OISST
v2.1 climatologies, constant salinity, seasonal flushing); (ii)~agent
movement via a correlated random walk with disease-state--dependent speed
modifiers; (iii)~disease transmission, progression, and recovery;
(iv)~waterborne pathogen dispersal among neighboring nodes; and
(v)~daily demographics (natural mortality, somatic growth, spawning
during the November--July season). At the end of each simulated year, a
dispersal step redistributes competent larvae among nodes via the
connectivity matrix, and genetic summary statistics (allele frequencies,
trait means, additive genetic variance) are recorded. Disease is
introduced at a configurable epidemic year by seeding exposed individuals
at each node.

\begin{figure}[t]
  \centering
  % \includegraphics[width=\textwidth]{figures/conceptual_diagram.pdf}
  \fbox{\parbox{0.9\textwidth}{\centering\vspace{2cm}%
    \textbf{Figure placeholder:} Conceptual diagram showing the four
    coupled modules, daily simulation loop, and key feedback
    pathways.\vspace{2cm}}}
  \caption{Conceptual overview of the \modelname{} framework. Arrows
    indicate directional coupling between modules. The daily loop
    (inner ring) resolves disease, movement, and mortality; the annual
    cycle (outer ring) resolves reproduction, larval dispersal, and
    genetic recording. Eco-evolutionary feedbacks arise because
    genetically determined defense traits modulate disease outcomes,
    which in turn impose selection that shifts allele frequencies
    across generations.}
  \label{fig:conceptual}
\end{figure}
