% ══════════════════════════════════════════════════════════════════════
% METHODS — DISEASE DYNAMICS
% Fact-checked against: sswd_evoepi/disease.py, config.py
% All parameter values verified against code defaults 2026-02-22
% ══════════════════════════════════════════════════════════════════════

Disease dynamics follow a stochastic SEIPD compartmental framework
embedded within the individual-based model. Each agent occupies one of
five disease states: Susceptible~(S), Exposed~(E), Infectious stage~1
(I$_1$, pre-symptomatic), Infectious stage~2 (I$_2$, symptomatic
wasting), or Dead from disease~(D). Recovery is possible from both I$_1$
and I$_2$, returning individuals to the susceptible pool (R~$\to$~S).
We model no acquired immunity because echinoderms lack an adaptive
immune system; sea stars treated for wasting have subsequently developed
the disease again, confirming the absence of immunological memory.
Heritable genetic traits---resistance, tolerance, and recovery---provide
the only defense mechanism.


\subsubsection{Force of infection}
\label{sec:force_of_infection}

Transmission is environmentally mediated through a waterborne pathogen
pool rather than direct contact. The per-individual instantaneous hazard
rate of infection is:
%
\begin{equation}
  \lambda_i
  = a
    \;\frac{P_k}{K_{1/2} + P_k}
    \;(1 - r_i)
    \;S_{\mathrm{sal}}
    \;f_{\mathrm{size}}(L_i),
  \label{eq:foi}
\end{equation}
%
where $a = 0.75$\,d$^{-1}$ is the baseline exposure rate,
$P_k$~(bacteria\,mL$^{-1}$) is the local \vp{} concentration at
node~$k$, $K_{1/2} = 87{,}000$\,bacteria\,mL$^{-1}$ is the
half-saturating dose (Michaelis--Menten dose--response), $r_i \in [0,1]$
is the individual's genetically determined resistance score
(Section~\ref{sec:genetics}), $S_{\mathrm{sal}}$ is a salinity modifier,
and $f_{\mathrm{size}}$ is a size-dependent susceptibility modifier.
The discrete daily infection probability is
$p_{\mathrm{inf}} = 1 - \exp(-\lambda_i \,\Delta t)$ with
$\Delta t = 1$\,d.

The salinity modifier suppresses \vp{} viability in low-salinity
waters, providing a mechanistic basis for the reduced SSWD prevalence
observed in fjord systems:
%
\begin{equation}
  S_{\mathrm{sal}} =
  \begin{cases}
    0 & S \leq 10\;\text{psu}, \\[3pt]
    \bigl(\frac{S - 10}{28 - 10}\bigr)^{2} & 10 < S < 28\;\text{psu}, \\[3pt]
    1 & S \geq 28\;\text{psu}.
  \end{cases}
  \label{eq:salinity}
\end{equation}

Size-dependent susceptibility follows \citet{eisenlord2016ochre}, who
reported an odds ratio of 1.23 per 10\,mm increase in radius:
%
\begin{equation}
  f_{\mathrm{size}}(L_i) = \exp\!\Bigl(
    \beta_L \,\frac{L_i - \bar{L}}{\sigma_L}
  \Bigr),
  \quad
  \beta_L = \frac{\ln 1.23}{10} \approx 0.021\;\text{mm}^{-1},
  \label{eq:size}
\end{equation}
%
with reference size $\bar{L} = 300$\,mm and normalization
$\sigma_L = 100$\,mm.

Following each spawning event, an individual enters a 28-day
immunosuppression window during which its effective resistance is halved
($r_{i,\mathrm{eff}} = r_i / \psi$, $\psi = 2.0$, clamped to
$[0,1]$), creating an evolutionary coupling between reproductive
investment and disease vulnerability.


\subsubsection{Disease progression}
\label{sec:progression}

Stage durations are drawn from Erlang distributions rather than
memoryless exponentials, producing more realistic peaked duration
profiles \citep{wearing2005appropriate}. When an individual enters
compartment $X$, a countdown timer is sampled:
%
\begin{equation}
  \tau_X \sim \mathrm{Erlang}(k_X,\; k_X \,\mu_X(T)),
  \quad \text{rounded to } \max(1, \mathrm{round}(\tau_X)) \;\text{days},
  \label{eq:erlang}
\end{equation}
%
where $k_X$ is the shape parameter (controlling regularity) and
$\mu_X(T)$ is the temperature-dependent transition rate. Shape
parameters are $k_E = 3$ (CV~$= 0.58$), $k_{I_1} = 2$ (CV~$= 0.71$),
and $k_{I_2} = 2$ (CV~$= 0.71$). Timers decrement by one each day;
upon reaching zero, the agent transitions to the next compartment.

All transition rates are temperature-scaled via the Arrhenius equation:
%
\begin{equation}
  \mu_X(T) = \mu_{X,\mathrm{ref}} \,\exp\!\left[
    \frac{E_{a,X}}{R} \left(
      \frac{1}{T_{\mathrm{ref}}} - \frac{1}{T}
    \right)
  \right],
  \label{eq:arrhenius}
\end{equation}
%
with reference temperature
$T_{\mathrm{ref}} = 293.15$\,K ($20\,^\circ$C), corresponding to the
\vp{} thermal growth optimum \citep{lambert1998virulence}. The
reference rates and activation energies are:
%
\begin{align}
  \mu_{E \to I_1,\mathrm{ref}}   &= 0.233\;\text{d}^{-1},
    &\quad E_a/R &= 4{,}000\;\text{K},
    \label{eq:mu_EI1} \\
  \mu_{I_1 \to I_2,\mathrm{ref}} &= 0.434\;\text{d}^{-1},
    &\quad E_a/R &= 5{,}000\;\text{K},
    \label{eq:mu_I1I2} \\
  \mu_{I_2 \to D,\mathrm{ref}}   &= 0.563\;\text{d}^{-1},
    &\quad E_a/R &= 2{,}000\;\text{K}.
    \label{eq:mu_I2D}
\end{align}
%
These values were calibrated to the experimental disease time course of
\citet{prentice2025kochs}, who established Koch's postulates for \vp{}
and reported a mean exposure-to-death interval of 11.6\,days and a mean
symptoms-to-death interval of 5.6\,days at $\sim\!13\,^\circ$C. Our
Arrhenius-corrected rates reproduce these targets: at $13\,^\circ$C, the
mean stage durations are 6.0\,d~(E), 3.5\,d~(I$_1$), and
2.1\,d~(I$_2$), summing to 11.6\,d total with 5.6\,d from first
symptoms to death. The notably lower activation energy for
$\mu_{I_2 \to D}$ ($E_a/R = 2{,}000$\,K vs.\ $4{,}000$--$5{,}000$\,K
for earlier transitions) reflects evidence that terminal tissue
degradation is less temperature-sensitive than the initial stages of
infection establishment.


\subsubsection{Host defense traits}
\label{sec:defense_traits}

Three genetically determined traits modulate individual disease
outcomes, each operating at a distinct point in the infection process:

\paragraph{Resistance ($r_i$).}
Immune exclusion reduces the force of infection via the $(1 - r_i)$
term in Eq.~\ref{eq:foi}. An individual with $r_i = 0.5$ has half the
baseline infection hazard. Resistance acts before infection and therefore
also reduces population-level pathogen pressure by lowering the number of
shedding hosts.

\paragraph{Tolerance ($t_i$).}
Damage limitation extends survival during the terminal I$_2$ stage by
reducing the effective I$_2 \to$ D rate:
%
\begin{equation}
  \mu_{I_2 \to D,\mathrm{eff}}
  = \mu_{I_2 \to D}(T)
    \times (1 - t_i \,\tau_{\max}),
  \quad
  \text{floored at}\; 0.05 \times \mu_{I_2 \to D}(T),
  \label{eq:tolerance}
\end{equation}
%
where $\tau_{\max} = 0.85$ sets the maximum mortality reduction at
$t_i = 1$. The 5\% floor prevents biologically implausible indefinite
survival. By extending I$_2$ duration, tolerance provides more
opportunities for recovery but also prolongs pathogen shedding---a
key epidemiological tradeoff.

\paragraph{Recovery ($c_i$).}
Pathogen clearance enables return to the susceptible pool. Each day,
an I$_2$ individual recovers with probability:
%
\begin{equation}
  p_{\mathrm{rec},I_2} = \rho_{\mathrm{rec}} \times c_i,
  \qquad
  \rho_{\mathrm{rec}} = 0.05\;\text{d}^{-1}.
  \label{eq:recovery}
\end{equation}
%
Early recovery from I$_1$ is possible only for individuals with
exceptionally high clearance ability ($c_i > 0.5$), at a reduced
probability $p_{\mathrm{rec},I_1} = \rho_{\mathrm{rec}} \times
2\,(c_i - 0.5)$. Recovered individuals are immediately susceptible to
reinfection.


\subsubsection{Environmental pathogen dynamics}
\label{sec:pathogen_dynamics}

The waterborne \vp{} concentration $P_k$ at node~$k$ evolves according
to:
%
\begin{equation}
  \frac{dP_k}{dt}
  = \underbrace{\sigma_1(T)\,n_{I_1} + \sigma_2(T)\,n_{I_2}
    + \sigma_D \,n_{D,\mathrm{fresh}}}_{\text{shedding}}
  - \underbrace{\xi(T)\,P_k}_{\text{decay}}
  - \underbrace{\phi_k \,P_k}_{\text{flushing}}
  + \underbrace{P_{\mathrm{env}}(T, S)}_{\text{reservoir}}
  + \underbrace{\textstyle\sum_j d_{jk} P_j}_{\text{dispersal}},
  \label{eq:vibrio}
\end{equation}
%
integrated via forward Euler ($\Delta t = 1$\,d), with $P_k \geq 0$.
Shedding rates from infectious hosts are temperature-dependent via
Arrhenius scaling ($E_a/R = 5{,}000$\,K): $\sigma_1 = 5.0$ and
$\sigma_2 = 50.0$\,bacteria\,mL$^{-1}$\,d$^{-1}$\,host$^{-1}$ at
$T_{\mathrm{ref}}$. The 10-fold increase from I$_1$ to I$_2$ reflects
the dramatic escalation of tissue degradation during wasting. Carcasses
(D~compartment) shed at a constant rate
$\sigma_D = 15$\,bacteria\,mL$^{-1}$\,d$^{-1}$\,carcass$^{-1}$ for a
3-day saprophytic window, tracked via a ring buffer of daily disease
deaths.

The natural decay rate $\xi(T)$ is interpolated log-linearly between
empirical estimates: $\xi = 1.0$\,d$^{-1}$ (half-life $\approx 0.7$\,d)
at $10\,^\circ$C and $\xi = 0.33$\,d$^{-1}$ (half-life
$\approx 2.1$\,d) at $20\,^\circ$C, reflecting the counter-intuitive
pattern of faster Vibrio decay at lower temperatures
\citep{lupo2020vibrionaceae}.

In the default \emph{ubiquitous} scenario, a background environmental
reservoir represents \vp{} persisting as viable-but-non-culturable
(VBNC) cells in sediments:
%
\begin{equation}
  P_{\mathrm{env}}(T, S)
  = P_{\mathrm{env,max}}
    \;\frac{1}{1 + e^{-\kappa(T - T_{\mathrm{VBNC}})}}
    \;g(T)
    \;S_{\mathrm{sal}},
  \label{eq:reservoir}
\end{equation}
%
where $P_{\mathrm{env,max}} = 500$\,bacteria\,mL$^{-1}$\,d$^{-1}$,
$\kappa = 1.0\,^\circ$C$^{-1}$, and
$T_{\mathrm{VBNC}} = 12\,^\circ$C. The thermal performance function
$g(T)$ follows an Arrhenius increase below $T_{\mathrm{opt}} =
20\,^\circ$C with quadratic decline above, reaching zero at
$T_{\max} = 30\,^\circ$C. This formulation produces near-zero pathogen
input during winter and a summer peak that triggers seasonal epidemics,
consistent with the observed pattern of SSWD outbreaks during warm-water
anomalies.
