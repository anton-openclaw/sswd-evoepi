% ══════════════════════════════════════════════════════════════════════
% METHODS — MODEL CALIBRATION
% Fact-checked against:
%   specs/calibration-validation-plan.md    (ABC-SMC protocol)
%   results/sensitivity_r4/MORRIS_R4_ANALYSIS.md (top-10 params)
%   MEMORY.md                               (calibration strategy)
% Note: Calibration is PLANNED, not yet executed.
% ══════════════════════════════════════════════════════════════════════

Model calibration follows an Approximate Bayesian Computation with
Sequential Monte Carlo sampling (ABC-SMC; \citealp{Toni2009})
approach, chosen because the individual-based model has no
closed-form likelihood function. ABC-SMC avoids likelihood
evaluation by comparing simulated summary statistics to empirical
targets, accepting parameter combinations that produce sufficiently
similar outputs. The sequential refinement of acceptance thresholds
concentrates sampling in high-posterior regions while maintaining
computational efficiency.

\subsubsection{Calibration parameters}
\label{sec:calib_params}

Calibration focuses on the ${\sim}$10 parameters identified by
Morris screening as highly influential (normalized
$\mu^* > 0.4$) that simultaneously lack strong empirical
constraints. These include the recovery rate coefficient
($\rho_\mathrm{rec}$), half-saturating pathogen dose ($K_{1/2}$),
environmental pathogen input ($P_\mathrm{env,max}$), settler
survival ($s_0$), symptomatic shedding rate
($\sigma_{2,\mathrm{eff}}$), von~Bertalanffy growth rate
($k_\mathrm{growth}$), and the initial mean recovery trait
($\bar{c}_0$). Parameters with well-constrained literature values
(e.g., disease progression rates calibrated to
\citealp{prentice2025kochs}) are fixed at their reference values.
Prior distributions for calibrated parameters are uniform over the
ranges used in the sensitivity analysis, informed by a systematic
literature review of 103~sources
(Supplementary~\ref{sec:appendix_parameters}).

\subsubsection{Calibration targets}
\label{sec:calib_targets}

The calibration targets consist of five summary statistics
derived from empirical observations of the 2013--2017 SSWD
epizootic:
%
\begin{enumerate}[nosep]
  \item Population decline magnitude: 80--99\% crash across the
    species range \citep{harvell2019disease,
    montecino2016sunflower};
  \item Timeline from pathogen introduction to population nadir:
    2--5 years \citep{montecino2016sunflower};
  \item North--south mortality gradient: southern populations
    experienced more severe declines
    \citep{hamilton2021nearshore};
  \item Fjord and semi-enclosed water refugia: higher survival
    in protected waters relative to open coast
    \citep{hamilton2021nearshore};
  \item Allele frequency shift at immune-associated loci:
    $\Delta q = 0.08$--0.15 \citep{schiebelhut2018collapse}.
\end{enumerate}
%
The distance between simulated and observed outcomes is computed as
a weighted sum of absolute deviations, normalized by the empirical
range of each statistic:
%
\begin{equation}
  d(\boldsymbol{\theta})
  = \sum_{i=1}^{5} w_i \,
    \frac{\bigl|S_i^{\mathrm{sim}}(\boldsymbol{\theta})
          - S_i^{\mathrm{obs}}\bigr|}{\sigma_i},
  \label{eq:abc_distance}
\end{equation}
%
where $S_i^{\mathrm{sim}}$ and $S_i^{\mathrm{obs}}$ are simulated
and observed summary statistics, $\sigma_i$ is a normalization
constant (empirical range or standard deviation), and $w_i$ is a
weight reflecting constraint quality. Well-quantified targets
(population crash magnitude, allele frequency shift) receive higher
weight ($w = 1.0$) than qualitative constraints (gradient sign,
refugia effect; $w = 0.5$).

\subsubsection{ABC-SMC protocol}
\label{sec:abc_protocol}

The ABC-SMC algorithm proceeds through $T = 5$--8 sequential
populations of $N_\mathrm{particles} = 1{,}000$ parameter
vectors. The initial acceptance threshold~$\varepsilon_1$ is set
at the 75th percentile of distances from a prior-predictive
sample, and is reduced by approximately 50\% at each subsequent
population until the acceptance rate falls below 1\% or
$\varepsilon$ stabilizes (change ${<}5$\% between consecutive
populations). Component-wise uniform perturbation kernels with
adaptive widths \citep{Beaumont2009} maintain particle diversity
across populations. Each parameter vector is evaluated using
three independent random seeds at the calibration scale
($K = 5{,}000$ per node), with the median distance across seeds
used to reduce stochastic noise.

Total computational cost is estimated at 10{,}000--50{,}000
forward simulations. Calibration is performed at $K = 5{,}000$
per node; posterior parameter estimates are subsequently validated
at $K = 100{,}000$ per node to verify scale-independence of the
fitted dynamics. The calibration framework uses the pyABC Python
library \citep{Klinger2018}.
