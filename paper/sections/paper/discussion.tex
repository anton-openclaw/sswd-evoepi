% ══════════════════════════════════════════════════════════════════════
% DISCUSSION — Publication Version
% Derived from work report discussion.tex; condensed and restructured
% for journal submission.
% ══════════════════════════════════════════════════════════════════════

The central prediction of \modelname{} is stark: \textit{Pycnopodia
helianthoides} populations crash by $>$99\% under every model
configuration examined, regardless of population scale, SST forcing
scheme, or spatial network topology. This result is not an artifact of
parameter tuning---the four-round sensitivity analysis demonstrates
that catastrophic decline is a robust emergent property of the
coupled eco-evolutionary system. Here we interpret this finding in the
context of echinoderm immunology, evolutionary rescue theory, and
conservation planning for captive-bred reintroduction.


% ──────────────────────────────────────────────────────────────────────
\subsection{Key findings}
\label{sec:disc_findings}

Three results merit particular emphasis.

First, the reinfection correction (R$\to$S) transforms the model's
evolutionary predictions. Under the (incorrect) assumption of
permanent post-recovery immunity, recovery ($c_i$) was the
fastest-evolving trait at every node, with Monterey showing
$\Delta c_i = +0.154$ over 20 years. Under the biologically correct
R$\to$S formulation---where recovered individuals return to the
susceptible pool---recovery evolution effectively ceases
($\overline{\Delta c_i} \approx +0.002$, indistinguishable from drift)
and selection shifts decisively to resistance. At Sitka,
$\Delta r_i$ increases 5.5-fold ($+0.011 \to +0.060$) when
reinfection is permitted. The mechanism is intuitive: when clearing an
infection confers no lasting protection, avoiding infection entirely
becomes the only viable adaptive pathway. This finding aligns with
echinoderm immunology---lacking adaptive immune systems,
echinoderms have no mechanism for immunological memory
\citep{Mydlarz2006,Smith2010}---and calls into question any marine
invertebrate disease model that assumes permanent acquired immunity.

Second, the base recovery rate ($\rho_{\mathrm{rec}}$) dominates
the sensitivity analysis across all four rounds, all 23 output
metrics, and all spatial configurations. Its mean normalized
$\mu^*$ of 0.889 exceeds the second-ranked parameter
($k_{\mathrm{growth}}$, 0.633) by 41\%. Yet this parameter has zero
empirical basis: whether \pyc{} can clear \textit{Vibrio pectenicida}
infections at all remains unknown. The SA thus identifies the single
highest-priority empirical question for constraining model
predictions: controlled challenge-recovery experiments in captive
\pyc{} \citep{Prentice2025}.

Third, larger populations fare no better than small ones.
Scaling carrying capacity 20-fold ($K = 5{,}000 \to 100{,}000$)
increases the metapopulation crash from 98.5\% to 98.9\%, because
deterministic epidemic dynamics dominate at large $N$, suppressing
the demographic stochasticity that occasionally permits small
populations to escape through random fluctuations. Stochastic
rescue does not scale---a finding with direct implications for
reintroduction programs that might assume larger release cohorts
will improve outcomes through demographic mass alone.


% ──────────────────────────────────────────────────────────────────────
\subsection{The R$\to$S paradigm shift}
\label{sec:disc_rs}

The assumption of permanent post-recovery immunity is ubiquitous in
epidemiological models of marine wildlife disease
\citep{Aalto2020,Gimenez-Romero2021}, yet it is biologically
unjustified for echinoderms. Asteroids rely exclusively on innate
immune defenses---coelomocyte-mediated phagocytosis, complement-like
lectins, and antimicrobial peptides \citep{Smith2010}---which lack
the clonal expansion and memory cell formation that underpin acquired
immunity in vertebrates. The assumption of permanent immunity was
expedient in earlier SIR-type models where individual genetic
identity is not tracked, but in an individual-based framework where
genotype-dependent resistance, tolerance, and recovery are
explicitly modeled, the immunological assumption becomes a first-order
determinant of evolutionary dynamics.

The R$\to$S correction has consequences beyond trait evolution.
Final population size drops by 67\% (365 $\to$ 122 survivors),
two of five nodes reach local extinction (vs.\ zero under permanent
immunity), and fewer total recoveries occur (276 vs.\ 365)---not
because recovery is rarer per infection, but because faster
population collapse leaves fewer individuals to recover. The
epidemic is more severe precisely because each recovered individual
re-enters the susceptible pool rather than being removed from the
transmission chain.

This result has broader implications for marine invertebrate disease
modeling. Sea urchin mass mortality events
\citep{Hewson2023,Clouse2022}, coral tissue loss disease
\citep{Meiling2021}, and abalone withering syndrome
\citep{Friedman2000} all involve taxa that lack adaptive immunity.
Models of these systems should explicitly address whether permanent
immunity is a defensible assumption, or whether R$\to$S dynamics
fundamentally alter predictions---as they do here.


% ──────────────────────────────────────────────────────────────────────
\subsection{Comparison with other eco-evolutionary disease models}
\label{sec:disc_comparison}

The closest methodological precedent for \modelname{} is the
eco-evolutionary IBM developed by \citet{clement2024coevolution} for
coevolution between Tasmanian devils (\textit{Sarcophilus harrisii})
and devil facial tumour disease (DFTD). Both models track individual
diploid genotypes, couple epidemiological dynamics with quantitative
genetic evolution, and ask whether evolutionary rescue can avert host
extinction following a novel disease introduction. However, the
systems diverge in three ways that produce fundamentally different
predictions.

\paragraph{Reproductive biology.}
Devils are iteroparous mammals with small litters and high maternal
investment. \pyc{} is a broadcast spawner producing $\sim$$10^7$
eggs per female, subject to sweepstakes reproductive success (SRS)
with $N_e/N \sim 10^{-3}$ \citep{Hedgecock2011}. SRS amplifies
genetic drift at the population level while creating the potential
for rapid frequency shifts at individual loci under strong selection
\citep{Eldon2024}---a reproductive mode absent from the Clement
et al.\ framework. This produces a paradox: the mechanism that
enables occasional rapid adaptation also reduces the efficacy of
selection relative to drift across most of the genome.

\paragraph{Pathogen transmission.}
DFTD is a transmissible cancer requiring direct physical contact.
\vpshort{} transmits environmentally through waterborne bacteria and
is maintained by a multi-species reservoir ($P_{\mathrm{env}}$) that
decouples pathogen persistence from \pyc{} population size. This
decoupling weakens the virulence--transmission tradeoff that enables
coevolutionary stabilization in the DFTD system: in our model, the
environmental reservoir sustains infection pressure even as host
populations collapse, preventing the pathogen attenuation that
\citet{clement2024coevolution} found critical for devil persistence.

\paragraph{Evolutionary rescue prospects.}
Clement et al.\ found a high probability of devil persistence over
50 generations ($\sim$150 years), driven by rapid host--pathogen
coevolution. Our model predicts no recovery to $>$5\% of carrying
capacity within 20 years ($\sim$4 \pyc{} generations) at any scale.
This contrast reflects the mismatch between \pyc{}'s long generation
time ($\sim$5 years vs.\ $\sim$3 years for devils), the extreme
$N_e/N$ depression under SRS, and the environmental pathogen
reservoir that maintains infection pressure independently of host
genetic composition.

Other marine disease models have addressed components of the SSWD
system in isolation: \citet{Aalto2020} modeled ocean-scale
epidemiological dynamics without genetics, \citet{Gimenez-Romero2021}
developed SIRP compartmental models for \textit{Pinna nobilis}
without spatial structure, and \citet{Arroyo-Esquivel2025} modeled
reintroduction epidemiology without evolution. \modelname{} integrates
these dimensions---individual-based genetics, spatially explicit
metapopulation dynamics, and coupled eco-evolutionary feedback---within
a single framework, enabling the emergent interactions among these
processes to be studied jointly rather than in isolation.


% ──────────────────────────────────────────────────────────────────────
\subsection{Conservation implications for reintroduction}
\label{sec:disc_conservation}

The model's central finding---that natural selection on polygenic
resistance cannot drive population recovery on conservation-relevant
timescales---has a direct practical implication: waiting for natural
evolution is not a viable recovery strategy. Active intervention
through captive breeding and managed release is essential. The AZA
SAFE program's captive population of $>$2{,}500 juveniles and
130+ reproductive adults \citep{AZA2024}, combined with the
progressive outplanting trials from 2023 caged experiments through
the 2024 uncaged release \citep{kuow2024seastar} to the December
2025 California outplanting \citep{ssl2025outplanting}, provides
the demographic foundation for such intervention.

The R$\to$S finding reframes the optimal breeding strategy. Under
permanent immunity, selecting for high recovery ($c_i$) was rational:
recovered individuals survived to reproduce. Under reinfection,
resistance ($r_i$) becomes the dominant adaptive response. Captive
breeding programs should prioritize individuals that resist infection
entirely, identifiable through challenge experiments and, as the
\pyc{} reference genome becomes annotated
\citep{Schiebelhut2024genome}, genome-wide association with
resistance loci. A combined strategy---selecting for high resistance
with moderate recovery as a secondary trait---may be optimal.

The sensitivity analysis provides further guidance. The identification
of recovery rate, growth rate, settler survival, and environmental
pathogen pressure as the top-ranked parameters suggests that
reintroduction success depends on the intersection of host biology
and site-level disease environment. Release site selection should
consider local pathogen pressure (proxied by $P_{\mathrm{env}}$),
temperature regime (which modulates disease progression rates),
connectivity to neighboring populations (for demographic rescue via
larval exchange), and seasonal timing relative to spawning windows
(when immunosuppression may elevate susceptibility).

A comprehensive conservation scenario module---simulating specific
release strategies with optimized timing, location, genetic
composition, and cohort size---is a natural extension of this work
and is under active development. The validation results presented
here establish the baseline against which intervention scenarios
will be evaluated: any strategy that cannot improve upon the $>$99\%
crash trajectory is insufficient.


% ──────────────────────────────────────────────────────────────────────
\subsection{Limitations}
\label{sec:disc_limitations}

We identify five principal limitations.

\paragraph{1.\ Single-pathogen model.}
\modelname{} attributes SSWD to \vpshort{}, consistent with Koch's
postulates confirmation \citep{Prentice2025}. However,
\citet{Hewson2025autecology} found that \vpshort{} was not detectable
in non-\pyc{} asteroid species, complicating the assumption of a
generalized multi-species reservoir. The etiology of SSWD may involve
microbiome dysbiosis, secondary opportunistic infections, or
multi-pathogen interactions not captured by a single-agent model.

\paragraph{2.\ Environmental pathogen reservoir is unconstrained.}
$P_{\mathrm{env,max}}$ ranks 4th in global sensitivity
($\mu^*_{\mathrm{norm}} = 0.598$, $\sigma/\mu^* = 1.92$) yet has
no empirical calibration target. This parameter absorbs the
complexity of multi-species pathogen maintenance, sediment reservoirs,
and environmental Vibrio dynamics into a single scalar. Field
measurements of waterborne \vpshort{} concentrations in \pyc{}
habitat are needed to constrain it.

\paragraph{3.\ Universal nonlinearity.}
All 47 parameters exhibit $\sigma/\mu^* > 1.0$ in the Morris
screening, indicating that every parameter's effect depends on the
values of every other parameter. While this is a realistic property
of complex biological systems \citep{Saltelli2008}, it means the
model cannot be calibrated by tuning parameters individually. Joint
estimation via ABC-SMC is computationally expensive and requires
well-defined calibration targets, several of which are currently
lacking.

\paragraph{4.\ Recovery rate has zero empirical basis.}
Whether \pyc{} can clear \vpshort{} infections at all is unknown.
The base recovery rate $\rho_{\mathrm{rec}}$ is the single most
influential parameter in the model, yet its value is entirely
assumed. Challenge-recovery experiments in captive animals
\citep{Prentice2025} could resolve this critical gap.

\paragraph{5.\ Spatial resolution.}
Validation runs use 5--11 nodes, well below the 150+ nodes needed
to represent the full NE Pacific range at ecologically meaningful
resolution. The dramatic rank gains of spatial parameters between
R3 (3 nodes) and R4 (11 nodes)---notably $n_{\mathrm{resistance}}$
rising from \#19 to \#5---suggest that further spatial refinement
may reveal additional emergent dynamics not captured at the current
resolution.


% ──────────────────────────────────────────────────────────────────────
\subsection{Future directions}
\label{sec:disc_future}

\paragraph{ABC-SMC calibration.}
The immediate priority is formal calibration via approximate Bayesian
computation with sequential Monte Carlo sampling. Summary statistics
will include range-wide population decline ($>$90\% within 2 years;
\citealp{Gravem2021}), latitudinal mortality gradient
\citep{Hamilton2021}, fjord protection effects \citep{Gehman2025},
allele frequency shifts at outlier loci \citep{Schiebelhut2018}, and
disease progression timelines \citep{Prentice2025}. The R4 sensitivity
analysis provides a natural parameter prioritization: the top 10--15
parameters can be calibrated jointly while fixing the remainder at
default values.

\paragraph{Conservation scenario evaluation.}
A conservation module will simulate captive-bred release strategies
parameterized from AZA SAFE protocols \citep{AZA2024}, including
release timing and location, cohort genetic composition, assisted
gene flow via cryopreserved gametes \citep{Hagedorn2021}, and
minimum viable release sizes informed by the Allee effect dynamics
identified in this study. Empirical validation targets from the 2024
and 2025 outplanting trials will constrain post-release survival
predictions.

\paragraph{Climate change projections.}
Warming sea surface temperatures will alter disease dynamics through
the temperature-dependent transition rates calibrated to
\citet{Prentice2025}. Projecting model behavior under RCP scenarios
will reveal whether warming accelerates population collapse (through
faster disease progression) or modulates it (through altered
seasonality and spatial redistribution of thermal refugia).

\paragraph{Genomic integration.}
The \pyc{} reference genome \citep{Schiebelhut2024genome} enables
GWAS to identify loci associated with resistance, tolerance, and
recovery, providing direct calibration targets for the genetic
architecture parameters. Comparing predicted allele frequency shifts
at the 51 outlier loci identified by \citet{Schiebelhut2018} with
temporal genomic samples from wild populations would provide a
powerful independent validation of the model's evolutionary
predictions.

\paragraph{Multi-species extension.}
Explicitly modeling \vpshort{} dynamics in other asteroid species
would replace the $P_{\mathrm{env}}$ abstraction with mechanistic
cross-species transmission. While architecturally straightforward
(shared pathogen pool with species-specific susceptibility and
shedding), this extension requires demographic and disease parameters
for multiple species that are currently unavailable, and constitutes
a multi-year research program in its own right.
