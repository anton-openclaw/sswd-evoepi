% introduction.tex — Introduction and Study System

\section{Introduction}
\label{sec:introduction}

\subsection{Sea Star Wasting Disease and the Collapse of \pyc{}}

Sea star wasting disease (SSWD) caused one of the largest documented wildlife mass mortality events in marine ecosystems when it swept through populations of over 20 asteroid species along the northeastern Pacific coast beginning in 2013 \citep{Hewson2014,Montecino-Latorre2016,Harvell2019}. The disease, characterized by behavioral changes (arm twisting, lethargy), loss of turgor, body wall lesions, ray autotomy, and rapid tissue degradation, devastated populations from Baja California to the Gulf of Alaska within months \citep{Miner2018,Hewson2019}. Among the species affected, the sunflower sea star (\pyc{}) suffered the most catastrophic decline, losing an estimated 5.75 billion individuals and experiencing a 90.6\% range-wide population reduction based on 61,043 surveys across 31 datasets \citep{Gravem2021,Heady2022}. Along the outer coast from Washington to Baja California, declines exceeded 97\%, with many regions recording zero individuals in subsequent surveys \citep{Hamilton2021,Gravem2021}. The species was assessed as Critically Endangered by the IUCN in 2021 \citep{Gravem2021} and is under consideration for listing as Threatened under the U.S. Endangered Species Act \citep{Lowry2022}.

As a large-bodied, mobile, generalist predator capable of consuming sea urchins at rates sufficient to structure entire subtidal communities, \pyc{} functions as a keystone species in northeastern Pacific kelp forest ecosystems \citep{Galloway2023,Burt2018,Mancuso2025}. Its precipitous decline has been linked to cascading trophic effects, including sea urchin population explosions and extensive kelp forest deforestation, with northern California losing 96\% of its kelp canopy since the 2014 marine heatwave \citep{Meunier2024,Rogers-Bennett2019}. The loss of this apex predator thus represents not only a conservation crisis for a single species but a destabilization of an entire marine ecosystem \citep{Hamilton2021,Langendorf2025}.

\subsection{Etiology: A Decade-Long Mystery Resolved}

For over a decade following the initial outbreak, the causative agent of SSWD remained contested. An early hypothesis implicating sea star associated densovirus (SSaDV; \citealt{Hewson2014}) was subsequently retracted after repeated failures to reproduce the original challenge experiments and the discovery that the virus is endemic in healthy echinoderm populations worldwide \citep{Hewson2018,Hewson2019,Hewson2025review}. An alternative hypothesis invoking boundary layer oxygen depletion (BLODL) at the animal--water interface proposed that microbial respiration on sea star surfaces draws down dissolved oxygen, leading to tissue hypoxia \citep{Aquino2021,Hewson2021}. While this mechanism may contribute to disease susceptibility, it did not identify a specific pathogen.

The breakthrough came with \citet{Prentice2025}, who fulfilled Koch's postulates by demonstrating that \vp{} strain FHCF-3, a Gram-negative marine bacterium, is a causative agent of SSWD in \pyc{}. Through seven controlled exposure experiments using captive-bred, quarantined sea stars, the authors showed that injection of cultured \vpshort{} FHCF-3 into the coelomic cavity reliably produced disease signs --- arm twisting, lesion formation, autotomy, and death within approximately two weeks. Heat-treated and 0.22~\textmu{}m filtered controls remained healthy, confirming a living bacterial agent. Critically, the pathogen was re-isolated from experimentally infected animals, completing Koch's postulates. Earlier investigations had missed \vpshort{} because they sampled body wall tissue rather than coelomic fluid, where the bacterium resides.

However, the etiological picture is not entirely resolved. \citet{Hewson2025autecology} demonstrated that \vpshort{} FHCF-3 was not consistently detected in non-\pyc{} species during the 2013--2014 mass mortality, suggesting it may be specific to \pyc{} or may function as an opportunistic pathogen rather than a universal SSWD agent across all affected asteroid taxa. The bacterium also exhibits explosive growth in the presence of decaying echinoderm tissue, raising questions about whether it acts primarily as a pathogen or a saprobe under different conditions \citep{Hewson2025autecology}. Nonetheless, for \pyc{} --- the focus of this study --- the evidence for \vpshort{} as the primary causative agent is robust. The identification of a specific bacterial pathogen with known temperature-dependent growth dynamics \citep{Lupo2020} provides a mechanistic basis for modeling disease transmission and environmental forcing.

\subsection{Conservation Urgency and Active Recovery Efforts}

The failure of \pyc{} populations to recover naturally in the decade following the initial epizootic --- contrasting with partial recovery observed in some co-occurring asteroid species \citep{Gravem2025} --- has motivated intensive conservation action. The species' long generation time ($\sim$30 years), broadcast spawning reproductive strategy, and vulnerability to Allee effects at low density \citep{Lundquist2004,Gascoigne2004} compound the challenge of natural recovery. Historical precedent is sobering: the Caribbean long-spined sea urchin \textit{Diadema antillarum}, which suffered a comparable 93--100\% mass mortality in 1983--1984, achieved only $\sim$12\% recovery after three decades \citep{Lessios2016}. Another asteroid, \textit{Heliaster kubiniji}, has never recovered from a 1975 mass mortality event in the Gulf of California \citep{Dungan1982}.

In response, a coordinated multi-partner recovery effort has emerged. The Association of Zoos and Aquariums (AZA) Saving Animals From Extinction (SAFE) program maintains over 2,500 captive juveniles and 130+ reproductive adults across 17 AZA institutions \citep{AZA2024}. The first experimental outplanting of captive-bred \pyc{} occurred in December 2025 in Monterey, California, with 47 of 48 juveniles surviving after four weeks \citep{Simon2025}. A Roadmap to Recovery developed by over 30 leading experts defines regionally nested recovery objectives, from local demographic benchmarks to range-wide genetic structure targets \citep{Heady2022}. Cryopreservation of gametes has been demonstrated for a congener and is under development for \pyc{} to enable assisted gene flow from genetically diverse founders \citep{Hagedorn2021,SSL2025}. In 2025, the California Ocean Protection Council approved \$630,000 in funding for captive breeding, disease diagnostics, and experimental outplanting \citep{CAOPC2025}. A reference genome has also been published \citep{Schiebelhut2024genome}, laying the groundwork for genome-wide association studies (GWAS) to identify resistance loci.

These recovery efforts require quantitative predictions: How many captive-bred individuals should be released, where, and when? What are the genetic consequences of releasing animals from a limited captive founder population? Can natural selection drive resistance evolution fast enough to matter on conservation timescales? How do pathogen evolution, environmental change, and spatial structure interact to shape recovery trajectories? Answering these questions demands a modeling framework that integrates disease dynamics with population genetics in an explicitly spatial context.

\subsection{The Need for an Eco-Evolutionary Framework}

Existing models of SSWD dynamics have focused on either epidemiological or ecological aspects in isolation. \citet{Aalto2020} coupled an SIR-type model with ocean circulation to explain the rapid continental-scale spread of SSWD, finding that temperature-dependent mortality best matched observed patterns. \citet{Tolimieri2022} conducted a population viability analysis using stage-structured matrix models but did not incorporate disease dynamics or host genetics. \citet{Arroyo-Esquivel2025} recently modeled epidemiological consequences of managed reintroduction following disease-driven host decline, but their framework lacks genetic evolution. None of these approaches captures the interplay between disease-driven selection, host genetic adaptation, and demographic recovery that is central to predicting conservation outcomes.

The theoretical motivation for coupling these processes is compelling. Mass mortality events impose intense directional selection on host populations \citep{Schiebelhut2018}, and in \textit{Pisaster ochraceus} --- a co-occurring sea star affected by SSWD --- rapid allele frequency shifts ($\Delta q \approx 0.08$--$0.15$ at outlier loci) were detected within a single generation of the epizootic, with geographic consistency across sites indicating selection rather than drift \citep{Schiebelhut2018}. However, in broadcast-spawning marine invertebrates, the genetic consequences of mass mortality are filtered through sweepstakes reproductive success (SRS), whereby variance in individual reproductive success is so large that effective population size ($N_e$) is orders of magnitude smaller than census size ($N_e/N \sim 10^{-3}$; \citealt{Hedgecock2011,Arnason2023}). SRS amplifies genetic drift on ecological timescales \citep{Vendrami2021}, can facilitate rapid adaptation when coupled with bottlenecks \citep{Eldon2024}, and generates chaotic genetic patchiness that confounds simple predictions of evolutionary trajectories. Any model of evolutionary rescue in \pyc{} must therefore account for this fundamental feature of marine broadcast spawner genetics.

The closest methodological precedent is the eco-evolutionary individual-based model (IBM) developed by \citet{Clement2024} for coevolution between Tasmanian devils (\textit{Sarcophilus harrisii}) and devil facial tumour disease (DFTD). That model coupled an SEI epidemiological framework with polygenic quantitative genetics, parameterized from two decades of field data and GWAS results, and found a high probability of host persistence over 50 generations through coevolutionary dynamics. Our model extends this approach to a marine system with fundamentally different reproductive biology --- broadcast spawning with sweepstakes reproductive success, external fertilization subject to Allee effects, and a pelagic larval phase mediating spatial connectivity --- challenges that no existing eco-evolutionary disease model has addressed.

\subsection{Model Overview}

We present \modelname{}, an individual-based, spatially explicit, eco-evolutionary epidemiological model designed to simulate SSWD dynamics and evolutionary responses in \pyc{} metapopulations across the northeastern Pacific. The model tracks individual sea stars as agents within a network of habitat nodes connected by larval dispersal and pathogen transport. Each agent carries a diploid genotype across 51 loci governing three fitness-related traits: resistance ($r_i$, 17 loci; immune exclusion reducing infection probability), tolerance ($t_i$, 17 loci; damage limitation extending survival during late-stage infection), and recovery ($c_i$, 17 loci; pathogen clearance enabling transition from infected to recovered states). Per-locus allele frequencies are drawn from a Beta(2,8) distribution, reflecting polygenic architecture with most loci at low frequency \citep{Hollinger2022}.

Disease dynamics follow an SEIR-type compartmental structure with exposed (E), early infected (I$_1$), and late infected (I$_2$) stages, coupled with an environmental pathogen reservoir (P) whose dynamics are temperature-dependent \citep{Lupo2020,Gimenez-Romero2021}. Pathogen evolution is modeled through a heritable virulence phenotype that evolves along a mechanistic tradeoff curve linking shedding rate to host survival duration. Reproduction incorporates sweepstakes reproductive success via a heavy-tailed offspring distribution producing $N_e/N$ ratios consistent with empirical estimates for marine broadcast spawners \citep{Hedgecock2011}, with sex-asymmetric spawning induction and post-spawning immunosuppression derived from species-specific observations. Spatial connectivity is implemented through distinct larval exchange and pathogen dispersal matrices computed from overwater distances across the model domain.

The model is implemented in Python with NumPy-vectorized agent operations, achieving sufficient performance for large-scale sensitivity analysis and calibration (75,000 agents across 150 nodes in $\sim$72~s). Four rounds of sensitivity analysis using Morris screening and Sobol variance decomposition across up to 47 parameters have identified the key drivers of model behavior, revealing strong nonlinear interactions and highlighting priority targets for empirical calibration.

\subsection{Paper Outline}

The remainder of this paper is organized as follows. Section~\ref{sec:model_architecture} describes the overall model architecture, agent representation, and simulation flow. Sections~\ref{sec:disease_module}--\ref{sec:spatial_module} detail the disease, genetics, population dynamics, and spatial modules, respectively. Section~\ref{sec:sensitivity_analysis} presents four rounds of global sensitivity analysis, identifying the parameters with greatest influence on epidemiological, demographic, and evolutionary outcomes. Section~\ref{sec:validation} describes model validation against available empirical data. Section~\ref{sec:discussion} synthesizes findings, discusses limitations, and outlines the path toward calibrated conservation scenario evaluation. Parameter tables and supplementary analyses are provided in Appendix~\ref{sec:appendix_parameters}.
