% ══════════════════════════════════════════════════════════════════════
% VALIDATION
% Fact-checked against:
%   results/validation_5k_fixed/results.json           (K=5K run)
%   results/three_trait_validation/validation_data.json (Phase 6 run)
%   results/three_trait_validation/VALIDATION_REPORT.md (Phase 6 report)
%   results/validation_100k/results.json                (K=100K run)
%   results/final_5node_20yr/SUMMARY.md                 (original prototype)
% ══════════════════════════════════════════════════════════════════════

\section{Validation}
\label{sec:validation}

We validate the \modelname{} model through a two-stage strategy:
calibration and behavioral verification at computationally cheap
population sizes ($K = 5{,}000$ per node, $\sim$25{,}000 total
agents), followed by scale-up validation at ecologically realistic
population sizes ($K = 100{,}000$ per node, 500{,}000 total agents).
This approach tests whether emergent dynamics---trait evolution
trajectories, spatial mortality gradients, and extinction vortex
behavior---are robust to a 20-fold increase in population size, or
whether they are artifacts of stochastic fluctuations in small
populations. All validation runs use a 5-node stepping-stone network
(Sitka, Howe Sound, San Juan Islands, Newport, Monterey), a 20-year
time horizon with disease introduction at year~3, seed~42, and the
three-trait genetic architecture described in
Section~\ref{sec:three_trait} (17 resistance / 17 tolerance / 17
recovery loci).


% ──────────────────────────────────────────────────────────────────────
\subsection{K = 5{,}000 Validation}
\label{sec:val_5k}

The small-population validation serves as the primary calibration
target, permitting rapid iteration ($\sim$108~s per 20-year
simulation) while retaining sufficient genetic variance for
trait-level dynamics to emerge. Table~\ref{tab:val_5k} reports
per-node demographic and evolutionary outcomes.

\begin{table}[H]
\centering
\caption{Per-node results for the $K = 5{,}000$ validation run
(5~nodes, 20~years, seed~42). $\Delta r_i$, $\Delta t_i$, and
$\Delta c_i$ denote changes in mean resistance, tolerance, and
recovery trait scores relative to initialization
(\(\bar{r}_0 = 0.15\), \(\bar{t}_0 = 0.10\), \(\bar{c}_0 = 0.02\)).
Pop\textsubscript{min} gives the minimum population reached at the
indicated year.}
\label{tab:val_5k}
\small
\begin{tabular}{lrrrrrrrrr}
\toprule
\textbf{Node} & \textbf{$N_0$} & \textbf{$N_{20}$} & \textbf{$N_{\min}$ (yr)}
  & \textbf{Crash} & \textbf{Deaths} & \textbf{Rec.}
  & \boldmath{$\Delta r_i$} & \boldmath{$\Delta t_i$} & \boldmath{$\Delta c_i$} \\
\midrule
Sitka       & 4{,}935 &  65 &  65 (19)  & 98.7\% & 7{,}409  &  60 & $+$0.011 & $+$0.005 & $+$0.029 \\
Howe Sound  & 4{,}937 &  60 &  60 (19)  & 98.8\% & 9{,}473  &  55 & $-$0.002 & $+$0.044 & $+$0.041 \\
SJI         & 4{,}918 &  50 &  50 (13)  & 99.0\% & 7{,}985  &  63 & $+$0.012 & $-$0.007 & $+$0.072 \\
Newport     & 4{,}998 &  27 &  27 (17)  & 99.5\% & 7{,}918  &  51 & $+$0.031 & $+$0.001 & $+$0.054 \\
Monterey    & 5{,}000 & 163 &  38 (10)  & 99.2\% & 9{,}183  & 136 & $+$0.025 & $+$0.027 & $+$0.154 \\
\midrule
\textbf{Total} & \textbf{24{,}788} & \textbf{365} & & \textbf{98.5\%}
  & \textbf{41{,}968} & \textbf{365} & & & \\
\bottomrule
\end{tabular}
\end{table}

Several key patterns emerge from the small-population run:

\paragraph{Severe, universal population crashes.}
All five nodes experience $>$98\% population decline over 17~years of
active disease, with total metapopulation crash of 98.5\%
(24{,}788~$\to$~365 individuals). No node recovers to pre-epidemic
levels, consistent with the persistent absence of \pyc{} across most
of its former range since 2013--2015
\citep{Hamilton2021, Gravem2021}.

\paragraph{Differential recovery at Monterey.}
Monterey exhibits a distinctive trajectory: the population crashes
to a minimum of 38 individuals at year~10 but partially rebounds
to 163 by year~20, driven by 136 disease recoveries---2.2$\times$
the next-highest node (SJI, 63 recoveries). This node also shows
the strongest evolutionary signal in recovery
($\Delta c_i = +0.154$), consistent with warmer temperatures
driving both higher disease pressure and stronger selection for
clearance ability.

\paragraph{Recovery is the fastest-evolving trait.}
Across all five nodes, the mean change in recovery trait score
($\overline{\Delta c_i} = +0.070$) exceeds that of resistance
($\overline{\Delta r_i} = +0.015$) by 4.5$\times$ and tolerance
($\overline{\Delta t_i} = +0.014$) by 5.0$\times$
(Table~\ref{tab:val_5k}). This asymmetry arises because recovery
acts as a multiplicative modifier on the daily probability of
transitioning from infected to recovered
($p_{\mathrm{rec}} = \rho_{\mathrm{rec}} \times c_i$;
Section~\ref{sec:disease_module}), creating strong directional
selection: individuals with higher $c_i$ survive infection and
contribute disproportionately to the next generation.

\paragraph{Resistance signal is weak and mixed.}
With only 17 loci encoding resistance (compared to 51 in the
original single-trait architecture), the per-locus allele frequency
shifts are small ($\Delta q \approx 0.001$--$0.004$).  Three of
five nodes show positive $\Delta r_i$ (Sitka, SJI, Newport), but
Howe Sound shows a negligible decline ($-$0.002), consistent with
genetic drift overwhelming weak directional selection at small
effective population sizes \citep{Hedgecock2011}.

\paragraph{Tolerance is effectively neutral.}
Mean tolerance change is negligible ($\overline{\Delta t_i} =
+0.014$), with one node showing a slight decrease (SJI,
$\Delta t_i = -0.007$). This is expected: tolerance extends
survival time during late infection (I\textsubscript{2}) via
timer-scaling (Section~\ref{sec:disease_module}), but this effect
is weak when recovery rates are low and late-stage mortality is
high. Tolerance becomes selectively relevant only when disease
mortality is moderated by other mechanisms, creating a conditional
neutrality that limits its evolutionary response under severe
epidemic conditions.

\begin{figure}[H]
  \centering
  \includegraphics[width=\textwidth]{figures/val_5k_pop_trajectories.png}
  \caption{Population trajectories for the $K = 5{,}000$ validation
  run. Disease is introduced at year~3. All nodes crash to
  $<$2\% of carrying capacity. Monterey (red) shows partial
  recovery from its nadir of 38 individuals at year~10, driven by
  elevated recovery trait evolution ($\Delta c_i = +0.154$).}
  \label{fig:val_5k_pop}
\end{figure}

\begin{figure}[H]
  \centering
  \includegraphics[width=\textwidth]{figures/val_5k_trait_shifts.png}
  \caption{Trait shifts ($\Delta r_i$, $\Delta t_i$, $\Delta c_i$)
  per node in the $K = 5{,}000$ validation. Recovery (blue)
  dominates at every node, with Monterey showing the largest
  shift ($\Delta c_i = +0.154$). Resistance changes are weak
  and variable in sign; tolerance is near-zero at most nodes.}
  \label{fig:val_5k_shifts}
\end{figure}


% ──────────────────────────────────────────────────────────────────────
\subsection{K = 100{,}000 Scale-Up Validation}
\label{sec:val_100k}

To test whether patterns observed at $K = 5{,}000$ persist at
ecologically realistic population sizes, we scale carrying capacity
20-fold to $K = 100{,}000$ per node (500{,}000 total agents). This
run required 42.6~minutes (2{,}558~s) and $\sim$1.5~GB peak memory,
compared to 108~s for the small-population equivalent---a
23.7$\times$ slowdown that is sublinear relative to the 20$\times$
population increase, consistent with the $O(N^{0.62})$ scaling
relationship established in Section~\ref{sec:sa_methods}.
Table~\ref{tab:val_100k} reports the results.

\begin{table}[H]
\centering
\caption{Per-node results for the $K = 100{,}000$ scale-up validation
(5~nodes, 20~years, seed~42). Trait values are final means;
$\Delta$ values computed relative to initialization targets
(\(\bar{r}_0 = 0.15\), \(\bar{t}_0 = 0.10\), \(\bar{c}_0 = 0.02\)).}
\label{tab:val_100k}
\small
\begin{tabular}{lrrrrrrrr}
\toprule
\textbf{Node} & \textbf{$N_{20}$} & \textbf{Crash}
  & \textbf{Deaths} & \textbf{Rec.}
  & \boldmath{$\Delta r_i$} & \boldmath{$\Delta t_i$} & \boldmath{$\Delta c_i$} \\
\midrule
Sitka       &    718 & 99.3\% & 109{,}151 &   875 & $-$0.004 & $+$0.002 & $+$0.059 \\
Howe Sound  &    633 & 99.4\% & 112{,}112 &   913 & $-$0.004 & $+$0.008 & $+$0.056 \\
SJI         &    733 & 99.3\% & 108{,}607 &   916 & $-$0.009 & $+$0.010 & $+$0.060 \\
Newport     &    639 & 99.4\% & 110{,}563 &   922 & $-$0.005 & $+$0.006 & $+$0.065 \\
Monterey    &  2{,}904 & 97.1\% & 125{,}061 & 1{,}319 & $-$0.002 & $+$0.000 & $+$0.075 \\
\midrule
\textbf{Total} & \textbf{5{,}627} & \textbf{98.9\%}
  & \textbf{565{,}494} & \textbf{4{,}945} & & & \\
\bottomrule
\end{tabular}
\end{table}

The scale-up validation reveals several important findings:

\paragraph{Crashes are worse, not better, at larger $N$.}
Total metapopulation crash increases from 98.5\% at $K = 5{,}000$ to
98.9\% at $K = 100{,}000$ (Table~\ref{tab:val_cross}). This
counterintuitive result refutes the hypothesis that larger
populations buffer against extinction through stochastic rescue.
In the \modelname{} framework, larger populations sustain higher
absolute disease transmission (more contacts per susceptible per
day) while the per-capita selection intensity remains constant,
meaning that deterministic epidemic dynamics dominate and
demographic stochasticity---which occasionally permits small
populations to ``escape'' the disease through random
fluctuations---is suppressed. The 0.4 percentage-point increase
in crash severity is small but directionally consistent across
all five nodes.

\paragraph{Resistance shifts become uniformly negative.}
At $K = 100{,}000$, all five nodes show negative $\Delta r_i$
(range: $-$0.002 to $-$0.009; mean $-$0.005), in contrast to
the mixed signal at $K = 5{,}000$ (three positive, one negative,
one near-zero). With 100{,}000 individuals per node, the
effective population size is large enough to suppress drift,
revealing that the net selection coefficient on resistance is
slightly negative under the current parameterization. This likely
reflects the cost structure: resistance reduces infection
probability multiplicatively ($p_{\mathrm{inf}} \propto
1 - r_i$), but the per-locus effect is small with 17 loci
($\Delta p_{\mathrm{inf}} \approx 0.003$ per locus), while
background environmental pathogen pressure ($P_{\mathrm{env}}$)
ensures continued exposure regardless of individual resistance.

\paragraph{Recovery dominance is amplified at scale.}
The trait evolution hierarchy becomes more pronounced at large
$N$: recovery ($\overline{\Delta c_i} = +0.063$) is
13.3$\times$ faster than resistance
($|\overline{\Delta r_i}| = 0.005$) and 12.2$\times$ faster than
tolerance ($\overline{\Delta t_i} = +0.005$), compared to
4.5$\times$ and 5.0$\times$ respectively at $K = 5{,}000$.  The
ratio increase occurs because drift no longer inflates
$|\Delta r_i|$ at large $N$, exposing the true (weak) directional
signal on resistance.

\paragraph{Monterey remains anomalous.}
Even at $K = 100{,}000$, Monterey shows the lowest crash
percentage (97.1\% vs.\ 99.3--99.4\% for other nodes), the
highest final population (2{,}904), the most recoveries
(1{,}319), and the strongest recovery evolution
($\Delta c_i = +0.075$). This is not a small-$N$ artifact but an
emergent property of Monterey's warmer temperatures, which
simultaneously drive higher disease pressure \emph{and} stronger
selection for clearance ability.


% ──────────────────────────────────────────────────────────────────────
\subsection{Cross-Scale Comparison}
\label{sec:val_cross}

Table~\ref{tab:val_cross} summarizes the comparison between the two
population scales, revealing which patterns are scale-invariant
(and therefore robust model predictions) versus scale-dependent
(and therefore artifacts or emergent threshold effects).

\begin{table}[H]
\centering
\caption{Cross-scale comparison of key metrics between $K = 5{,}000$
and $K = 100{,}000$ validation runs. ``Ratio'' column gives
the 100K value divided by the 5K value.}
\label{tab:val_cross}
\small
\begin{tabular}{lrrr}
\toprule
\textbf{Metric} & \boldmath{$K = 5\text{K}$} & \boldmath{$K = 100\text{K}$} & \textbf{Ratio} \\
\midrule
Total crash (\%)              & 98.5    & 98.9    & 1.004  \\
Mean $\Delta r_i$             & $+$0.015  & $-$0.005  & ---    \\
Mean $\Delta t_i$             & $+$0.014  & $+$0.005  & 0.38   \\
Mean $\Delta c_i$             & $+$0.070  & $+$0.063  & 0.90   \\
Total recoveries              & 365     & 4{,}945   & 13.5   \\
Monterey crash (\%)           & 99.2    & 97.1    & 0.979  \\
Monterey $\Delta c_i$         & $+$0.154  & $+$0.075  & 0.49   \\
Runtime (s)                   & 108     & 2{,}558   & 23.7   \\
\bottomrule
\end{tabular}
\end{table}

Three categories of behavior emerge:

\begin{enumerate}
  \item \textbf{Scale-invariant patterns} (robust predictions):
    \begin{itemize}
      \item Population crashes are catastrophic ($>$97\%) at both
        scales, with no recovery to pre-epidemic levels.
      \item Recovery ($c_i$) is the fastest-evolving trait at every
        node and both scales.
      \item Monterey is consistently the most resilient node.
      \item The extinction vortex---positive feedback between
        small population size, Allee effects, and continued
        pathogen pressure---operates at both scales.
    \end{itemize}

  \item \textbf{Scale-sensitive patterns} (require caution):
    \begin{itemize}
      \item Resistance evolution: positive at $K = 5{,}000$ (mean
        $+$0.015), negative at $K = 100{,}000$ (mean $-$0.005).
        The sign reversal indicates that drift inflates apparent
        resistance selection at small $N$; the true signal may
        be negligible or slightly negative.
      \item Monterey's recovery evolution is 2$\times$ stronger at
        small $N$ ($\Delta c_i = +0.154$ vs.\ $+$0.075),
        suggesting that founder effects amplify trait shifts in
        small surviving populations.
      \item Tolerance shifts shrink from $+$0.014 to $+$0.005,
        confirming conditional neutrality.
    \end{itemize}

  \item \textbf{Scale-revealing patterns} (insights from large $N$):
    \begin{itemize}
      \item Uniformly negative $\Delta r_i$ at $K = 100{,}000$
        reveals that 17 loci provide insufficient genetic
        variance for resistance evolution to outpace pathogen
        pressure, consistent with the sensitivity analysis finding
        that \texttt{n\_resistance} is the 5th most important
        parameter (Section~\ref{sec:sensitivity_analysis}).
      \item The crash percentage \emph{increases} at larger $N$,
        demonstrating that stochastic rescue is not a viable
        recovery mechanism and that demographic rescue through
        immigration or captive breeding is required.
    \end{itemize}
\end{enumerate}


% ──────────────────────────────────────────────────────────────────────
\subsection{Reinfection Dynamics: R$\to$S Validation}
\label{sec:val_rs}

Echinoderms lack adaptive immunity: there is no evidence of acquired
resistance to SSWD following recovery, and stars treated for wasting
have subsequently become reinfected. To reflect this biology, recovered
individuals in \modelname{} return to the susceptible pool rather than
entering a permanently immune state. We validate the impact of this
reinfection dynamic by comparing the $K = 5{,}000$ baseline
(permanent immunity) with the corrected R$\to$S formulation under
both sinusoidal and satellite SST forcing.

\begin{table}[H]
\centering
\caption{Impact of R$\to$S reinfection dynamics on $K = 5{,}000$
validation (5~nodes, 20~years, seed~42). The R$\to$S correction
dramatically worsens population outcomes and fundamentally alters
the evolutionary trajectory.}
\label{tab:val_rs}
\small
\begin{tabular}{lrrr}
\toprule
\textbf{Metric} & \textbf{Perm.\ immunity} & \textbf{R$\to$S (sinusoidal)} & \textbf{R$\to$S (satellite)} \\
\midrule
Overall crash (\%)    & 98.5    & 99.7    & 99.9 \\
Final population      & 365     & 122     & 146 \\
Node extinctions      & 0       & 2       & 2 \\
Total recoveries      & 365     & 276     & 241 \\
Recovery rate (\%)    & 0.87    & 0.76    & 0.71 \\
Mean $\Delta r_i$     & $+$0.015 & $+$0.012 & --- \\
Mean $\Delta c_i$     & $+$0.070 & $+$0.002 & --- \\
\bottomrule
\end{tabular}
\end{table}

The R$\to$S correction produces four critical changes:

\paragraph{Recovery trait no longer evolves upward.}
Under permanent immunity, recovery ($c_i$) was the fastest-evolving
trait, with Monterey showing $\Delta c_i = +0.154$. With R$\to$S,
the strongest surviving-node shift is $+$0.030 (Newport)---a 5-fold
reduction. The mechanism is clear: recovered stars immediately
re-enter the susceptible pool and face reinfection, so high-recovery
alleles do not accumulate because their carriers keep getting
reinfected and dying.

\paragraph{Local extinctions emerge.}
San Juan Islands and Monterey crash to zero population under R$\to$S
but maintained small populations (50 and 163, respectively) with
permanent immunity. Without the ``safe harbor'' of an immune recovered
class, relentless reinfection cycles drive these nodes to local
extinction.

\paragraph{Selection shifts from recovery to resistance.}
In surviving nodes, resistance shows the strongest positive selection
under R$\to$S: Sitka $\Delta r_i = +0.060$ (vs.\ $+$0.011 baseline,
a 5.5$\times$ increase). When recovery does not confer lasting
protection, avoiding infection entirely becomes more valuable than
clearing infection.

\paragraph{Satellite vs.\ sinusoidal SST.}
The two SST forcing modes produce qualitatively identical dynamics
(99.7\% vs.\ 99.9\% crash), but satellite forcing shifts which
specific nodes persist, reflecting real coastal oceanographic
heterogeneity captured by the NOAA OISST v2.1 climatology.

These results fundamentally change the model's conservation
implications: evolutionary rescue via the recovery trait, which
appeared promising under the (incorrect) permanent immunity
assumption, is not viable with realistic echinoderm biology.
This strengthens the case for active intervention through
captive breeding with selection for resistance ($r_i$) rather
than recovery alone.


% ──────────────────────────────────────────────────────────────────────
\subsection{Key Scientific Findings}
\label{sec:val_findings}

The validation runs, taken together with the four-round sensitivity
analysis (Section~\ref{sec:sensitivity_analysis}), yield several
findings with direct implications for conservation management and
evolutionary theory.

\subsubsection{Evolutionary Rescue Is Insufficient}
\label{sec:val_evo_rescue}

The central question motivating \modelname{} is whether natural
selection on polygenic resistance can drive population recovery
following the SSWD pandemic. Our results provide a clear negative
answer under current parameterization: even over 20~years
($\sim$4 generations for \pyc{}), evolved resistance produces
negligible demographic benefit.  At $K = 100{,}000$, resistance
\emph{declines} at all nodes despite ongoing selection against
susceptible individuals. Two mechanisms explain this failure:

\begin{enumerate}
  \item \textbf{Insufficient genetic architecture.}
    With only 17 resistance loci, the maximum resistance score
    achievable by selection is constrained. Per-locus allele
    frequency shifts of $\sim$0.001--0.003 per generation are an
    order of magnitude below the $\Delta q \approx 0.08$--0.15
    reported by \citet{Schiebelhut2018} for SSWD-associated loci
    in \textit{Pisaster ochraceus}. This discrepancy may reflect
    either a true species difference or an indication that more
    loci of larger effect contribute to resistance in nature than
    are modeled here.

  \item \textbf{Environmental pathogen reservoir.}
    The background environmental pathogen concentration
    ($P_{\mathrm{env}}$) ensures continued disease exposure
    regardless of evolved host resistance. Even if a subpopulation
    achieves high mean resistance, $P_{\mathrm{env}}$ maintains
    baseline infection rates that prevent population recovery
    below the Allee threshold. The sensitivity analysis identified
    $P_{\mathrm{env,max}}$ as the 4th most influential parameter
    globally, and the most influential for spatial protection
    metrics.
\end{enumerate}

This finding is consistent with evolutionary rescue theory
\citep{Clement2024}, which predicts that rescue is most
likely when standing genetic variance is high, generation times
are short relative to population decline rates, and the
environment permits population persistence long enough for
adaptation to occur. For \pyc{}, with generation times of
$\sim$5~years and crash timescales of $\sim$2~years, the
mismatch is severe.

\subsubsection{Recovery as the Primary Adaptive Pathway}
\label{sec:val_recovery}

The consistent dominance of recovery evolution ($c_i$) across both
scales and all five nodes suggests that pathogen clearance, rather
than infection prevention (resistance) or damage limitation
(tolerance), is the primary adaptive pathway available to
\pycshort{} under SSWD. This is mechanistically intuitive:
recovery acts directly on the transition probability from infected
to recovered state ($p_{\mathrm{rec}} = \rho_{\mathrm{rec}} \times
c_i$), creating strong phenotype--fitness mapping. Individuals
that clear infection survive and reproduce; those that do not, die.
The fitness gradient is steep and unambiguous.

However, the absolute recovery trait values remain low even after
20~years of evolution (final $\bar{c}_i \approx 0.07$--0.09 at
$K = 100{,}000$), corresponding to daily clearance probabilities
of only 0.35--0.45\% ($p_{\mathrm{rec}} = 0.05 \times c_i$).
While selection detectably increases $c_i$, the resulting
clearance rates are far below what is needed to substantially
reduce disease-induced mortality.

\subsubsection{The Extinction Vortex Persists at Realistic Scales}
\label{sec:val_vortex}

The persistence of $>$97\% population crashes at $K = 100{,}000$
demonstrates that the extinction vortex identified in the original
prototype is not an artifact of small population sizes. Three
reinforcing feedbacks maintain the vortex:

\begin{enumerate}
  \item \textbf{Density-dependent transmission:} as the population
    declines, per-capita contact rates remain high because
    pathogen concentration ($P_{\mathrm{env}}$) does not decline
    proportionally.
  \item \textbf{Allee effects in reproduction:} below critical
    densities, broadcast-spawning fertilization success collapses
    due to sperm dilution \citep{Gascoigne2004}, reducing
    recruitment even when surviving individuals are genetically
    resistant.
  \item \textbf{Sweepstakes reproductive success:} SRS amplifies
    drift and further reduces $N_e$ relative to census $N$,
    diminishing the efficacy of selection \citep{Hedgecock2011}.
\end{enumerate}

The monotonic population decline with no recovery inflection point
is consistent with field observations: seven years after the
initial 2013--2015 pandemic, \pycshort{} remains functionally
absent from most of its former range \citep{Hamilton2021,
Gravem2025}, with only scattered observations of wild individuals
in California since 2025 \citep{SSL2025}.

\subsubsection{Implications for Captive Breeding}
\label{sec:val_captive}

The model results strongly reinforce the case for captive breeding
and managed release as the primary conservation strategy for
\pycshort{} \citep{Hodin2021, AZA2024}. Three specific model
predictions support this conclusion:

\begin{enumerate}
  \item \textbf{No natural recovery trajectory exists:} at no node
    and at no population scale does the model predict recovery to
    $>$5\% of carrying capacity within 20~years. Without
    demographic intervention, populations remain in the extinction
    vortex.

  \item \textbf{Recovery trait evolution is the most promising
    pathway:} if captive breeding programs can select for high
    $c_i$ (pathogen clearance ability), released individuals may
    have elevated survival probability in endemic disease
    environments. The strong fitness gradient on $c_i$ suggests
    that any heritable variation in clearance ability will be
    rapidly amplified by natural selection post-release.

  \item \textbf{Scale matters:} the worse-at-larger-$N$ result
    implies that releasing large numbers of individuals is
    necessary but not sufficient; releases must also achieve
    densities above the Allee threshold at the local scale to
    enable reproductive success.
\end{enumerate}

These predictions align with early empirical results from
outplanting trials: the first uncaged release of 20 captive-bred
juveniles off San Juan Island in July--August 2024
\citep{kuow2024seastar}, and the Sunflower Star Laboratory's
December 2025 California outplanting, where 47 of 48 juveniles
survived four weeks at Monterey Bay \citep{ssl2025outplanting}---the
same node that shows the highest resilience in our simulations.
