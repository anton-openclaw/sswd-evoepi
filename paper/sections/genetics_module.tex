% ══════════════════════════════════════════════════════════════════════
% GENETICS MODULE
% Fact-checked against: sswd_evoepi/genetics.py, config.py
%   (GeneticsSection), types.py, three-trait-genetic-architecture-spec.md
% ══════════════════════════════════════════════════════════════════════

\section{Genetics Module}
\label{sec:genetics_module}

The genetics module tracks a diploid genotype at 51 biallelic loci for
every individual, partitioned into three quantitative defense traits:
\emph{resistance}, \emph{tolerance}, and \emph{recovery}. Genotypes
are transmitted via Mendelian inheritance with free recombination,
mutated at a per-allele rate $\mu = 10^{-8}$ per generation
\citep{Lynch2010}, and subject to natural selection through the
coupling of trait scores to disease dynamics
(Section~\ref{sec:disease_module}). The module additionally implements
sweepstakes reproductive success (SRS) to capture the extreme
reproductive variance characteristic of broadcast-spawning marine
invertebrates \citep{Hedgecock2011}.


% ──────────────────────────────────────────────────────────────────────
\subsection{Three-Trait Architecture}
\label{sec:three_trait}

Each individual carries a $(51 \times 2)$ genotype array of
\texttt{int8} alleles, where the 51 loci are partitioned into three
contiguous blocks:

\begin{table}[H]
\centering
\caption{Three-trait genetic architecture. The partition is
configurable (constraint: $n_R + n_T + n_C = 51$); the default
17/17/17 split is used in all analyses reported here.}
\label{tab:trait_partition}
\small
\begin{tabular}{lcccp{6.2cm}}
\toprule
\textbf{Trait} & \textbf{Symbol} & \textbf{Loci} & \textbf{Indices} & \textbf{Mechanistic role} \\
\midrule
Resistance & $r_i$ & $n_R = 17$ & $0$--$16$  & Immune exclusion: reduces probability of $\text{S} \to \text{E}$ transition \\
Tolerance  & $t_i$ & $n_T = 17$ & $17$--$33$ & Damage limitation: extends $\text{I}_2$ survival via mortality rate reduction \\
Recovery   & $c_i$ & $n_C = 17$ & $34$--$50$ & Pathogen clearance: daily probability of $\text{I}_1$/$\text{I}_2 \to \text{S}$ transition (recovery) \\
\bottomrule
\end{tabular}
\end{table}

These three traits represent biologically distinct immune strategies
with different epidemiological consequences
\citep{raberg2009decomposing}:
%
\begin{itemize}
  \item \textbf{Resistance} ($r_i$) acts \emph{before} infection via
    receptor polymorphisms, barrier defenses, and innate pathogen
    recognition. Resistant individuals reduce pathogen pressure on the
    population by preventing shedding entirely.

  \item \textbf{Tolerance} ($t_i$) acts \emph{during} infection via
    tissue repair, anti-inflammatory regulation, and metabolic
    compensation. Tolerant hosts survive longer while infected but
    continue to shed pathogen---they are epidemiological ``silent
    spreaders'' that maintain transmission pressure while saving
    themselves.

  \item \textbf{Recovery} ($c_i$) acts \emph{during late infection}
    via coelomocyte phagocytosis and immune effector mobilization.
    Recovering hosts actively clear the pathogen and return to the
    susceptible pool (S), removing a shedding host from the population.
    Because echinoderms lack adaptive immunity, recovered individuals
    may be reinfected.
\end{itemize}

The locus count of 51 is motivated by \citet{Schiebelhut2018}, who
identified $\sim$51 loci under selection in \textit{Pisaster
ochraceus} SSWD survivors. No genome-wide association study (GWAS) data
currently distinguish resistance, tolerance, and recovery loci in
\pycshort{}; the equal 17/17/17 partition is a simplifying assumption
whose sensitivity is explored via the $n_R$ parameter in the global
sensitivity analysis (Section~\ref{sec:sensitivity_analysis}). A
reference genome for \pycshort{} is now available
\citep{Schiebelhut2024genome}, enabling future empirical partitioning.

\paragraph{Removal of EF1A overdominant locus.}
An earlier model version included a 52nd locus representing the EF1A
elongation factor with overdominant fitness effects, based on
\citet{Wares2016} who documented allele frequency shifts at this locus
in \textit{Pisaster ochraceus} following SSWD. We removed this locus
because (1) the EF1A finding is specific to \textit{Pisaster} with no
evidence of overdominance in \pycshort{}, and (2) a single
overdominant locus imposed a hard floor on heterozygosity loss that
was biologically unjustified for our focal species.


% ──────────────────────────────────────────────────────────────────────
\subsection{Trait Score Computation}
\label{sec:trait_scores}

At each locus $\ell$, an individual carries two alleles $g_{\ell,0},
g_{\ell,1} \in \{0, 1\}$, where $1$ denotes the derived (protective)
allele and $0$ the ancestral allele. Each locus within a trait block
has a fixed effect size $e_\ell > 0$, and an individual's trait score
is the effect-weighted mean allele dosage:

\begin{equation}
  \label{eq:trait_score}
  \theta_i = \sum_{\ell \in \mathcal{L}_\theta}
    e_\ell \, \frac{g_{\ell,0} + g_{\ell,1}}{2}
\end{equation}

\noindent
where $\mathcal{L}_\theta$ denotes the locus set for trait
$\theta \in \{r, t, c\}$ and $\theta_i \in [0, \sum e_\ell]$.
Effect sizes within each trait block are normalized so
$\sum_{\ell \in \mathcal{L}_\theta} e_\ell = 1$, bounding all
trait scores to $[0, 1]$.


\subsubsection{Effect Size Distribution}
\label{sec:effect_sizes}

Per-locus effect sizes are drawn from an exponential distribution
$e_\ell \sim \text{Exp}(\lambda = 1)$, normalized to sum to 1.0
within each trait, and sorted in descending order. This produces a
distribution where a few loci have large effects and the remainder
have small effects, consistent with empirical QTL architectures for
disease resistance traits \citep{Lotterhos2015}. A fixed seed ensures
identical effect sizes across simulation runs. Each trait block
receives independently drawn effect sizes.


\subsubsection{Coupling to Disease Dynamics}
\label{sec:trait_disease_coupling}

The three traits feed into the disease module
(Section~\ref{sec:disease_module}) as follows:

\begin{enumerate}
  \item \textbf{Resistance} reduces the per-individual force of
    infection:
    \begin{equation}
      \label{eq:resistance_foi}
      \lambda_i = a \cdot \frac{P}{K_{1/2} + P}
        \cdot (1 - r_i) \cdot S_\text{sal} \cdot f_L(L_i)
    \end{equation}
    where $a$ is the exposure rate, $P$ the local \vp{} concentration,
    $K_{1/2}$ the half-infective dose, $S_\text{sal}$ the salinity
    modifier, and $f_L(L_i)$ the size-dependent susceptibility factor.

  \item \textbf{Tolerance} reduces the I$_2 \to$ D transition rate
    via a timer-scaling mechanism:
    \begin{equation}
      \label{eq:tolerance_mortality}
      \mu_{\text{I}_2\text{D},i}^\text{eff}
        = \mu_{\text{I}_2\text{D}}(T)
        \cdot \bigl(1 - t_i \cdot \tau_\text{max}\bigr)
    \end{equation}
    where $\tau_\text{max} = 0.85$ is the maximum mortality reduction
    achievable at $t_i = 1$. A floor of $5\%$ of the baseline rate
    prevents complete elimination of disease mortality. Tolerant
    individuals survive longer while infected but continue shedding,
    creating a selective conflict between individual and population-level
    fitness.

  \item \textbf{Recovery} determines the daily clearance probability:
    \begin{equation}
      \label{eq:recovery_prob}
      p_{\text{rec},i} = \rho_\text{rec} \times c_i
    \end{equation}
    where $\rho_\text{rec} = 0.05$~d$^{-1}$ is the base recovery rate.
    Recovery from I$_1$ requires $c_i > 0.5$ (early clearance); recovery
    from I$_2$ has no threshold. Successful recovery transitions the
    agent to the susceptible pool (S), as echinoderms lack acquired immunity.
\end{enumerate}


% ──────────────────────────────────────────────────────────────────────
\subsection{Genotype Initialization}
\label{sec:geno_init}

Initial allele frequencies are drawn independently for each locus from
a Beta distribution:
\begin{equation}
  q_\ell \sim \text{Beta}(a, b) \quad \text{(default } a = 2,\; b = 8\text{)}
\end{equation}
producing a right-skewed frequency spectrum where most protective
alleles are rare ($\mathbb{E}[q] = a/(a+b) = 0.2$), consistent with
standing variation in immune genes maintained by mutation--selection
balance. The raw frequencies are then rescaled per-trait so that the
expected population-mean trait score equals a configurable target:

\begin{table}[H]
\centering
\caption{Default target population-mean trait scores at initialization.}
\label{tab:trait_targets}
\small
\begin{tabular}{lcc}
\toprule
\textbf{Trait} & \textbf{Target mean} & \textbf{Rationale} \\
\midrule
Resistance ($r_i$) & 0.15 & Pre-epidemic standing variation \\
Tolerance ($t_i$)  & 0.10 & Moderate damage limitation \\
Recovery ($c_i$)   & 0.02 & Rare standing variation for clearance \\
\bottomrule
\end{tabular}
\end{table}

\noindent
Recovery is initialized with the lowest mean because active pathogen
clearance is assumed to be the rarest phenotype prior to epidemic
exposure. Per-locus frequencies are clipped to $[0.001, 0.5]$ to
prevent fixation at initialization while ensuring the derived allele
never begins at majority frequency. Genotypes are then sampled
assuming Hardy--Weinberg equilibrium at each locus: each allele copy
is independently drawn as a Bernoulli trial with probability $q_\ell$.


% ──────────────────────────────────────────────────────────────────────
\subsection{Mendelian Inheritance and Mutation}
\label{sec:inheritance}

At reproduction, each offspring inherits one randomly chosen allele
from each parent at every locus (independent assortment, no linkage).
The vectorized implementation draws allele choices for all
$n_\text{offspring} \times 51 \times 2$ positions simultaneously,
then indexes into parental genotype arrays.

Mutations are applied to offspring genotypes at rate
$\mu = 10^{-8}$ per allele per generation \citep{Lynch2010}. The
total number of mutations per cohort is drawn from a Poisson
distribution: $n_\text{mut} \sim \text{Pois}(\mu \times
n_\text{offspring} \times 51 \times 2)$. Each mutation flips the
allele at a randomly chosen position ($0 \to 1$ or $1 \to 0$),
providing bidirectional mutational pressure. At the default mutation
rate, mutations are negligible within the 20--100 year simulation
horizon (expected $\sim 10^{-6}$ mutations per offspring), and
evolution proceeds primarily through selection on standing variation.


% ──────────────────────────────────────────────────────────────────────
\subsection{Sweepstakes Reproductive Success}
\label{sec:srs}

Broadcast-spawning marine invertebrates exhibit sweepstakes
reproductive success (SRS): a tiny fraction of adults contribute the
majority of surviving offspring in any given cohort
\citep{Hedgecock2011}. This phenomenon produces
$N_e/N$ ratios on the order of $10^{-3}$ in empirical observations
\citep{Arnason2023} and dramatically amplifies genetic drift while
simultaneously accelerating the fixation of favorable alleles in
post-epidemic populations \citep{Eldon2024}.

\modelname{} implements SRS via a Pareto-weighted reproductive lottery.
Each spawning adult receives a random weight drawn from a Pareto
distribution with shape parameter $\alpha_\text{SRS}$ (default 1.35):

\begin{equation}
  w_i \sim \text{Pareto}(\alpha_\text{SRS}) + 1
\end{equation}

Female weights are additionally multiplied by size-dependent fecundity
(Section~\ref{sec:fecundity}), so larger females that win the
sweepstakes lottery contribute disproportionately:

\begin{equation}
  \tilde{w}_{i,\text{female}} = w_i \times
    \left(\frac{L_i}{L_\text{ref}}\right)^b
\end{equation}

\noindent
where $b = 2.5$ is the fecundity allometric exponent and
$L_\text{ref} = 500$~mm. Male weights use the raw Pareto draw without
fecundity modulation. Parents are then sampled with replacement from
the normalized weight distributions, and offspring receive
Mendelian-inherited genotypes.

The Pareto shape $\alpha_\text{SRS} = 1.35$ was chosen to produce
$N_e/N$ ratios consistent with empirical estimates of $\sim\!10^{-3}$
in marine broadcast spawners \citep{Hedgecock2011, Arnason2023}. A
small annual variation in $\alpha$ (drawn from
$\mathcal{N}(\alpha_\text{SRS}, \sigma_\alpha^2)$ with
$\sigma_\alpha = 0.10$) produces temporal fluctuation in the variance
of reproductive success across cohorts.

\paragraph{Effective population size.}
$N_e$ is computed from the realized offspring distribution using the
standard formula \citep{Hedgecock2011}:
\begin{equation}
  N_e = \frac{4N - 2}{V_k + 2}
\end{equation}
where $N$ is the number of spawning parents and $V_k$ is the variance
in offspring number. Sex-specific $N_e$ values are computed for
females and males separately, then combined via harmonic mean:
$N_e = 4 N_{e,f} N_{e,m} / (N_{e,f} + N_{e,m})$.


% ──────────────────────────────────────────────────────────────────────
\subsection{Genetic Diagnostics and Tracking}
\label{sec:genetics_diagnostics}

The model records a suite of genetic summary statistics at each node
at annual intervals:

\begin{itemize}
  \item \textbf{Per-trait means and variances:}
    $\bar{r}$, $\bar{t}$, $\bar{c}$ and
    $\text{Var}(r)$, $\text{Var}(t)$, $\text{Var}(c)$.

  \item \textbf{Additive genetic variance} ($V_A$) per trait:
    \begin{equation}
      V_{A,\theta} = \sum_{\ell \in \mathcal{L}_\theta}
        2 \, e_\ell^2 \, q_\ell \, (1 - q_\ell)
    \end{equation}
    where $q_\ell$ is the derived allele frequency at locus $\ell$.
    $V_A$ determines the potential rate of evolutionary response to
    selection.

  \item \textbf{Heterozygosity:} Observed ($H_o$) and expected ($H_e$)
    heterozygosity averaged across all 51 loci.

  \item \textbf{$F_{ST}$:} Weir--Cockerham-style $F_{ST}$ across nodes,
    computed as $F_{ST} = \text{Var}(\bar{q}) / [\bar{q}(1-\bar{q})]$
    averaged across polymorphic loci.

  \item \textbf{Pre- and post-epidemic allele frequency snapshots:}
    Full 51-locus allele frequency vectors taken immediately before
    pathogen introduction and two years after the epidemic onset,
    enabling direct measurement of allele frequency shifts
    ($\Delta q$) attributable to selection.
\end{itemize}

\paragraph{No cost of resistance.}
A cost-of-resistance parameter (fecundity penalty for high $r_i$) was
considered but excluded following discussion with the senior author.
No empirical evidence supports a measurable fecundity cost for disease
resistance alleles in \pycshort{}, and including an unparameterized
cost would introduce a free parameter with no calibration target.
Fecundity depends solely on body size
(Section~\ref{sec:fecundity}).


% ──────────────────────────────────────────────────────────────────────
\subsection{Genotype Bank (Tier~2 Nodes)}
\label{sec:genotype_bank}

For Tier~2 spatial nodes that use simplified demographics rather than
full agent tracking, the genetics module maintains a \emph{genotype
bank} of $N_\text{bank} = 100$ representative diploid genotypes with
associated frequency weights. The bank is created by random sampling
from the alive population and preserves all three trait scores and
allele frequencies. When agents migrate from a Tier~2 to a Tier~1
node, genotypes are expanded from the bank using SRS-weighted sampling
(Pareto weights $\times$ bank frequency weights) to reconstruct
individual-level genetic variation.
