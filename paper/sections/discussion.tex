% ══════════════════════════════════════════════════════════════════════
% DISCUSSION
% ══════════════════════════════════════════════════════════════════════

\section{Discussion}
\label{sec:discussion}

\modelname{} represents, to our knowledge, the first individual-based
model to couple eco-evolutionary host--pathogen dynamics with
sweepstakes reproductive success in a marine broadcast spawner.
By tracking diploid genotypes at 51 loci across three defense traits
(resistance, tolerance, recovery) while simultaneously resolving
disease transmission, pathogen virulence evolution, and spatially
explicit metapopulation dynamics, the model provides a framework for
evaluating conservation interventions that depend on the interplay
between ecological and evolutionary processes. Here we discuss the
principal contributions of this work, its relationship to existing
eco-evolutionary disease models, key limitations, and priorities for
future development.


% ──────────────────────────────────────────────────────────────────────
\subsection{Summary of Contributions}
\label{sec:disc_contributions}

This study makes four principal contributions:

\begin{enumerate}
  \item \textbf{Integrated eco-evolutionary framework for SSWD.}
    Previous models of SSWD dynamics have addressed epidemiology
    \citep{Aalto2020}, population viability \citep{Tolimieri2022},
    and reintroduction epidemiology \citep{Arroyo-Esquivel2025}
    in isolation. \modelname{} integrates these processes,
    enabling the emergent dynamics that arise from feedback loops
    between disease-driven selection, host genetic adaptation,
    demographic recovery, and pathogen counter-adaptation to be
    studied within a single coherent framework.

  \item \textbf{Three-trait genetic architecture.}
    Decomposing host defense into resistance (immune exclusion),
    tolerance (damage limitation), and recovery (pathogen
    clearance)---following the conceptual framework of
    \citet{raberg2009decomposing}---reveals that these traits
    evolve at markedly different rates under SSWD selection
    pressure. Recovery ($c_i$) emerges as the fastest-evolving
    trait at every node and population scale
    (Section~\ref{sec:validation}), a prediction that is
    testable with longitudinal genomic sampling.

  \item \textbf{Comprehensive global sensitivity analysis.}
    Four rounds of progressive SA spanning 47 parameters, up to 23
    output metrics, and spatial configurations from 3 to 11 nodes
    identify the parameters most influential for model behavior
    (Section~\ref{sec:sensitivity_analysis}). The finding that all
    47 parameters exhibit nonlinear interactions
    ($\sigma/\mu^* > 1.0$) establishes that joint calibration via
    approximate Bayesian computation is essential, and identifies
    $\rho_{\text{rec}}$, $k_{\text{growth}}$, $K_{\text{half}}$,
    $P_{\text{env,max}}$, and $n_{\text{resistance}}$ as the
    highest-priority empirical targets.

  \item \textbf{Scale-invariant behavioral predictions.}
    Cross-scale validation from $K = 5{,}000$ to $K = 100{,}000$
    per node demonstrates that catastrophic population crashes
    ($>$97\%), recovery trait dominance, and the extinction vortex
    are robust predictions of the model, not artifacts of small
    population sizes (Section~\ref{sec:val_cross}).
\end{enumerate}


% ──────────────────────────────────────────────────────────────────────
\subsection{Comparison with Clement et al.\ (2024)}
\label{sec:disc_clement}

The closest methodological precedent for \modelname{} is the
eco-evolutionary IBM developed by \citet{clement2024coevolution} for
coevolution between Tasmanian devils (\textit{Sarcophilus harrisii})
and devil facial tumour disease (DFTD). Both models track individual
diploid genotypes, couple SEI-type disease dynamics with quantitative
genetic evolution, and explore the conditions under which evolutionary
rescue can avert host extinction following a novel disease
introduction. However, several fundamental differences in the study
systems produce divergent model architectures and predictions.

\paragraph{Reproductive biology.}
Tasmanian devils are iteroparous mammals with deterministic
reproduction: each female produces a small litter ($\sim$4 young)
per season, with high maternal investment per offspring. \pyc{}
is a broadcast spawner producing $\sim$$10^7$ eggs per female, with
fertilization success dependent on gamete encounter rates in the
water column and subject to Allee effects at low density
\citep{Lundquist2004}. \modelname{} implements sweepstakes
reproductive success (SRS) via a Pareto-distributed offspring
contribution ($\alpha \approx 1.35$), producing $N_e/N \sim 10^{-3}$
\citep{Hedgecock2011}---a reproductive mode with no analog in the
Clement et al.\ framework. SRS amplifies genetic drift at the
population level while simultaneously creating the potential for
rapid frequency shifts at individual loci when combined with strong
selection \citep{Eldon2024}, fundamentally altering the
evolutionary dynamics compared to a mammalian system.

\paragraph{Spatial structure.}
The Clement et al.\ model operates on a single well-mixed population,
reflecting the relatively continuous distribution of Tasmanian
devils across Tasmania. \modelname{} represents the NE Pacific range
of \pyc{} as a metapopulation network of up to 150 discrete habitat
nodes connected by larval dispersal and waterborne pathogen transport.
This spatial complexity introduces dispersal--selection interactions
that are absent in the single-population case: local adaptation can
proceed at different rates across nodes (as observed in the
differential recovery at Monterey; Section~\ref{sec:val_5k}), and
larval exchange can either homogenize or maintain genetic
differentiation depending on the balance of gene flow and
spatially heterogeneous selection.

\paragraph{Pathogen evolution.}
Clement et al.\ model DFTD as a clonally transmitted cancer whose
evolution follows a phenotypic difference model, with host resistance
and tumor growth rate coevolving along continuous trait axes. Their
key finding---that coevolution enables host persistence over 50
generations---relies on the tumor's capacity to evolve reduced
virulence in response to host resistance. \modelname{} implements
pathogen evolution through a heritable virulence phenotype that
scales shedding rate, host mortality rate, and disease progression
along mechanistic tradeoff curves
(Section~\ref{sec:sensitivity_analysis}). The qualitative prediction
differs: in our model, the environmental pathogen reservoir
($P_{\text{env}}$) decouples pathogen fitness from individual host
survival, weakening the virulence--transmission tradeoff that drives
attenuation in the Clement et al.\ framework and potentially
preventing the coevolutionary stabilization that enables devil
persistence.

\paragraph{Evolutionary rescue prospects.}
Clement et al.\ found a high probability of devil persistence over
50 generations ($\sim$150 years for devils), driven by rapid
coevolutionary dynamics. Our model produces a starkly different
prediction: no recovery to $>$5\% of carrying capacity within 20
years ($\sim$4 \pyc{} generations) at any node or population scale.
This contrast likely reflects the fundamental mismatch between
\pyc{}'s long generation time ($\sim$5 years vs.\ $\sim$3 years for
devils), the extreme variance in reproductive success under SRS
(which reduces the efficacy of selection relative to drift), and the
environmental pathogen reservoir that maintains infection pressure
independently of the host population's genetic composition.


% ──────────────────────────────────────────────────────────────────────
\subsection{The Environmental Pathogen Reservoir as a Multi-Species Abstraction}
\label{sec:disc_penv}

The environmental pathogen concentration $P_{\text{env}}$ is the
most conceptually novel---and most empirically unconstrained---element
of the \modelname{} disease module. Rather than explicitly modeling
\vpshort{} dynamics in non-\pyc{} host species, $P_{\text{env}}$
serves as an aggregate abstraction for all pathogen sources external
to the focal \pyc{} population: other asteroid species, marine
sediment reservoirs, and environmental bacteria. This design choice
was motivated by two considerations.

First, the multi-species nature of the 2013--2015 SSWD pandemic,
which affected $>$20 asteroid species \citep{Miner2018,Hewson2019},
implies that \vpshort{} (or closely related Vibrio strains) can
persist in the environment independently of any single host species.
\citet{Hewson2025autecology} demonstrated explosive \vpshort{} growth
in the presence of decaying echinoderm tissue, suggesting a
saprophytic lifestyle that can sustain environmental pathogen pools
even when live \pyc{} are absent. The fjord refuge mechanism
identified by \citet{Gehman2025}---where reduced salinity and
temperature suppress Vibrio growth---operates at the community level,
further supporting a spatially varying environmental reservoir.

Second, explicitly modeling multi-species SSWD dynamics would require
parameterizing disease susceptibility, shedding rates, and population
dynamics for $>$20 additional asteroid species, most of which lack
even basic demographic data. The $P_{\text{env}}$ abstraction captures
the functional consequence (sustained pathogen pressure at the
community level) without requiring species-specific parameterization.

However, this abstraction comes at a cost. The sensitivity analysis
reveals that $P_{\text{env,max}}$ is the 4th most influential
parameter globally and the most influential parameter for the fjord
protection metric (Section~\ref{sec:r4_results}). Its interaction
ratio ($\sigma/\mu^* = 1.92$) indicates strong nonlinear coupling
with other parameters, meaning that uncertainty in $P_{\text{env}}$
propagates broadly through the model. Calibrating $P_{\text{env}}$
against field data (e.g., environmental Vibrio concentrations in
\pyc{} habitat, disease prevalence in non-\pyc{} asteroids) is a
high priority for constraining model predictions.


% ──────────────────────────────────────────────────────────────────────
\subsection{Conservation Implications}
\label{sec:disc_conservation}

\subsubsection{Evolutionary Rescue Is Too Slow}

The central finding of both the validation and sensitivity analyses
is that natural selection on polygenic resistance cannot drive
population recovery on conservation-relevant timescales. At
$K = 100{,}000$ per node, resistance trait scores actually
\emph{decline} over 20 years (mean $\Delta r_i = -0.005$;
Table~\ref{tab:val_100k}), and even the fastest-evolving trait
(recovery, $\Delta c_i \approx +0.06$) produces daily clearance
probabilities of only 0.35--0.45\%, far below what is needed to
substantially reduce disease mortality. This finding is consistent
with evolutionary rescue theory, which predicts that rescue requires
standing genetic variance $\times$ selection intensity to exceed the
rate of population decline \citep{Clement2024}. For \pyc{}, the
mismatch is severe: generation times of $\sim$5 years versus crash
timescales of $\sim$2 years mean that $<$1 generation of selection
can act before populations enter the extinction vortex.

This result has a direct conservation implication: \textbf{waiting
for natural evolution is not a viable recovery strategy}. Active
intervention through captive breeding and managed release is
essential to prevent functional extinction. The AZA SAFE program's
existing captive population of $>$2{,}500 juveniles and 130+
reproductive adults \citep{AZA2024}, combined with the successful
December 2025 pilot outplanting at Monterey \citep{Simon2025},
provides the demographic foundation for such intervention.

\subsubsection{Recovery Trait as a Breeding Target}

The model's prediction that recovery ($c_i$) evolves 7--13$\times$
faster than resistance ($r_i$) suggests a specific strategy for
captive breeding programs: selecting for pathogen clearance ability
rather than infection prevention. If \pyc{} recovery ability has a
heritable genetic basis---as implied by the strong fitness gradient
on $c_i$ in the model---then challenge experiments in captive
facilities could identify high-clearance individuals for preferential
breeding. The Pycnopodia reference genome \citep{Schiebelhut2024genome}
enables genome-wide association studies to identify the genomic basis
of clearance variation.

\subsubsection{Release Site Selection}

The consistent identification of Monterey as the most resilient
node---with the lowest crash percentage (97.1\% at $K = 100{,}000$),
highest recovery count, and strongest recovery trait
evolution---reflects an emergent property of warmer temperatures
driving stronger selection for clearance. This suggests that
southern sites may be preferable for initial releases if the goal
is to establish self-sustaining populations with elevated disease
resistance. However, warmer temperatures also increase disease
pressure, creating a tension between maximizing selective benefit
and minimizing initial mortality. The planned conservation module
(Section~\ref{sec:disc_future}) will enable explicit optimization
of release timing, location, and genetic composition.


% ──────────────────────────────────────────────────────────────────────
\subsection{Model Limitations}
\label{sec:disc_limitations}

We identify five principal limitations of the current model:

\paragraph{1.\ No multi-species dynamics.}
\modelname{} focuses exclusively on \pyc{}, abstracting all
community-level interactions into the $P_{\text{env}}$ term.
This excludes potential competitive release of sea urchins
following \pyc{} decline \citep{Rogers-Bennett2019,Galloway2023},
cross-species transmission dynamics \citep{Miner2018}, and the
possibility that recovery of \pyc{} could itself alter the
selective environment for disease. The trophic cascade from
\pyc{} loss to urchin proliferation to kelp deforestation
\citep{Meunier2024} represents a feedback loop that could
modify habitat quality and, consequently, sea star survival,
but is not represented.

\paragraph{2.\ Environmental pathogen reservoir is unconstrained.}
$P_{\text{env,max}}$ ranks 4th in global sensitivity yet has no
empirical calibration target. Field measurements of waterborne
\vpshort{} concentrations in \pyc{} habitat are needed to
constrain this parameter. Until such data are available, model
predictions about the feasibility of local disease elimination
via host removal or habitat management should be treated as
exploratory.

\paragraph{3.\ 47-parameter model with universal nonlinearity.}
The SA reveals that all 47 parameters interact nonlinearly
($\sigma/\mu^* > 1.0$ for every parameter;
Section~\ref{sec:universal_nonlinearity}). While this is a
realistic property of complex biological systems, it means that
the model cannot be calibrated by tuning individual parameters
in isolation. Joint calibration via ABC-SMC
\citep{Saltelli2008} is computationally expensive ($>$$10^4$
model evaluations) and requires well-defined summary statistics
and calibration targets, many of which are currently lacking for
\pyc{}.

\paragraph{4.\ Recovery rate has zero empirical basis.}
The base recovery rate $\rho_{\text{rec}}$ is the single most
influential parameter in the model (Section~\ref{sec:r4_ranking}),
yet whether \pyc{} can clear \vpshort{} infections at all is
unknown. The SA finding that $\rho_{\text{rec}}$ explains more
output variance than any other parameter underscores this as the
highest-priority empirical gap. Challenge-recovery experiments
in captive \pyc{} \citep{Prentice2025} could provide direct
estimates of clearance probability as a function of dose,
temperature, and individual genotype.

\paragraph{5.\ Spatial resolution.}
The validation runs use 5--11 nodes, far below the 150+ nodes
needed to represent the full NE Pacific range of \pyc{} at
ecologically meaningful resolution. Scaling analysis
(Section~\ref{sec:sa_methods}) confirms computational feasibility
($\sim$72~s for 75{,}000 agents, 150 nodes), but the reduced-node
configurations used here may underestimate the importance of
spatial heterogeneity, as demonstrated by the dramatic rank gains
of spatial parameters between R3 (3 nodes) and R4 (11 nodes).


% ──────────────────────────────────────────────────────────────────────
\subsection{Future Directions}
\label{sec:disc_future}

\subsubsection{ABC-SMC Calibration}

The immediate next step is formal calibration using approximate
Bayesian computation with sequential Monte Carlo sampling
(ABC-SMC). Summary statistics will include: (i)~range-wide
population decline ($>$90\% crash within 2 years of disease
introduction), (ii)~latitudinal mortality gradient
\citep{Hamilton2021}, (iii)~fjord protection effect
\citep{Gehman2025}, (iv)~allele frequency shifts at outlier loci
\citep{Schiebelhut2018}, and (v)~disease progression timelines
from challenge experiments \citep{Prentice2025}. The R4 SA
results (Table~\ref{tab:r4_morris_full}) provide a natural
prioritization: the top 10--15 parameters can be calibrated
jointly while fixing the remaining 32--37 at their default
values with minimal loss of model fidelity.

\subsubsection{Conservation Scenario Evaluation}

A conservation module is under development to simulate specific
management interventions:
\begin{itemize}
  \item Captive-bred release: number, timing, location, and genetic
    composition of released cohorts, parameterized from AZA SAFE
    protocols \citep{AZA2024};
  \item Assisted gene flow: introduction of cryopreserved gametes
    from genetically diverse wild-caught founders
    \citep{Hagedorn2021};
  \item Marine protected areas: local reduction of environmental
    stressors that may interact with disease susceptibility.
\end{itemize}
The December 2025 Monterey outplanting data \citep{Simon2025}
will provide the first empirical validation target for
captive-bred survival post-release.

\subsubsection{Full Coastline Network}

Expanding the spatial network to 150 nodes spanning the full NE
Pacific range of \pyc{} (Baja California to the Aleutian Islands)
will test whether the patterns identified at 5--11 nodes---the
north--south mortality gradient, fjord protection, Monterey
resilience---scale to the full metapopulation. The overwater
distance matrix for 489 candidate sites has been computed
(Section~\ref{sec:spatial_module}), and computational scaling
analysis confirms feasibility ($\sim$66~s for 150 nodes;
Section~\ref{sec:sa_methods}).

\subsubsection{Integration with Empirical Data}

The publication of the \pyc{} reference genome
\citep{Schiebelhut2024genome} enables future GWAS to identify
resistance-, tolerance-, and recovery-associated loci, providing
direct calibration targets for the genetic architecture parameters
($n_{\text{resistance}}$, $n_{\text{tolerance}}$, $n_{\text{recovery}}$,
trait-specific effect size distributions). The Koch's-postulates
confirmation of \vpshort{} as the causative agent \citep{Prentice2025}
opens the door to controlled challenge experiments that can estimate
dose-dependent infection probability, stage-specific duration,
recovery rate, and temperature sensitivity---the parameters that
the SA identifies as most influential. Combining these empirical
constraints with ABC-SMC calibration will substantially reduce
parametric uncertainty and increase confidence in conservation
scenario predictions.


% ──────────────────────────────────────────────────────────────────────
\subsection{Conclusions}
\label{sec:disc_conclusions}

\modelname{} provides a comprehensive computational framework for
exploring the eco-evolutionary dynamics of SSWD in \pyc{}. The model
reveals that evolutionary rescue through natural selection on
polygenic resistance is insufficient to prevent population collapse
on conservation timescales, that pathogen clearance (recovery) rather
than infection prevention (resistance) is the primary adaptive
pathway, and that the extinction vortex persists at ecologically
realistic population sizes. These findings reinforce the scientific
case for captive breeding and managed release as the essential
conservation strategy for this critically endangered species. The
four-round sensitivity analysis establishes clear priorities for
empirical research---recovery rate, environmental pathogen pressure,
genetic architecture, and growth rate---that will enable formal
model calibration and, ultimately, quantitative predictions for
guiding \pyc{} recovery efforts across the northeastern Pacific.
