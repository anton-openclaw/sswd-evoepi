% ══════════════════════════════════════════════════════════════════════
% DISEASE MODULE
% Fact-checked against: sswd_evoepi/disease.py, config.py
% ══════════════════════════════════════════════════════════════════════

\section{Disease Module}
\label{sec:disease_module}

The disease module implements a stochastic, environmentally driven
SEIPD+R (Susceptible--Exposed--Infectious$_1$--Infectious$_2$--Dead,
plus Recovered) compartmental framework operating at the individual level.
Each agent carries its own disease state, countdown timer, genetic
defense traits ($r_i$, $t_i$, $c_i$), and (when pathogen evolution is
enabled) the virulence phenotype $v_i$ of its infecting strain. Disease
dynamics are resolved daily at each spatial node, coupled to the
environmental forcing module for temperature-dependent rates and to the
genetics module for individual susceptibility.


\subsection{Compartmental Structure}
\label{sec:compartments}

The disease pathway consists of five compartments plus a recovery state
(Figure~\ref{fig:disease_flow}):

\begin{itemize}
  \item \textbf{S (Susceptible):} Healthy, at risk of infection.
  \item \textbf{E (Exposed):} Latently infected; not yet shedding pathogen.
        Duration is Erlang-distributed with shape $k_E = 3$.
  \item \textbf{I$_1$ (Early infectious):} Pre-symptomatic shedding at
        rate $\sigma_1(T)$. Duration is Erlang-distributed with shape
        $k_{I_1} = 2$. Agents with high clearance ability ($c_i > 0.5$) may
        recover early.
  \item \textbf{I$_2$ (Late infectious):} Symptomatic wasting with high
        shedding rate $\sigma_2(T)$. Duration is Erlang-distributed with
        shape $k_{I_2} = 2$. Agents may recover with probability
        $p_\text{rec} = \rho_\text{rec} \times c_i$ per day.
  \item \textbf{D (Dead from disease):} Carcass continues to shed pathogen
        saprophytically for a 3-day window at rate $\sigma_D$.
  \item \textbf{R (Recovered):} Immune; functionally equivalent to S for
        demographics but not susceptible to reinfection.
\end{itemize}


\subsubsection{Erlang-Distributed Stage Durations}
\label{sec:erlang}

Durations in compartments E, I$_1$, and I$_2$ are drawn from Erlang
distributions rather than geometric (exponential) distributions. The
Erlang distribution with shape parameter $k$ and rate parameter $k \mu$
has mean $1/\mu$ and coefficient of variation $\text{CV} = 1/\sqrt{k}$,
producing more realistic, peaked duration distributions compared to the
memoryless exponential \citep{wearing2005appropriate}. For each
individual entering a compartment, a duration is sampled as:
\begin{equation}
  \tau \sim \text{Erlang}(k,\; k \mu(T)),
  \quad\text{rounded to}\quad \max(1,\;\text{round}(\tau))
  \;\text{days},
  \label{eq:erlang}
\end{equation}
where $\mu(T)$ is the temperature-dependent transition rate at the
current SST (Section~\ref{sec:arrhenius}). The shape parameters are:
\begin{equation}
  k_E = 3 \;\;(\text{CV} = 0.58),
  \quad
  k_{I_1} = 2 \;\;(\text{CV} = 0.71),
  \quad
  k_{I_2} = 2 \;\;(\text{CV} = 0.71).
  \label{eq:k_shapes}
\end{equation}
Timers count down by one each day; when the timer reaches zero, the
agent transitions to the next compartment.


\subsection{Force of Infection}
\label{sec:force_of_infection}

The per-individual instantaneous hazard rate of infection is:
\begin{equation}
  \lambda_i
  = a_\text{exp}
    \;\underbrace{\frac{P_k}{K_{1/2} + P_k}}_{\text{dose--response}}
    \;\underbrace{(1 - r_i)}_{\text{resistance}}
    \;\underbrace{S_\text{sal}}_{\text{salinity}}
    \;\underbrace{f_\text{size}(L_i)}_{\text{size}},
  \label{eq:force_of_infection}
\end{equation}
where:
\begin{itemize}
  \item $a_\text{exp} = 0.75$\,d$^{-1}$ is the baseline exposure rate;
  \item $P_k$ is the local Vibrio concentration (bacteria\,mL$^{-1}$) at
        node $k$;
  \item $K_{1/2} = 87{,}000$\,bacteria\,mL$^{-1}$ is the half-infective
        dose (Michaelis--Menten dose--response);
  \item $r_i \in [0,1]$ is the individual's resistance score (immune
        exclusion; Section~\ref{sec:genetics_module});
  \item $S_\text{sal}$ is the salinity modifier
        (Section~\ref{sec:salinity_mod});
  \item $f_\text{size}(L_i)$ is the size-dependent susceptibility
        modifier (Section~\ref{sec:size_suscept}).
\end{itemize}

The discrete daily probability of infection is:
\begin{equation}
  p_\text{inf} = 1 - \exp\!\left(-\lambda_i \,\Delta t\right),
  \quad \Delta t = 1\;\text{day}.
  \label{eq:p_infection}
\end{equation}


\subsubsection{Dose--Response Function}
\label{sec:dose_response}

Pathogen exposure follows a Michaelis--Menten (saturating) dose--response:
\begin{equation}
  D(P_k) = \frac{P_k}{K_{1/2} + P_k}.
  \label{eq:dose_response}
\end{equation}
At low concentrations ($P_k \ll K_{1/2}$), infection probability scales
linearly with pathogen density; at high concentrations ($P_k \gg K_{1/2}$),
it saturates at $D \to 1$, reflecting physiological limits on pathogen
uptake.


\subsubsection{Salinity Modifier}
\label{sec:salinity_mod}

Vibrio viability is suppressed at low salinities, providing a mechanistic
basis for the reduced SSWD prevalence observed in fjord systems:
\begin{equation}
  S_\text{sal} =
  \begin{cases}
    0 & \text{if } S \leq S_\text{min} = 10\;\text{psu}, \\[4pt]
    \displaystyle\left(\frac{S - S_\text{min}}{S_\text{full} - S_\text{min}}
    \right)^{\!\eta} & \text{if } S_\text{min} < S < S_\text{full}, \\[6pt]
    1 & \text{if } S \geq S_\text{full} = 28\;\text{psu},
  \end{cases}
  \label{eq:salinity_mod}
\end{equation}
where $\eta = 2$ produces a convex response (low salinity is strongly
protective).


\subsubsection{Size-Dependent Susceptibility}
\label{sec:size_suscept}

Larger \pyc{} are more susceptible to SSWD, consistent with the
empirical finding of \citet{eisenlord2016ochre} (odds ratio 1.23 per
10\,mm increase in radius). The size modifier is:
\begin{equation}
  f_\text{size}(L_i)
  = \exp\!\left(\beta_L \,\frac{L_i - \bar{L}}{\sigma_L}\right),
  \label{eq:size_suscept}
\end{equation}
where $\beta_L = 0.021$\,mm$^{-1}$ (= $\ln 1.23 / 10$), $\bar{L} = 300$\,mm
is the reference size, and $\sigma_L = 100$\,mm normalizes the deviation.
An individual of diameter $L_i = 500$\,mm has $\sim$1.5$\times$ the
infection hazard of a 300\,mm individual.


\subsubsection{Post-Spawning Immunosuppression}
\label{sec:immunosuppression}

Spawning imposes a transient immune cost. Following each spawning event,
an individual enters a 28-day immunosuppression window during which its
effective resistance is reduced:
\begin{equation}
  r_{i,\text{eff}} = \frac{r_i}{\psi_\text{spawn}},
  \quad
  \psi_\text{spawn} = 2.0,
  \label{eq:immunosuppression}
\end{equation}
clamped to $[0, 1]$. This halves effective resistance during the
immunosuppressed period, creating an evolutionary coupling between
reproductive investment and disease vulnerability.


\subsection{Disease Progression and Recovery}
\label{sec:progression}

Disease progression rates are temperature-dependent via an Arrhenius
function (Section~\ref{sec:arrhenius}). At each daily step, disease
timers are decremented; when a timer reaches zero, the agent transitions
to the next state. Recovery can occur before timer expiry.


\subsubsection{Transition Rates}
\label{sec:transition_rates}

The base progression rates at reference temperature $T_\text{ref} = 20\,^\circ$C
are:
\begin{align}
  \mu_{E \to I_1}   &= 0.57\;\text{d}^{-1}
    & (E_a/R &= 4{,}000\;\text{K}), \label{eq:mu_EI1} \\
  \mu_{I_1 \to I_2} &= 0.40\;\text{d}^{-1}
    & (E_a/R &= 5{,}000\;\text{K}), \label{eq:mu_I1I2} \\
  \mu_{I_2 \to D}   &= 0.173\;\text{d}^{-1}
    & (E_a/R &= 2{,}000\;\text{K}). \label{eq:mu_I2D}
\end{align}
The activation energy for I$_2 \to$ D is notably lower ($E_a/R = 2{,}000$\,K
vs.\ 5,000--6,000\,K for other transitions), reflecting evidence that
terminal wasting is less temperature-sensitive than earlier disease stages
(Errata~E1).


\subsubsection{Temperature Scaling (Arrhenius)}
\label{sec:arrhenius}

All temperature-dependent rates are scaled via the Arrhenius equation:
\begin{equation}
  k(T) = k_\text{ref} \,\exp\!\left[
    \frac{E_a}{R}\left(\frac{1}{T_\text{ref}} - \frac{1}{T}\right)
  \right],
  \label{eq:arrhenius}
\end{equation}
where $T_\text{ref} = 293.15$\,K ($20\,^\circ$C) is the reference
temperature corresponding to the \vp{} thermal optimum
\citep{lambert1998virulence}, and $E_a/R$ is the activation energy
divided by the gas constant. The Arrhenius formulation ensures that
colder temperatures slow disease progression (longer E, I$_1$, I$_2$
durations) and reduce shedding rates, consistent with the observed
latitudinal gradient in SSWD severity.


\subsubsection{Tolerance: Extending I$_2$ Duration}
\label{sec:tolerance}

The tolerance trait $t_i$ operates as a damage-limitation mechanism that
reduces the effective I$_2 \to$ D mortality rate, extending survival time
while infected:
\begin{equation}
  \mu_{I_2 \to D,\text{eff}} = \mu_{I_2 \to D}(T)
    \times \left(1 - t_i \,\tau_\text{max}\right),
  \quad
  \text{floored at } 0.05 \times \mu_{I_2 \to D}(T),
  \label{eq:tolerance}
\end{equation}
where $\tau_\text{max} = 0.85$ is the maximum mortality reduction at
$t_i = 1$. The floor prevents biologically implausible indefinite
survival. The effective rate is used when sampling the I$_2$ timer
(Eq.~\ref{eq:erlang}), so tolerant individuals spend longer in I$_2$---
which may prolong both recovery opportunity and pathogen shedding.


\subsubsection{Recovery}
\label{sec:recovery}

Recovery from infection proceeds via the clearance trait $c_i$, which
represents the host's capacity for pathogen elimination.

\paragraph{Recovery from I$_2$.}
Each day, an I$_2$ individual has probability:
\begin{equation}
  p_{\text{rec},I_2} = \rho_\text{rec} \times c_i,
  \quad
  \rho_\text{rec} = 0.05\;\text{d}^{-1},
  \label{eq:recovery_I2}
\end{equation}
of transitioning to the R compartment. At $c_i = 0$ (no clearance
ability), recovery is impossible; at $c_i = 1$, the daily recovery
probability is 5\%.

\paragraph{Early recovery from I$_1$.}
Individuals with exceptionally high clearance ability ($c_i > 0.5$) can
recover during the pre-symptomatic stage:
\begin{equation}
  p_{\text{rec},I_1} =
  \begin{cases}
    0 & \text{if } c_i \leq 0.5, \\
    \rho_\text{rec} \times 2\,(c_i - 0.5) & \text{if } c_i > 0.5.
  \end{cases}
  \label{eq:recovery_I1}
\end{equation}
At $c_i = 1.0$, the early recovery probability equals $\rho_\text{rec}$,
identical to I$_2$ recovery at maximum clearance. The threshold at
$c_i = 0.5$ ensures that only rare, high-clearance individuals can clear
infection before progressing to the symptomatic stage.


\subsection{Vibrio Dynamics}
\label{sec:vibrio}

The concentration of waterborne \vp{} at node $k$ evolves according to:
\begin{equation}
  \frac{dP_k}{dt}
  = \underbrace{\sigma_1(T)\,n_{I_1} + \sigma_2(T)\,n_{I_2}
    + \sigma_D\,n_{D,\text{fresh}}}_{\text{shedding}}
  - \underbrace{\xi(T)\,P_k}_{\text{decay}}
  - \underbrace{\phi_k\,P_k}_{\text{flushing}}
  + \underbrace{P_\text{env}(T,S)}_{\text{reservoir}}
  + \underbrace{\textstyle\sum_j d_{jk}\,P_j}_{\text{dispersal}},
  \label{eq:vibrio_ode}
\end{equation}
integrated via forward Euler with $\Delta t = 1$\,day, subject to
$P_k \geq 0$.


\subsubsection{Shedding}
\label{sec:shedding}

Pathogen shedding from live infectious hosts is temperature-dependent:
\begin{align}
  \sigma_1(T) &= 5.0 \times \text{Arr}(T)
    \quad\text{(I$_1$: pre-symptomatic)}, \label{eq:shed1} \\
  \sigma_2(T) &= 50.0 \times \text{Arr}(T)
    \quad\text{(I$_2$: symptomatic)}, \label{eq:shed2}
\end{align}
where $\text{Arr}(T)$ denotes the Arrhenius factor
(Eq.~\ref{eq:arrhenius}) with $E_a/R = 5{,}000$\,K. The 10-fold
difference between early and late shedding reflects the dramatic
increase in tissue degradation and pathogen release during the wasting
phase. Rates are given in bacteria\,mL$^{-1}$\,d$^{-1}$\,host$^{-1}$
and represent field-effective values (Errata~E2).


\subsubsection{Carcass Shedding}
\label{sec:carcass}

Dead individuals ($D$ compartment) continue to shed pathogen
saprophytically for a 3-day window at a constant rate $\sigma_D =
15$\,bacteria\,mL$^{-1}$\,d$^{-1}$\,carcass$^{-1}$ (field-effective;
Code~Errata~CE-6). A ring buffer of daily disease death counts over the
most recent 3~days tracks the number of ``fresh'' carcasses contributing
to shedding:
\begin{equation}
  n_{D,\text{fresh}}(t) = \sum_{\tau=0}^{2} \text{deaths}(t - \tau).
  \label{eq:carcass_fresh}
\end{equation}


\subsubsection{Vibrio Decay}
\label{sec:decay}

\vp{} survives longer in warmer water. The natural decay rate $\xi(T)$
is interpolated log-linearly between empirical estimates:
\begin{equation}
  \xi(T) =
  \begin{cases}
    1.0\;\text{d}^{-1} & T \leq 10\,^\circ\text{C}
      \;\;(\text{half-life} \approx 0.7\;\text{d}), \\
    0.33\;\text{d}^{-1} & T \geq 20\,^\circ\text{C}
      \;\;(\text{half-life} \approx 2.1\;\text{d}), \\
    \exp\!\bigl[(1{-}f)\ln\xi_{10} + f\ln\xi_{20}\bigr]
    & \text{otherwise},
  \end{cases}
  \label{eq:decay}
\end{equation}
where $f = (T - 10)/10$ and values are clamped outside the
10--20\,$^\circ$C range. This counter-intuitive pattern (faster decay at
cold temperatures) reflects the environmental Vibrio literature
\citep{lupo2020vibrionaceae}.


\subsubsection{Environmental Reservoir}
\label{sec:reservoir}

In the ubiquitous scenario (default), \vp{} is assumed to persist in
the sediment as viable-but-non-culturable (VBNC) cells that resuscitate
when SST exceeds a threshold. The background input rate is:
\begin{equation}
  P_\text{env}(T, S)
  = P_\text{env,max}
    \;\underbrace{\frac{1}{1 + e^{-\kappa_\text{VBNC}\,(T - T_\text{VBNC})}}
    }_{\text{VBNC sigmoid}}
    \;\underbrace{g_\text{peak}(T)}_{\text{thermal performance}}
    \;\underbrace{S_\text{sal}}_{\text{salinity}},
  \label{eq:P_env}
\end{equation}
where:
\begin{itemize}
  \item $P_\text{env,max} = 500$\,bacteria\,mL$^{-1}$\,d$^{-1}$ is the
        maximum input rate;
  \item $\kappa_\text{VBNC} = 1.0\,^\circ\text{C}^{-1}$ controls the
        steepness of VBNC resuscitation;
  \item $T_\text{VBNC} = 12\,^\circ$C is the midpoint temperature;
  \item $g_\text{peak}(T)$ is a thermal performance curve with Arrhenius
        increase below $T_\text{opt} = 20\,^\circ$C and quadratic decline
        above, reaching zero at $T_\text{max} = 30\,^\circ$C.
\end{itemize}

In the invasion scenario, $P_\text{env} = 0$ everywhere until the
pathogen is explicitly introduced.


\subsection{Pathogen Evolution}
\label{sec:pathogen_evolution}

When pathogen evolution is enabled, each infectious agent carries a
continuous virulence phenotype $v_i$ that modulates disease rates via
mechanistic tradeoff functions.

\subsubsection{Virulence--Tradeoff Curves}
\label{sec:virulence_tradeoff}

More virulent strains kill faster, shed more, and progress more rapidly,
but also remove themselves from the host population sooner:
\begin{align}
  \sigma_{1,v}(T) &= \sigma_1(T) \times
    \exp\!\bigl(\alpha_\text{shed}\,\gamma_\text{early}\,(v - v^*)\bigr),
    \label{eq:sigma1_strain} \\
  \sigma_{2,v}(T) &= \sigma_2(T) \times
    \exp\!\bigl(\alpha_\text{shed}\,(v - v^*)\bigr),
    \label{eq:sigma2_strain} \\
  \mu_{I_1 \to I_2,v}(T) &= \mu_{I_1 \to I_2}(T) \times
    \exp\!\bigl(\alpha_\text{prog}\,(v - v^*)\bigr),
    \label{eq:mu_I1I2_strain} \\
  \mu_{I_2 \to D,v}(T) &= \mu_{I_2 \to D}(T) \times
    \exp\!\bigl(\alpha_\text{kill}\,(v - v^*)\bigr),
    \label{eq:mu_I2D_strain}
\end{align}
where $v^* = 0.5$ is the ancestral virulence (identity point),
$\alpha_\text{shed} = 1.5$, $\alpha_\text{prog} = 1.0$,
$\alpha_\text{kill} = 2.0$, and $\gamma_\text{early} = 0.3$
attenuates the shedding effect in the pre-symptomatic stage.

\subsubsection{Transmission and Mutation}
\label{sec:strain_inheritance}

When a new infection occurs, the infecting strain is inherited either from
a shedding individual (weighted by shedding rate) or from the environmental
reservoir (with virulence $v_\text{env} = 0.5$). The probability of
inheriting from a shedder is proportional to the total host-derived
shedding relative to total pathogen input:
\begin{equation}
  P(\text{from shedder})
  = \frac{\sum_j \sigma_j(v_j, T)}
         {\sum_j \sigma_j(v_j, T) + P_\text{env}(T, S)}.
  \label{eq:p_from_shedder}
\end{equation}

The inherited virulence is then subject to mutation:
\begin{equation}
  v_\text{new} = \text{clip}\!\left(
    v_\text{parent} + \mathcal{N}(0,\;\sigma_{v,\text{mut}}^2),
    \;v_\text{min},\;v_\text{max}
  \right),
  \label{eq:mutation}
\end{equation}
with $\sigma_{v,\text{mut}} = 0.02$, $v_\text{min} = 0$,
$v_\text{max} = 1$.


\subsection{Basic Reproduction Number}
\label{sec:R0}

The basic reproduction number provides a summary measure of epidemic
potential at a node:
\begin{equation}
  R_0 = \frac{a_\text{exp} \,S_0\,(1 - \bar{r})\,S_\text{sal}}
             {K_{1/2}\,(\xi(T) + \phi_k)}
  \left[
    \frac{\sigma_1(T)}{\mu_{I_1 \to I_2}(T)}
    + \frac{\sigma_2(T)}{\mu_{I_2 \to D,\text{eff}}(T)
            + \rho_\text{rec}\,\bar{c}}
    + \sigma_D \,\tau_D
  \right],
  \label{eq:R0}
\end{equation}
where $S_0$ is the number of susceptibles, $\bar{r}$ and $\bar{c}$ are
population-mean resistance and recovery scores, $\mu_{I_2 \to D,\text{eff}}$
incorporates population-mean tolerance (Eq.~\ref{eq:tolerance}),
$\rho_\text{rec}\,\bar{c}$ adds the recovery exit rate from I$_2$,
and $\tau_D = 3$\,days is the carcass shedding duration. The three
bracketed terms represent the pathogen contribution from each infectious
compartment (I$_1$, I$_2$, and D carcasses, respectively).


\subsection{Daily Update Sequence}
\label{sec:daily_disease_sequence}

Within each daily timestep, the disease module executes the following
steps in order:

\begin{enumerate}
  \item \textbf{Update Vibrio concentration} via Euler integration of
        Eq.~\ref{eq:vibrio_ode}, using current compartment counts and
        environmental conditions.

  \item \textbf{Transmission (S $\to$ E):} For each susceptible agent,
        compute the force of infection $\lambda_i$
        (Eq.~\ref{eq:force_of_infection}), convert to daily probability
        (Eq.~\ref{eq:p_infection}), and draw a Bernoulli infection event.
        Newly exposed agents receive an Erlang-sampled E-stage timer.
        When pathogen evolution is active, the infecting strain is
        inherited and mutated (Section~\ref{sec:strain_inheritance}).

  \item \textbf{Disease progression:} Decrement all disease timers.
        For agents with expired timers: E $\to$ I$_1$, I$_1$ $\to$ I$_2$
        (with tolerance-adjusted timer), I$_2$ $\to$ D. For agents with
        active timers: check recovery from I$_2$
        (Eq.~\ref{eq:recovery_I2}) and early recovery from I$_1$
        (Eq.~\ref{eq:recovery_I1}).

  \item \textbf{Carcass tracking:} Record today's disease deaths in the
        3-day ring buffer for saprophytic shedding.

  \item \textbf{Update diagnostics:} Recount compartments, update
        cumulative statistics (total infections, deaths, recoveries),
        track peak prevalence and peak Vibrio.
\end{enumerate}

All operations are vectorized using NumPy batch sampling and
array-level random draws for computational efficiency, achieving
$O(N)$ scaling in population size.
