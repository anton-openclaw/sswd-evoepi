\section{Screening Theory: Finding Good Founders}
\label{sec:screening}

\subsection{The basic screening problem}

Let $F(\tau)$ be the trait CDF in the wild population. To find at least one individual with $\tau \geq \tau^*$ from $n$ independent samples:

\begin{equation}
    \Prob(\text{at least one} \geq \tau^*) = 
    1 - [F(\tau^*)]^n = 1 - (1 - p)^n
    \label{eq:screening_basic}
\end{equation}
where $p = 1 - F(\tau^*) = \Prob(\tau \geq \tau^*)$ is the 
exceedance probability.

\begin{definition}[Required sample size]
For confidence level $\gamma$ (e.g., 0.95):
\begin{equation}
    n(\gamma, p) = \left\lceil 
    \frac{\ln(1 - \gamma)}{\ln(1 - p)} 
    \right\rceil
    \label{eq:required_n}
\end{equation}
\end{definition}

For small $p$, this simplifies to $n \approx -\ln(1-\gamma) / p$. 
At 95\% confidence: $n \approx 3/p$. At 50\% confidence: 
$n \approx 0.7/p$.

\subsection{Trait distribution from allele frequencies}

The exceedance probability $p$ requires the trait distribution. With $\tau = \sum_\ell \alpha_\ell x_\ell$ where $x_\ell \sim (1/2) \text{Binomial}(2, q_\ell)$, the exact distribution is a convolution of scaled binomials — tractable numerically but unwieldy analytically.

\begin{proposition}[Normal approximation]
By the Lyapunov CLT, for sufficiently many loci with no single 
dominant effect:
\begin{equation}
    \tau \overset{d}{\approx} \mathcal{N}(\mu_\tau, \sigma_\tau^2)
    \label{eq:normal_approx}
\end{equation}
where:
\begin{align}
    \mu_\tau &= \sum_\ell \alpha_\ell q_\ell 
    \label{eq:trait_mean} \\
    \sigma_\tau^2 &= \sum_\ell \frac{\alpha_\ell^2}{2} 
    q_\ell (1 - q_\ell)
    \label{eq:trait_var}
\end{align}
The factor of $1/2$ in the variance arises because $x_\ell = 
(a_1 + a_2)/2$ with $\Var[a] = q(1-q)$.
\end{proposition}

\begin{remark}
The normal approximation underestimates tail probabilities due to right-skewness, making screening predictions conservative. For quantitative planning, use the exact (simulated) distribution.
\end{remark}

\subsection{Expected best individual from a sample}

The expected maximum from a sample of $n$:
\begin{equation}
    \E[\tau_{(n)}] = \int_{-\infty}^{\infty} 
    \tau \cdot n \cdot f(\tau) \cdot [F(\tau)]^{n-1} \, d\tau
    \label{eq:expected_max}
\end{equation}
Under normality \citep{david2003order}:
\begin{equation}
    \E[\tau_{(n)}] \approx \mu_\tau + \sigma_\tau 
    \cdot \left(\Phi^{-1}\!\left(\frac{n}{n+1}\right)\right)
    \label{eq:expected_max_normal}
\end{equation}
which grows as $\sigma_\tau \sqrt{2 \ln n}$ — the mathematical basis for \textbf{diminishing returns} of screening.

\subsection{Multi-site screening}

With site-specific distributions $F_k(\tau)$, the optimal screening allocation across $K$ sites with budget $N = \sum_k n_k$ maximizes:
\begin{equation}
    \E\left[\max_{k} \tau_{(n_k)}^{(k)}\right]
    \label{eq:multisite_max}
\end{equation}
Sites with higher mean resistance yield better individuals per sample; sites with higher variance have heavier tails; very small populations may not justify sampling. The calibrated model provides the site-specific distributions needed for optimal allocation.

\subsection{Screening for complementarity}
\label{subsec:screening_complementarity}

For breeding, we need a \emph{set of founders} with complementary genotypes — protective alleles at different loci.

\begin{definition}[Locus union]
For individuals $i$ and $j$, the \textbf{locus union} is the 
number of resistance loci at which at least one parent carries 
at least one protective allele:
\begin{equation}
    U(i, j) = \sum_{\ell=1}^{n_R} 
    \mathbf{1}\!\left[(a_{i,\ell,1} + a_{i,\ell,2}) > 0 
    \;\lor\; (a_{j,\ell,1} + a_{j,\ell,2}) > 0\right]
    \label{eq:locus_union}
\end{equation}
Maximum value: $n_R$ (every locus covered).
\end{definition}

\begin{definition}[Complementarity score]
\begin{equation}
    C(i, j) = U(i, j) - O(i, j)
    \label{eq:complementarity}
\end{equation}
where $O(i,j)$ is the overlap. High $C$ means parents cover different loci, so offspring can achieve higher resistance than either parent.
\end{definition}

\begin{proposition}[Expected union of two random individuals]
If locus $\ell$ has protective allele frequency $q_\ell$, the 
probability that at least one of two random individuals carries 
$\geq 1$ protective allele at this locus is:
\begin{equation}
    \Prob(\text{locus } \ell \text{ covered}) = 
    1 - (1 - q_\ell)^4
    \label{eq:locus_coverage_prob}
\end{equation}
(since each individual has 2 independent allele draws, so 4 total).
The expected union is:
\begin{equation}
    \E[U] = \sum_{\ell=1}^{n_R} 
    \left[1 - (1 - q_\ell)^4\right]
    \label{eq:expected_union}
\end{equation}
\end{proposition}

\subsection{Screening cost model}
\label{subsec:screening_cost}

A simple cost model:
\begin{equation}
    C_{\text{total}} = c_{\text{collect}} \cdot n + 
    c_{\text{assay}} \cdot n + 
    c_{\text{hold}} \cdot n_{\text{keep}} \cdot T_{\text{hold}}
    \label{eq:screening_cost}
\end{equation}
where $n$ is total screened, $n_{\text{keep}}$ retained, and $T_{\text{hold}}$ is holding duration. Cost parameters must be supplied by conservation practitioners.

\subsection{Phenotyping constraints: the challenge assay}
\label{subsec:binary_phenotyping}

The theory above assumes continuous resistance measurement. In practice, no validated marker panel exists for SSWD resistance in \pyc{}. The primary phenotyping method is the \textbf{challenge assay}: expose to \textit{Vibrio pectenicida} and observe outcomes \citep{prentice2025koch}. This section develops a framework for what challenge assays reveal about underlying genetics and how this constrains breeding design.

\note{All numerical examples in this section use pre-calibration 
parameter values as illustrations. The specific ratios and 
probabilities will change after model calibration. What persists 
is the \emph{structure} of the problem: the relationships between 
traits, observables, assay design, and selection response.}

\subsubsection{What does a challenge assay actually measure?}

With regular monitoring (e.g., daily health assessment), a challenge assay produces four observables:

\begin{enumerate}
    \item \textbf{Whether symptoms appear} — depends on resistance $r_i$ and stochastic luck (infection probability draw).
    \item \textbf{Whether a symptomatic individual recovers or dies} — depends on recovery $c_i$ and tolerance $t_i$.
    \item \textbf{Time from symptom onset to death} — influenced by tolerance (\Cref{eq:tolerance}).
    \item \textbf{Incubation period} — governed by E $\to$ I$_1$ rate, which is \textbf{not trait-dependent} (temperature only) and therefore \textbf{uninformative} for selection.
\end{enumerate}

\subsubsection{Three survival pathways}

Survivors fall into two genetically distinct categories:

\begin{description}
    \item[Pathway A --- Resistant (never infected):] Avoided infection via $r_i$ (\Cref{eq:infection_hazard}). Enriched for resistance alleles.
    \item[Pathway B --- Tolerant/Recoverer (infected, survived):] Cleared pathogen via $c_i$ (\Cref{eq:recovery,eq:early_recovery}), aided by $t_i$ (\Cref{eq:tolerance}). Enriched for tolerance and recovery alleles.
\end{description}

Without diagnostics (e.g., PCR for pathogen load), both categories appear as ``survived.'' Symptom monitoring provides a partial decomposition.

\begin{remark}
The relative sizes of Pathways A and B depend on parameters and will change with calibration. The \emph{structure} persists: resistance acts before infection, tolerance/recovery after. The assay conflates them unless symptom status is tracked.
\end{remark}

\subsubsection{Dose as a design parameter}
\label{subsubsec:dose}

Dose is a \textbf{design choice} that determines assay information content. The fraction avoiding infection depends on both dose and resistance:
\begin{equation}
    p_{\text{avoid}}(r_i, P) = \exp\!\left(-a \cdot 
    \frac{P}{K_{1/2} + P} \cdot (1 - r_i) \cdot S_{\text{sal}} 
    \cdot f_{\text{size}}\right)
    \label{eq:avoid_infection}
\end{equation}

This creates a dose trade-off:

\begin{center}
\begin{tabular}{p{0.28\textwidth}p{0.30\textwidth}p{0.30\textwidth}}
\toprule
\textbf{Dose regime} & \textbf{Advantage} & \textbf{Disadvantage} \\
\midrule
High dose & 
Strong discrimination: survivors almost certainly resistant & 
High mortality, loses moderately resistant stock \\
\addlinespace
Moderate dose & 
Balanced survival with moderate signal & 
Mixed signal: luck + genetics both contribute \\
\addlinespace
Low dose & 
Preserves stock & 
Weak discrimination: most survival is stochastic \\
\bottomrule
\end{tabular}
\end{center}

The optimal dose depends on stock availability, founder requirements, and whether the goal is identifying the most resistant few or broadly enriching. The calibrated model simulates each choice.

\subsubsection{Heritability of the binary phenotype}

The binary outcome is analyzed via the \textbf{threshold model} \citep{falconer1996introduction}, relating observed-scale heritability to the underlying continuous liability:
\begin{equation}
    h^2_{\text{obs}} = h^2_{\text{liability}} \cdot 
    \frac{z^2}{P(1 - P)}
    \label{eq:h2_observed}
\end{equation}
where $P$ is the population survival rate and 
$z = \phi(\Phi^{-1}(P))$ is the standard normal ordinate.

At low $P$ (stringent): moderate $h^2_{\text{obs}}$. At $P \approx 0.5$: minimum $h^2_{\text{obs}}$ (maximally noisy). At high $P$ (weak): higher $h^2_{\text{obs}}$ but limited selection differential. Key implication: \textbf{assay information content depends on dose, genetics, and environment}; as breeding advances (shifting $P$ upward), the assay must be recalibrated.

\subsubsection{Repeated exposures}

Re-challenging survivors amplifies discrimination exponentially:
\begin{equation}
    p_{\text{surv}}^{(k)} = \left(p_{\text{surv},i}\right)^k
    \label{eq:repeated_exposure}
\end{equation}
Two exposures roughly square the survival probability ratio between phenotypic classes. Echinoderm biology facilitates this: lacking adaptive immunity, recovered individuals return to the susceptible state (R $\to$ S), enabling sequential screening within a single cohort. However, each round kills unlucky but genetically valuable individuals, requiring cost--benefit optimization.

\subsubsection{Family-based selection}
\label{subsubsec:family_selection}

High fecundity enables a different approach: challenge \textbf{offspring groups} from controlled crosses and compare \textbf{family survival rates}. For family $j$ with $n_j$ challenged and $k_j$ surviving, $\hat{p}_j = k_j / n_j$ estimates parental breeding value with precision:
\begin{equation}
    \text{SE}(\hat{p}_j) = \sqrt{\frac{\hat{p}_j(1 - \hat{p}_j)}
    {n_j}}
    \label{eq:family_se}
\end{equation}

With $n_j = 100$, a family with $p = 0.30$ has SE $\approx 0.046$ — converting noisy individual binary outcomes into a \textbf{precise family-level continuous phenotype}. Family-mean heritability:
\begin{equation}
    h^2_{\text{family}} = \frac{h^2}{1 + (n - 1) r_{ICC}}
    \cdot n \cdot r_{ICC}
    \label{eq:h2_family}
\end{equation}
where $r_{ICC}$ is the intraclass correlation and $n$ is family size. For large families, $h^2_{\text{family}} \to 1$.

\subsubsection{Strategic crossing of survival categories}

If Pathways A and B can be distinguished, strategic crossing exploits their different genetic content:

\begin{center}
\begin{tabular}{lcc}
\toprule
& \textbf{Pathway A} & \textbf{Pathway B} \\
\midrule
Enriched for & Resistance ($r$) & Tolerance ($t$) + Recovery ($c$) \\
Best cross with & Other Pathway A & Pathway A (complementary traits) \\
Offspring advantage & Stacked resistance & Resistance + recovery \\
\bottomrule
\end{tabular}
\end{center}

A$\times$A maximizes resistance; A$\times$B creates offspring hedging with both resistance and tolerance/recovery alleles.

\subsubsection{Genomic markers: the transformative alternative}

If the \citet{schiebelhut2018} loci can be validated as resistance markers, non-lethal genotyping would replace challenge assays — eliminating false negatives, providing continuous trait measurement, enabling screening without killing, and making the \Cref{sec:screening} framework directly applicable. \textbf{Marker validation should be a high priority for the conservation program.}

\subsubsection{What the model provides}

The calibrated model serves as a \textbf{virtual laboratory}: generate populations with known genetics, simulate challenge assays at specified doses, track individual outcomes with timing, compute trait distributions per outcome category, evaluate selection differentials across assay designs, and predict multi-generation trajectories under realistic phenotyping constraints.

