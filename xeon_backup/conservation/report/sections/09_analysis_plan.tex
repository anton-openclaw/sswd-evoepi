\section{Analysis Plan}
\label{sec:analysis_plan}

This section specifies the concrete analyses to execute once 
the model is calibrated. Each analysis maps to a code template 
in \texttt{conservation/analyses/}.

\subsection{Analysis 1: Current Genetic State}
\label{subsec:analysis1}

\textbf{Goal:} Predict the trait distributions and genetic 
diversity of surviving \pyc{} at each of the 11 stepping-stone 
sites in 2026.

\textbf{Method:}
\begin{enumerate}
    \item Initialize the model with calibrated parameters, 
    pre-epidemic population sizes and allele frequencies
    \item Run from 2013 (pre-epidemic) to 2026 using actual 
    satellite SST time series
    \item At the 2026 endpoint, extract full genotype arrays 
    for all surviving individuals at each node
    \item Compute per-site: trait distributions 
    ($r$, $t$, $c$), $V_A$, $H_e$, $N_e$, allele frequencies 
    per locus
    \item Repeat for 50 random seeds to characterize 
    stochastic variation
\end{enumerate}

\textbf{Output:}
\begin{itemize}
    \item Site $\times$ trait mean matrix (11 sites $\times$ 3 traits)
    \item Per-site screening effort tables (\Cref{eq:required_n})
    \item Latitude $\times$ year heatmap of resistance evolution
    \item Genetic diversity gradient (north--south)
    \item Confidence intervals from seed ensemble
\end{itemize}

\textbf{Code:} \texttt{analyses/01\_current\_genetic\_state.py}

\textbf{Requirements:}
\begin{itemize}
    \item[$\square$] Calibrated model parameters
    \item[$\square$] Satellite SST time series 2013--2026 
    (extend current climatology)
    \item[$\square$] Genotype snapshot capability in recorder 
    (model extension needed)
    \item[$\square$] Pre-epidemic population size estimates by site
\end{itemize}

\subsection{Analysis 2: Screening Effort by Site}
\label{subsec:analysis2}

\textbf{Goal:} For each site's predicted 2026 population, 
compute the number of individuals to screen for various 
resistance thresholds.

\textbf{Method:}
\begin{enumerate}
    \item From Analysis 1 endpoints, compute empirical trait 
    CDFs per site
    \item Apply \Cref{eq:required_n} for thresholds 
    $r^* \in \{0.20, 0.25, 0.30, 0.35, 0.40\}$
    \item Compute expected best-of-$n$ (\Cref{eq:expected_max}) 
    for practical sample sizes
    \item Compute complementarity statistics 
    (\Cref{eq:locus_union,eq:complementarity}) for top 
    individuals
    \item Derive multi-site optimal sampling allocation 
    (\Cref{eq:multisite_max})
\end{enumerate}

\textbf{Output:}
\begin{itemize}
    \item Per-site screening effort tables (threshold $\times$ 
    required $n$ $\times$ confidence)
    \item Cross-site comparison: which sites have the best 
    screening return?
    \item Optimal sampling allocation across sites for a 
    fixed total budget
    \item Complementarity analysis of top founders
\end{itemize}

\textbf{Code:} \texttt{analyses/02\_screening\_effort.py}

\subsection{Analysis 3: Breeding Program Optimization}
\label{subsec:analysis3}

\textbf{Goal:} Compare breeding strategies and determine 
optimal program design.

\textbf{Method:}
\begin{enumerate}
    \item Draw founders from Analysis 1 endpoint populations
    \item Simulate 8--15 generations of breeding under each 
    strategy:
    \begin{itemize}
        \item Random mating (baseline)
        \item Assortative mating by resistance
        \item Complementary mating (maximize locus union)
        \item Optimal contribution selection 
        (resistance gain constrained by $\Delta F$)
        \item Selection index (weighted multi-trait)
    \end{itemize}
    \item Track per generation: $\bar{r}$, $\max(r)$, $V_A$, 
    $H_e$, $F$, $N_e$, loci fixed, alleles lost
    \item Vary: number of founders (50--500), selection 
    intensity ($p = 0.01$--$0.50$), family structure
    \item 100 replicate seeds per scenario
\end{enumerate}

\textbf{Output:}
\begin{itemize}
    \item Strategy $\times$ generation trajectory plots
    \item Generations to resistance targets by strategy
    \item Gain--diversity frontier (Pareto plot of $\Delta r$ 
    vs.\ $\Delta F$ per generation)
    \item Optimal program design recommendation
    \item Sensitivity to number of founders
\end{itemize}

\textbf{Code:} \texttt{analyses/03\_breeding\_optimization.py}

\textbf{Requirements:}
\begin{itemize}
    \item[$\square$] Inbreeding tracking in breeding simulator
    \item[$\square$] Optimal contribution selection algorithm
    \item[$\square$] Genomic relationship matrix computation
\end{itemize}

\subsection{Analysis 4: Reintroduction Scenarios}
\label{subsec:analysis4}

\textbf{Goal:} Predict outcomes of different reintroduction 
strategies using the full spatial model.

\textbf{Method:}
\begin{enumerate}
    \item From Analysis 3, generate captive-bred populations 
    at different resistance levels (3, 5, 8 generations of 
    breeding)
    \item Inject these into the spatial model at specified 
    nodes and times
    \item Run 20--50 years forward
    \item Vary: release size (100--5000), release location 
    (single node, multiple nodes, stepping-stone), release 
    timing (relative to disease season), release frequency 
    (one-time, annual, every 2 years), genetic composition 
    of released stock
    \item Track: population trajectories, allele frequency 
    trajectories, persistence probability, mean resistance 
    at 10/20/50 years
\end{enumerate}

\textbf{Output:}
\begin{itemize}
    \item Scenario comparison table (success metrics by strategy)
    \item Critical release size threshold (below which release 
    has no lasting effect)
    \item Optimal release node(s) — considering connectivity
    \item Time to population recovery under best strategies
    \item Allele spread maps (how introduced alleles propagate 
    via larval dispersal)
\end{itemize}

\textbf{Code:} \texttt{analyses/04\_reintroduction\_scenarios.py}

\textbf{Requirements:}
\begin{itemize}
    \item[$\square$] Release mechanism in model (introduce 
    individuals at specified node/time)
    \item[$\square$] Calibrated model parameters
    \item[$\square$] Computational budget (many scenarios 
    $\times$ many seeds $\times$ long runs)
\end{itemize}

\subsection{Analysis 5: Integrated Recommendations}
\label{subsec:analysis5}

\textbf{Goal:} Synthesize Analyses 1--4 into actionable 
conservation recommendations.

\textbf{Deliverables:}
\begin{enumerate}
    \item Recommended sampling protocol (which sites, how 
    many, what to genotype)
    \item Recommended breeding program design (strategy, 
    founder count, generations, selection intensity)
    \item Recommended release strategy (where, when, how 
    many, how often)
    \item Timeline from program initiation to population 
    recovery targets
    \item Key uncertainties and decision points for 
    adaptive management
\end{enumerate}

\textbf{Code:} \texttt{analyses/05\_recommendations.py} 
(generates summary tables and figures for the paper)

\subsection{Computational budget estimate}

\begin{center}
\begin{tabular}{lccr}
\toprule
Analysis & Runs & Wall time (est.) & Platform \\
\midrule
1. Genetic state & $50 \times 11 \times 13\text{yr}$ & 
4--8 hours & Xeon \\
2. Screening & analytical + sampling & minutes & Local \\
3. Breeding & $5 \times 100 \times 15\text{gen}$ & 
1--2 hours & Local \\
4. Reintroduction & $\sim$500 $\times$ 50 $\times$ 20yr & 
2--5 days & Xeon \\
5. Recommendations & analytical & minutes & Local \\
\bottomrule
\end{tabular}
\end{center}

Total: $\sim$3--6 days of Xeon time for the full suite, 
comparable to the current Sobol R4 run.
