\section{Reintroduction Genetics}
\label{sec:reintroduction}

Releasing captive-bred individuals into the wild creates a 
genetic mixing event. The outcome depends on the genetic 
composition of the released stock, the recipient population, 
and the ongoing disease dynamics. This section develops the 
theory.

\subsection{Allele frequency shift from supplementation}

When $N_r$ captive-bred individuals (with allele frequency 
$q_r$ at locus $\ell$) are released into a wild population 
of $N_w$ individuals (with frequency $q_w$):
\begin{equation}
    q_{\text{post}} = \frac{N_w \cdot q_w + N_r \cdot q_r}
    {N_w + N_r}
    \label{eq:admixture}
\end{equation}

The frequency shift is:
\begin{equation}
    \Delta q = q_{\text{post}} - q_w = 
    \frac{N_r}{N_w + N_r} \cdot (q_r - q_w)
    \label{eq:freq_shift}
\end{equation}

\begin{remark}
The shift is proportional to $N_r / (N_w + N_r)$. When wild 
populations are severely depleted ($N_w \ll K$), a modest 
release can have a large genetic impact. This is the 
\textbf{genetic rescue} scenario — the depleted wild 
population is easily ``swamped'' by captive stock.
\end{remark}

\subsection{Genetic rescue vs.\ genetic swamping}

\begin{definition}[Genetic rescue]
Introduction of new genetic variation into an inbred or 
genetically depauperate population, increasing fitness 
through heterosis (masking of deleterious recessives) and/or 
introduction of beneficial alleles 
\citep{whiteley2015genetic}.
\end{definition}

\begin{definition}[Genetic swamping]
Replacement of locally adapted alleles by maladapted 
introduced alleles, reducing population fitness 
\citep{rhymer1996extinction}.
\end{definition}

For \pyc{}, the risk of genetic swamping is low because:
\begin{enumerate}
    \item The captive stock is derived from wild populations 
    (no interspecific hybridization)
    \item The target trait (disease resistance) is universally 
    beneficial across the range
    \item Local adaptation to non-disease factors (temperature, 
    salinity) is likely weak relative to the disease-driven 
    selection pressure
\end{enumerate}

The primary risk is \textbf{outbreeding depression}: if 
captive-bred stock from one population is released into a 
genetically divergent population, offspring may have reduced 
fitness due to disruption of co-adapted gene complexes. 
However, for traits controlled by our 51 additive loci, 
this is not possible by construction (no epistasis). 
Outbreeding depression would come from the rest of the genome, 
which we do not model.

\subsection{Effective migration rate}

In the spatial model, captive-bred releases function as a 
human-mediated migration event. The effective migration rate 
at node $k$ from a release of $N_r$ individuals is:
\begin{equation}
    m_k^{\text{eff}} = \frac{N_r}{N_k + N_r}
    \label{eq:effective_migration}
\end{equation}

For the genetic effects to be sustained, releases must either:
\begin{enumerate}
    \item Be large enough to shift allele frequencies 
    significantly in a single event
    \item Be repeated over multiple generations to maintain 
    elevated frequencies against the erosion from wild-type 
    reproduction
\end{enumerate}

\subsection{Persistence of introduced alleles}

After a one-time release, the introduced allele frequency 
decays if the captive stock has lower overall fitness 
(e.g., maladaptation to local conditions, inbreeding depression). 
However, if the introduced alleles confer a \emph{selective 
advantage} (higher disease resistance), they will increase in 
frequency:
\begin{equation}
    \Delta q_\ell = s_\ell \cdot q_\ell (1 - q_\ell)
    \label{eq:freq_change_selection}
\end{equation}
where $s_\ell$ is the selection coefficient at locus $\ell$. 
In our model, this is determined by the fitness function 
(\Cref{eq:fitness_function}) and the local disease pressure.

\begin{proposition}[Resistance alleles are self-sustaining]
\label{prop:self_sustaining}
In populations where disease pressure maintains $s_\ell > 0$ 
for resistance alleles, a one-time release that shifts $q_\ell$ 
above the drift threshold ($q_\ell > 1/\sqrt{N_e}$) will lead 
to continued frequency increase through natural selection. The 
release provides the initial ``push''; selection does the rest.
\end{proposition}

This is the optimistic scenario for conservation: captive-bred 
stock doesn't need to permanently replace the wild population. 
It just needs to inject enough resistant alleles that natural 
selection can amplify them.

\subsection{Release strategy optimization}

Given a total budget of $N_{\text{total}}$ captive-bred 
individuals, how should they be distributed?

\subsubsection{Spatial allocation}

For $K$ release sites with wild populations $N_1, \ldots, N_K$:
\begin{equation}
    \text{max}_{\{n_k\}} \; \sum_{k=1}^{K} 
    \phi_k(n_k, N_k, q_k^w, q^r)
    \quad \text{s.t.} \quad 
    \sum_k n_k = N_{\text{total}}
    \label{eq:spatial_allocation}
\end{equation}
where $\phi_k$ is a node-specific benefit function 
(e.g., expected 20-year population size, or mean resistance 
at year 20).

Intuition:
\begin{itemize}
    \item Depleted populations benefit most per released 
    individual (higher $m_k^{\text{eff}}$)
    \item But populations near extinction may not be viable 
    regardless (wasted effort)
    \item Populations with some natural resistance gain 
    less from supplementation
    \item Connectivity matters: releases at well-connected 
    nodes spread alleles further via larval dispersal
\end{itemize}

\subsubsection{Temporal allocation}

Should we release all at once or spread across years?

\textbf{Arguments for single large release:}
\begin{itemize}
    \item Maximizes initial frequency shift 
    (\Cref{eq:freq_shift})
    \item Dilutes local disease pressure through density effects
    \item Immediate demographic rescue
\end{itemize}

\textbf{Arguments for repeated releases:}
\begin{itemize}
    \item Hedges against stochastic die-off of released cohorts
    \item Allows improving genetic quality as breeding 
    program advances
    \item Maintains genetic influx against drift erosion
\end{itemize}

The optimal strategy depends on disease dynamics (seasonal 
peaks, inter-annual variation) and the breeding program's 
trajectory — questions the calibrated model can answer.

\subsection{Monitoring and adaptive management}

Post-release monitoring should track:
\begin{enumerate}
    \item \textbf{Survival} of released individuals (mark--recapture)
    \item \textbf{Allele frequencies} at marker loci 
    (non-invasive genetic sampling)
    \item \textbf{Resistance phenotype} if challenge assays 
    are feasible
    \item \textbf{Population growth rate} ($\lambda$) — 
    is the population recovering?
    \item \textbf{Connectivity signal} — are introduced alleles 
    spreading to adjacent nodes via larvae?
\end{enumerate}

The model provides predicted trajectories for all of these, 
which serve as benchmarks for assessing whether the 
reintroduction is on track.
