\section{Inbreeding and Genetic Diversity}
\label{sec:inbreeding}

\subsection{Inbreeding coefficient}

The inbreeding coefficient $F_i$ — probability that two alleles at a random locus are identical by descent (IBD):
\begin{equation}
    F_i = \Prob(\text{two alleles at a locus are IBD})
    \label{eq:f_definition}
\end{equation}

Estimated from genotype data as excess homozygosity:
\begin{equation}
    \hat{F}_i = 1 - \frac{H_{\text{obs},i}}{H_{\text{exp}}}
    = 1 - \frac{\frac{1}{L}\sum_\ell 
    \mathbf{1}[a_{i,\ell,1} \neq a_{i,\ell,2}]}
    {\frac{1}{L}\sum_\ell 2 q_\ell (1 - q_\ell)}
    \label{eq:f_genomic}
\end{equation}
\subsection{Rate of inbreeding}

Inbreeding accumulates at:
\begin{equation}
    \Delta F = \frac{1}{2 N_e}
    \label{eq:delta_f}
\end{equation}
After $g$ generations:
\begin{equation}
    F_g = 1 - (1 - \Delta F)^g \approx 1 - e^{-g/(2N_e)}
    \label{eq:f_after_g}
\end{equation}

\subsubsection{Effective population size under selection}

With $N_m$ male and $N_f$ female parents:
\begin{equation}
    N_e = \frac{4 N_m N_f}{N_m + N_f}
    \label{eq:ne_unequal_sex}
\end{equation}
Equal sex allocation ($N_m = N_f = N_s/2$) gives $N_e = N_s$. With unequal family sizes ($\sigma_k^2$):
\begin{equation}
    N_e = \frac{4N - 4}{2 + \sigma_k^2}
    \label{eq:ne_family_size}
\end{equation}
Equal family sizes ($\sigma_k^2 = 0$) double $N_e$ vs.\ Poisson.

\subsection{The 50/500 rule}
\label{subsec:50_500}

Classic conservation thresholds \citep{franklin1980evolutionary, jamieson2012applicability}: $N_e \geq 50$ avoids severe short-term inbreeding ($\Delta F \leq 1\%$); $N_e \geq 500$ maintains long-term evolutionary potential.

\begin{center}
\begin{tabular}{cccc}
\toprule
$N_e$ & $\Delta F$/gen & $F$ after 5 gen & $F$ after 10 gen \\
\midrule
25 & 2.0\% & 9.6\% & 18.3\% \\
50 & 1.0\% & 4.9\% & 9.6\% \\
100 & 0.5\% & 2.5\% & 4.9\% \\
200 & 0.25\% & 1.2\% & 2.5\% \\
500 & 0.10\% & 0.5\% & 1.0\% \\
\bottomrule
\end{tabular}
\end{center}

\begin{remark}
With $N_e \approx 50$--$100$ and 2-year generations, reaching $F = 10\%$ takes 5--10 generations (10--20 years) — a real constraint for multi-generation selective breeding.
\end{remark}

\subsection{Inbreeding depression}
\label{subsec:inbreeding_depression}

Expected fitness decline from inbreeding:
\begin{equation}
    \bar{w}(F) = \bar{w}(0) \cdot e^{-BF}
    \label{eq:inbreeding_depression}
\end{equation}
where $B$ is lethal equivalents per diploid genome (2--12 for marine invertebrates; \citealp{obrien1994genetic, hedrick2002inbreeding}).

\begin{remark}
Our model \textbf{does not implement inbreeding depression} — the 51 loci control disease traits only. Adding it requires either explicit deleterious loci or a phenotypic $F$-penalty. Flagged as a \textbf{priority extension}.
\end{remark}

\subsection{Diversity metrics}

\subsubsection{Expected heterozygosity}
\begin{equation}
    H_e = \frac{1}{L} \sum_{\ell=1}^{L} 2 q_\ell (1 - q_\ell)
    \label{eq:he}
\end{equation}

\subsubsection{Allelic richness}
Number of polymorphic loci. Under drift:
\begin{equation}
    \Prob(\text{allele lost by generation } g) \approx 
    1 - e^{-g/(2N_e)} \quad \text{(for rare alleles)}
    \label{eq:allele_loss}
\end{equation}

\subsubsection{Additive genetic variance}
\begin{equation}
    V_A^{(g)} = \sum_\ell 2 q_\ell^{(g)} 
    (1 - q_\ell^{(g)}) \alpha_\ell^2
    \label{eq:va_over_time}
\end{equation}
This is the ``fuel'' for future selection response. Once 
$V_A \to 0$, no further genetic gain is possible through 
selection alone.

\subsection{Managing the gain--diversity trade-off}
\label{subsec:gain_diversity}

The fundamental trade-off: stronger selection increases 
short-term genetic gain but accelerates diversity loss.

\subsubsection{Constrained optimization approach}

The optimal contribution selection framework 
(\Cref{eq:ocs}) solves this formally. In practice, for 
our discrete-locus model, we can implement a simpler 
version:

\begin{enumerate}
    \item Rank all candidates by breeding value (resistance 
    score, or selection index)
    \item Starting from the top, add candidates to the 
    breeding pool
    \item For each candidate, compute the marginal change 
    in $N_e$ if they are included
    \item Stop when either: (a) the target pool size is 
    reached, or (b) including the next candidate would 
    push $\Delta F$ above the threshold
\end{enumerate}

\subsubsection{Practical guideline}

For a breeding program targeting resistance while maintaining 
diversity:
\begin{equation}
    N_{\text{breeding}} \geq 
    \max\!\left(N_{\min}^{(\Delta F)}, \; 
    N_{\min}^{(\text{alleles})}\right)
    \label{eq:breeding_pool_size}
\end{equation}
where $N_{\min}^{(\Delta F)}$ ensures $\Delta F \leq$ target, 
and $N_{\min}^{(\text{alleles})}$ ensures retention of rare 
alleles. For 51 biallelic loci with minimum allele frequency 
$q_{\min} \approx 0.01$:
\begin{equation}
    N_{\min}^{(\text{alleles})} \approx 
    \frac{1}{q_{\min}} = 100
    \label{eq:nmin_alleles}
\end{equation}
(need $\sim$100 individuals to expect $\geq 1$ copy of a 
1\% frequency allele).
