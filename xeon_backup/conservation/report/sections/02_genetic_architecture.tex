\section{Genetic Architecture}
\label{sec:architecture}

\subsection{Locus structure}

The model implements 51 biallelic loci, motivated by the 
\citet{schiebelhut2018} genome scan identifying 51 loci with 
significant allele frequency shifts between pre- and post-SSWD 
\pyc{} populations, partitioned into three non-overlapping 
trait blocks:

\begin{align}
    n_R &= 17 \quad \text{(resistance loci, indices 0--16)} \\
    n_T &= 17 \quad \text{(tolerance loci, indices 17--33)} \\
    n_C &= 17 \quad \text{(recovery loci, indices 34--50)}
\end{align}

The partition is configurable (\texttt{n\_resistance}, 
\texttt{n\_tolerance}, \texttt{n\_recovery} in model config), 
subject to the constraint $n_R + n_T + n_C = 51$.

\subsection{Genotype representation}

Each individual $i$ carries a diploid genotype at each locus $\ell$:
\begin{equation}
    g_{i,\ell} = (a_{i,\ell,1}, \, a_{i,\ell,2}) \in \{0, 1\}^2
\end{equation}
where $a = 1$ denotes the derived (protective) allele and $a = 0$ 
the ancestral allele. The genotypic value at locus $\ell$ is the 
allele mean:
\begin{equation}
    x_{i,\ell} = \frac{a_{i,\ell,1} + a_{i,\ell,2}}{2} \in \{0, \, 0.5, \, 1\}
\end{equation}

\subsection{Effect sizes}

Each trait block has effect sizes $\alpha_\ell$ drawn from 
$\text{Exp}(\lambda)$ and normalized to sum to 1:
\begin{equation}
    \alpha_\ell^{(\text{raw})} \sim \text{Exp}(1), \qquad
    \alpha_\ell = \frac{\alpha_\ell^{(\text{raw})}}
    {\sum_{j=1}^{n} \alpha_j^{(\text{raw})}}
    \label{eq:effect_sizes}
\end{equation}
so $\sum_{\ell=1}^{n} \alpha_\ell = 1$ exactly, sorted descending. This creates a realistic distribution where a few loci contribute disproportionately \citep{barton2002understanding}.

\begin{remark}
Because $\sum \alpha_\ell = 1$, the trait value of a fully 
homozygous-derived individual ($x_{i,\ell} = 1$ at all loci) is 
exactly $1.0$. This makes trait values interpretable as a fraction 
of the theoretical maximum.
\end{remark}

\subsection{Trait computation}

The trait score for individual $i$ in trait block $\mathcal{L}$ 
(with effect sizes $\boldsymbol{\alpha}$) is:
\begin{equation}
    \tau_i = \sum_{\ell \in \mathcal{L}} \alpha_\ell \, x_{i,\ell}
    = \boldsymbol{\alpha}^T \mathbf{x}_i
    \label{eq:trait_score}
\end{equation}

This is \textbf{strictly additive}: no dominance, no epistasis. Trait values are continuous on $[0, 1]$, heterozygotes are exactly intermediate, and traits from different blocks are genetically independent (no pleiotropy, no linkage).

\subsection{Three traits and their phenotypic effects}

Each trait controls exactly one aspect of the host--pathogen interaction:

\subsubsection{Resistance ($r_i$)}
Immune exclusion. Reduces the instantaneous infection hazard rate:
\begin{equation}
    \lambda_i = a \cdot \frac{P_k}{K_{1/2} + P_k} \cdot (1 - r_i) 
    \cdot S_{\text{sal}} \cdot f_{\text{size}}(L_i)
    \label{eq:infection_hazard}
\end{equation}
Daily infection probability: $p_{\text{inf}} = 1 - e^{-\lambda_i}$ ($r_i = 1$: immune; $r_i = 0$: fully susceptible).

\note{Resistance is multiplicative in the hazard rate, so its marginal fitness benefit scales with pathogen pressure — strongly selected in high-disease environments, nearly neutral in low-disease ones.}

\subsubsection{Tolerance ($t_i$)}
Damage limitation. Reduces the I$_2 \to$ D transition rate:
\begin{equation}
    \mu_{I_2D}^{\text{eff}} = \mu_{I_2D} \cdot 
    \left(1 - t_i \cdot \tau_{\max}\right),
    \qquad \text{floor: } \mu_{I_2D}^{\text{eff}} \geq 
    0.05 \cdot \mu_{I_2D}
    \label{eq:tolerance}
\end{equation}
where $\tau_{\max} = 0.85$. At $t_i = 1$, I$_2$ survival is extended $6.67\times$; the 5\% floor prevents immortality.

\subsubsection{Recovery ($c_i$)}
Pathogen clearance. Daily probability of recovery from I$_2$:
\begin{equation}
    p_{\text{rec}} = \rho_{\text{rec}} \cdot c_i
    \label{eq:recovery}
\end{equation}
where $\rho_{\text{rec}} = 0.05$ d$^{-1}$. Additionally, early 
recovery from I$_1$ is possible if $c_i > 0.5$:
\begin{equation}
    p_{\text{early}} = \rho_{\text{rec}} \cdot 2 \cdot (c_i - 0.5),
    \qquad c_i > 0.5
    \label{eq:early_recovery}
\end{equation}

\subsection{Allele frequency initialization}

Per-locus protective allele frequencies are drawn from a scaled 
Beta distribution:
\begin{equation}
    q_\ell^{(\text{raw})} \sim \text{Beta}(a, b), \qquad
    q_\ell = \text{clip}\!\left(q_\ell^{(\text{raw})} 
    \cdot \frac{\bar{\tau}_{\text{target}}}
    {\boldsymbol{\alpha}^T \mathbf{q}^{(\text{raw})}}, \;
    0.001, \; 0.5\right)
    \label{eq:allele_freq_init}
\end{equation}
with $a = 2$, $b = 8$ (right-skewed). The scaling ensures $\E[\tau_i] \approx \bar{\tau}_{\text{target}}$. Clipping to $[0.001, 0.5]$ prevents fixation and ensures protective alleles remain below majority frequency (consistent with pre-epidemic rarity).

\subsection{Key properties for breeding}

\begin{proposition}[Trait independence]
\label{prop:independence}
Because the three trait blocks occupy non-overlapping loci, and 
allele frequencies are drawn independently per block, the trait 
values $(r_i, t_i, c_i)$ are statistically independent at the 
population level. The correlation between any two traits is 
$\rho \approx 0$ (confirmed empirically: $|\rho| < 0.005$ in 
simulations with $N = 100{,}000$).
\end{proposition}

\begin{proposition}[Maximum possible trait value]
\label{prop:max_trait}
An individual homozygous for the protective allele at all $n$ 
loci of a trait block achieves $\tau_i = \sum_\ell \alpha_\ell 
\cdot 1 = 1.0$. The probability of this occurring in a random 
individual is:
\begin{equation}
    \Prob(\tau_i = 1) = \prod_{\ell=1}^{n} q_\ell^2
    \label{eq:prob_max}
\end{equation}
which is astronomically small for typical $q_\ell$ values 
($\sim 0.1$--$0.3$).
\end{proposition}

\begin{proposition}[Additive variance]
\label{prop:additive_variance}
For a biallelic locus with frequency $q$ and effect $\alpha$, 
the additive genetic variance contribution is:
\begin{equation}
    V_{A,\ell} = 2 q_\ell (1 - q_\ell) \alpha_\ell^2
    \label{eq:va_locus}
\end{equation}
The total additive variance for a trait is:
\begin{equation}
    V_A = \sum_{\ell=1}^{n} 2 q_\ell (1 - q_\ell) \alpha_\ell^2
    \label{eq:va_total}
\end{equation}
This is exact under our purely additive model (no dominance 
or epistatic variance).
\end{proposition}
