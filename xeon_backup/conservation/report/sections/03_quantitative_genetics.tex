\section{Quantitative Genetics of Selection Response}
\label{sec:quant_gen}

\subsection{The breeder's equation}

The breeder's equation \citep{lush1937animal, falconer1996introduction}:
\begin{equation}
    R = h^2 \cdot S
    \label{eq:breeders}
\end{equation}
where $R = \bar{\tau}_{t+1} - \bar{\tau}_t$ is the selection response, $h^2 = V_A / V_P$ is narrow-sense heritability, and $S = \bar{\tau}_{\text{selected}} - \bar{\tau}_t$ is the selection differential.

\subsection{Heritability in our model}

In our model, traits are purely genetically determined:
\begin{equation}
    V_P = V_A + \underbrace{V_D}_{= 0} + \underbrace{V_E}_{= 0} = V_A
    \label{eq:heritability_model}
\end{equation}
so $h^2 = 1.0$ exactly.

\begin{remark}
In real \pyc{}, $h^2 < 1$ due to environmental variance ($V_E > 0$). For \emph{relative} comparisons between strategies, $h^2 = 1$ simplifies without affecting rankings. For absolute generation counts, apply a correction factor $h^2_{\text{real}}$.
\end{remark}

\subsection{Selection differential from truncation selection}

Under truncation selection retaining the top fraction $p$:
\begin{equation}
    S = i \cdot \sigma_P
    \label{eq:selection_differential}
\end{equation}
where the selection intensity $i = \phi(z_p)/p$ with $z_p = \Phi^{-1}(1 - p)$:
\begin{equation}
    i = \frac{\phi(z_p)}{1 - \Phi(z_p)} = \frac{\phi(z_p)}{p}
    \label{eq:selection_intensity}
\end{equation}

\begin{center}
\begin{tabular}{ccc}
\toprule
Fraction selected ($p$) & Selection intensity ($i$) & Practical meaning \\
\midrule
0.50 & 0.80 & Keep top half \\
0.20 & 1.40 & Keep top fifth \\
0.10 & 1.76 & Keep top tenth \\
0.05 & 2.06 & Intensive selection \\
0.01 & 2.67 & Extreme selection \\
\bottomrule
\end{tabular}
\end{center}

\subsection{Predicted gain per generation}

With $h^2 = 1$: $R = i \cdot \sigma_A = i \cdot \sqrt{V_A}$, where for resistance ($n_R = 17$ loci):
\begin{equation}
    R = i \cdot \sigma_A = i \cdot \sqrt{V_A}
    \label{eq:gain_per_gen}
\end{equation}
\begin{equation}
    \sigma_A = \sqrt{\sum_{\ell=1}^{17} \frac{\alpha_\ell^2}{2}\, 
    q_\ell (1 - q_\ell)}
    \label{eq:sigma_a_resistance}
\end{equation}

\begin{remark}[Scaling factor]
The factor $\alpha_\ell^2 / 2$ (not $2\alpha_\ell^2$) arises because traits use mean-of-alleles encoding: $\tau = \sum_\ell \alpha_\ell (a_1 + a_2)/2$, so the substitution effect is $\alpha_\ell/2$. Then $V_A = 2 \sum (\alpha_\ell/2)^2 q(1-q) = \sum \alpha_\ell^2 q(1-q) / 2$ (confirmed by simulation, \Cref{sec:validation}). Note that $V_A$ decreases across generations as selection drives $q \to 1$, yielding diminishing returns.
\end{remark}

\subsection{Multi-generation prediction}

Per-locus allele frequency change under truncation selection \citep{barton2000multilocus}:
\begin{equation}
    q_\ell^{(g+1)} = q_\ell^{(g)} + \Delta q_\ell^{(g)}
    \label{eq:allele_freq_update}
\end{equation}
\begin{equation}
    \Delta q_\ell \approx i \cdot 
    \frac{(\alpha_\ell / 2) \cdot q_\ell(1-q_\ell)}{\sigma_P}
    \label{eq:delta_q}
\end{equation}
Consistency check: $\Delta\E[\tau] = \sum_\ell \alpha_\ell \Delta q_\ell = (i/\sigma_P) \sum (\alpha_\ell^2/2)\,q(1-q) = i \cdot \sigma_P = R$, recovering the breeder's equation. This system of coupled difference equations is iterated forward numerically.

\begin{definition}[Fixation]
Locus $\ell$ is \textbf{fixed} for the protective allele when 
$q_\ell = 1$ (all individuals homozygous derived). The trait 
reaches its maximum when all loci are fixed.
\end{definition}

\begin{proposition}[Time to fixation scales with initial frequency]
\label{prop:fixation_time}
Loci with initially high $q_\ell$ fix first. Loci with very low 
$q_\ell$ and small $\alpha_\ell$ take the longest — they contribute 
little to the trait and experience weak selection. The ``last loci'' 
to fix determine the total generations needed to reach $\tau \approx 1$.
\end{proposition}

\subsection{The selection--variance trade-off}
\label{subsec:sel_var_tradeoff}

Intensive selection (small $p$) reduces effective population size:
\begin{equation}
    N_e \approx \frac{4 N_{\text{selected}} \cdot N_{\text{total}}}
    {N_{\text{selected}} + N_{\text{total}}}
    \label{eq:ne_selection}
\end{equation}
For $N_{\text{selected}} \ll N_{\text{total}}$: $N_e \approx 4 N_{\text{selected}}$. Small $N_e$ accelerates drift, inbreeding ($\Delta F = 1/(2N_e)$), and loss of rare alleles. This creates the fundamental breeding tension: \textbf{stronger selection gives faster gain but erodes diversity needed for long-term adaptation} (\Cref{sec:inbreeding}).

\subsection{Response in non-normal distributions}

Our trait distributions are right-skewed (Beta-distributed allele frequencies, many loci with low $q$). For non-normal distributions, the selection differential must be computed directly from the full trait distribution:
\begin{equation}
    S = \E[\tau_i \mid \tau_i \geq \tau^*] - \E[\tau_i]
    \label{eq:s_nonnormal}
\end{equation}

\subsection{Multi-trait selection}
\label{subsec:multi_trait}

For multi-trait selection, the Smith--Hazel index \citep{smith1936discriminant, hazel1943genetic}:
\begin{equation}
    I_i = \mathbf{b}^T \boldsymbol{\tau}_i, \qquad
    \mathbf{b} = \mathbf{G}^{-1} \mathbf{w}
    \label{eq:selection_index}
\end{equation}
where $\mathbf{G}$ is the genetic variance--covariance matrix and $\mathbf{w}$ the fitness weights. Since traits are independent (\Cref{prop:independence}), $\mathbf{G}$ is diagonal:
\begin{equation}
    \mathbf{G} = \text{diag}\!\left(V_A^{(r)}, \; V_A^{(t)}, \; V_A^{(c)}\right)
    \label{eq:g_matrix}
\end{equation}
and the index simplifies to:
\begin{equation}
    I_i = \frac{w_r}{V_A^{(r)}} r_i + 
    \frac{w_t}{V_A^{(t)}} t_i + 
    \frac{w_c}{V_A^{(c)}} c_i
    \label{eq:index_simplified}
\end{equation}
Appropriate fitness weights are derived in \Cref{sec:breeding}.
