\section{Breeding Program Design}
\label{sec:breeding}

\subsection{Fitness weights for multi-trait selection}

The probability of surviving a single exposure event:
\begin{equation}
    w(r, t, c) = \underbrace{r}_{\text{avoid infection}} + 
    \underbrace{(1 - r)}_{\text{get infected}} \cdot 
    \underbrace{s(t, c)}_{\text{survive infection}}
    \label{eq:fitness_function}
\end{equation}
where $s(t, c)$ is the probability of recovering given infection:
\begin{equation}
    s(t, c) = p_{\text{rec,I}_1}(c) + 
    [1 - p_{\text{rec,I}_1}(c)] \cdot 
    p_{\text{rec,I}_2}(t, c)
    \label{eq:survival_given_infection}
\end{equation}

with I$_2$ recovery probability:
\begin{equation}
    p_{\text{rec,I}_2}(t, c) = 
    1 - (1 - \rho_{\text{rec}} \cdot c)^{D_{I_2}(t)}
    \label{eq:p_rec_i2}
\end{equation}
and mean I$_2$ duration:
\begin{equation}
    D_{I_2}(t) = \frac{1}{\mu_{I_2D} \cdot 
    \max(1 - t \cdot \tau_{\max}, \; 0.05)}
    \label{eq:i2_duration}
\end{equation}

\subsubsection{Marginal fitness effects}

Partial derivatives:
\begin{align}
    \frac{\partial w}{\partial r} &= 1 - s(t, c) 
    \label{eq:dw_dr} \\
    \frac{\partial w}{\partial t} &= (1 - r) \cdot 
    \frac{\partial s}{\partial t} 
    \label{eq:dw_dt} \\
    \frac{\partial w}{\partial c} &= (1 - r) \cdot 
    \frac{\partial s}{\partial c}
    \label{eq:dw_dc}
\end{align}

At population-mean values ($r = 0.15$, $t = 0.10$, $c = 0.02$, $s \approx 0.002$):
\begin{align}
    \frac{\partial w}{\partial r} &\approx 0.998 \\
    \frac{\partial w}{\partial t} &\approx 0.85 \cdot 
    (1 - 0.15) \cdot 0.001 \approx 0.001 \\
    \frac{\partial w}{\partial c} &\approx (1 - 0.15) \cdot 
    0.05 \cdot 1.9 \approx 0.08
\end{align}

\begin{remark}
Resistance is \textbf{$\sim$1000$\times$} more important than tolerance and \textbf{$\sim$12$\times$} more important than recovery at population means. This ordering persists across biologically plausible ranges. At high resistance ($r = 0.5$), tolerance and recovery gain marginal importance as infection events become rarer.
\end{remark}

\subsection{Crossing strategies}

\subsubsection{Strategy 1: Random mating}

Parents paired uniformly at random. Expected offspring mean:
\begin{equation}
    \E[\tau_{\text{offspring}}] = \bar{\tau}_{\text{parents}}
    \label{eq:random_mating_mean}
\end{equation}
(midparent value under additivity). Offspring variance:
\begin{equation}
    V_{\text{offspring}} = \frac{1}{2} V_{\text{within-parents}} + 
    V_{\text{segregation}}
    \label{eq:offspring_variance}
\end{equation}
where segregation variance is:
\begin{equation}
    V_{\text{seg}} = \sum_{\ell} \frac{\alpha_\ell^2}{16} 
    \cdot h_\ell^{(p_1)} \cdot h_\ell^{(p_2)}
    \label{eq:segregation_var}
\end{equation}
with $h_\ell^{(p)} = \mathbf{1}[\text{parent } p \text{ heterozygous at } \ell]$. The $\alpha_\ell^2/16$ factor: substitution effect $\alpha_\ell/2$, segregation contributes $(\alpha_\ell/2)^2 \times 1/4$ per heterozygous locus (confirmed in \Cref{sec:validation}).

\subsubsection{Strategy 2: Assortative mating}

Rank parents by $r$, pair 1st with 2nd, etc. Maximizes offspring mean resistance but \emph{not} the maximum — parents homozygous-derived at the same loci produce identical offspring at those loci.

\subsubsection{Strategy 3: Complementary mating}
\label{subsec:complementary}

Pair parents covering different loci. Expected offspring resistance:
\begin{equation}
    \E[r_{\text{offspring}}(i,j)] = \sum_{\ell} \alpha_\ell 
    \cdot \bar{q}_\ell^{(i,j)}
    \label{eq:complementary_expected}
\end{equation}
where $\bar{q}_\ell^{(i,j)}$ is the expected frequency of the 
protective allele in offspring from parents $i$ and $j$ at 
locus $\ell$:
\begin{equation}
    \bar{q}_\ell^{(i,j)} = \frac{1}{2}\left(
    \frac{a_{i,\ell,1} + a_{i,\ell,2}}{2} + 
    \frac{a_{j,\ell,1} + a_{j,\ell,2}}{2}\right)
    = \frac{g_{i,\ell} + g_{j,\ell}}{4}
    \label{eq:offspring_freq}
\end{equation}
where $g_{i,\ell} = a_{i,\ell,1} + a_{i,\ell,2} \in \{0, 1, 2\}$ 
is the count of protective alleles.

Key insight: if parents carry $g = 2$ at \emph{different} loci $\ell$ and $\ell'$, offspring inherit $\alpha_\ell/2 + \alpha_{\ell'}/2$ — more than either parent's single-locus contribution.

\begin{proposition}[Complementary $>$ assortative for max offspring]
Under additive genetics with multiple loci, complementary mating produces higher maximum offspring trait values than assortative mating by combining alleles from different loci rather than duplicating the same ones.
\end{proposition}

\subsubsection{Strategy 4: Optimal contribution selection}

Balances gain and diversity \citep{meuwissen1997maximizing, woolliams2015genetic}:
\begin{equation}
    \text{max}_{\mathbf{c}} \; \mathbf{c}^T \boldsymbol{\tau}
    \quad \text{subject to} \quad 
    \mathbf{c}^T \mathbf{A} \mathbf{c} \leq 
    \frac{1}{2N_e^*}
    \label{eq:ocs}
\end{equation}
where $\mathbf{c}$ is parental contributions, $\boldsymbol{\tau}$ the trait vector, $\mathbf{A}$ the additive relationship matrix (computable from genotypes: $A_{ij} = (2/L) \sum_\ell \sum_k a_{i,\ell,k} a_{j,\ell,k}$), and $N_e^*$ the target effective size constraining $\Delta F \leq 1/(2N_e^*)$.

\subsection{Expected generations to resistance targets}

Generations to reach target $\tau^*$:
\begin{equation}
    G(\tau^*) = \min\left\{g : \bar{\tau}_0 + 
    \sum_{k=0}^{g-1} R_k \geq \tau^*\right\}
    \label{eq:gens_to_target}
\end{equation}
computed iteratively since $R_g$ depends on $V_A^{(g)}$.

\subsection{Family structure and within-family selection}
\label{subsec:family_selection}

High fecundity enables \textbf{within-family selection}: from cross $i \times j$, select the best offspring. Within-family variance is the segregation variance (\Cref{eq:segregation_var}), maximized when parents are heterozygous at many loci.

\begin{proposition}[Within-family gain]
The expected best-of-$m$ offspring from a cross has resistance:
\begin{equation}
    \E[r_{(m)}^{(i \times j)}] = 
    \E[r_{\text{offspring}}^{(i \times j)}] + 
    \sigma_{\text{seg}}^{(i \times j)} \cdot 
    \E[Z_{(m)}]
    \label{eq:within_family_best}
\end{equation}
where $\E[Z_{(m)}]$ is the expected maximum of $m$ standard 
normal draws, approximately $\sqrt{2 \ln m}$ for large $m$.
\end{proposition}

With $m = 100$: $\E[Z_{(100)}] \approx 2.51$; with $m = 1000$: $\E[Z_{(1000)}] \approx 3.09$. This leverages fecundity for extra gain without reducing family count (and thus without increasing inbreeding).
