\section{Spawning Timing Parameters}

The spawning timing parameters control the seasonal reproductive dynamics in the SSWD-EvoEpi model, determining when and how frequently individuals initiate spawning during the extended breeding season. These parameters are critical for population recruitment success, Allee effect dynamics, and disease transmission patterns through post-spawning immunosuppression.

\subsection{p\_spontaneous\_female: Daily Spontaneous Spawning Probability (Females)}

\paragraph{First Principles}
The daily probability for a reproductively ready female to initiate spawning spontaneously. At the current value of 0.012 d$^{-1}$, the expected wait time is approximately 83 days. Over a 270-day spawning season with Gaussian seasonal modulation, this ensures most females participate in spawning while maintaining temporal clustering essential for fertilization success in broadcast spawners. If too low, many females never spawn; if too high, spawning becomes completely asynchronous, reducing fertilization rates.

\paragraph{Literature Evidence}
No species-specific quantitative data exist for \textit{Pycnopodia helianthoides} daily spawning probabilities. However, several lines of evidence inform this parameter:

\textbf{Seasonal Pattern:} Animal Diversity Web reports that \textit{P. helianthoides} breeds via broadcast fertilization ``between March and July'' with the main peak in ``May and June,'' consistent with our model's seasonal timing.

\textbf{Spawning Synchrony:} Research on the crown-of-thorns starfish (\textit{Acanthaster planci}) emphasizes that spawning synchrony is ``fundamental for achieving high rates of fertilization'' in broadcast spawners \citep{Crown-of-thorns-PMC5371309}. This supports moderate spontaneous probabilities that maintain temporal clustering.

\textbf{Allee Effects:} Lundquist \& Botsford (2004) demonstrated that broadcast spawners experience fertilization success decline at low population densities, reinforcing the importance of spawning synchrony and appropriate spontaneous rates \citep{lundquist-botsford-2004-allee-broadcast-spawner}.

\paragraph{Recommendation}
\textbf{Confidence: $\star\star\bigcirc\bigcirc\bigcirc$ (moderate uncertainty)}

The current value of 0.012 d$^{-1}$ appears reasonable based on first principles and the need to balance participation with synchrony. However, analysis of captive breeding observations at Friday Harbor Laboratories may provide more precise species-specific estimates.

\subsection{p\_spontaneous\_male: Daily Spontaneous Spawning Probability (Males)}

\paragraph{First Principles}
Males can spawn multiple times per season (2--3 bouts) unlike females, so their base rate should be similar to or slightly higher than females to ensure adequate sperm availability throughout the breeding season. The current value of 0.0125 d$^{-1}$ reflects this capacity for multiple spawning events.

\paragraph{Literature Evidence}
\textit{Pycnopodia helianthoides} shows no sexual dimorphism \citep{animal-diversity-web}, and both sexes participate simultaneously in broadcast spawning. However, energetic costs differ dramatically---sperm production is metabolically inexpensive compared to egg mass development. The literature provides no specific data on male spawning frequency in \textit{Pycnopodia}, but the potential for repeated spawning is supported by low energetic costs relative to females.

\paragraph{Recommendation}
\textbf{Confidence: $\star\star\bigcirc\bigcirc\bigcirc$ (moderate uncertainty)}

The current value slightly exceeds the female rate (0.0125 vs 0.012 d$^{-1}$), reflecting the potential for multiple male spawning events. This parameter requires field validation through captive breeding programs.

\subsection{peak\_width\_days: Seasonal Peak Standard Deviation}

\paragraph{First Principles}
Standard deviation of the Gaussian seasonal readiness curve controlling the temporal spread of spawning activity. At 60 days, 95\% of spawning occurs within approximately a 4-month window. Narrower peaks ($\sigma < 30$ days) increase fertilization success but raise extinction risk from mistimed environmental cues; wider peaks ($\sigma > 90$ days) provide bet-hedging against environmental variability but reduce fertilization efficiency.

\paragraph{Literature Evidence}
\textbf{Observed Season:} Animal Diversity Web reports \textit{P. helianthoides} spawning ``between March and July'' (5 months total) with ``main peak in May and June'' (2-month peak window). This pattern strongly supports a peak width of approximately 60 days.

\textbf{Comparative Context:} The Antarctic sea star \textit{Odontaster validus} reproduces ``once a year during the winter season, between the months of April and June, with peak spawning occurring during June'' \citep{animal-diversity-web}, suggesting 2--3 month concentrated breeding windows are typical for cold-water asteroids.

\textbf{Phylogenetic Constraint:} Schiebelhut et al. (2022) found phylogenetic signals in asteroid reproductive seasons, indicating evolutionary constraints on spawning timing that support species-specific optimization \citep{schiebelhut-2022-traits-sswd-susceptibility}.

\paragraph{Recommendation}
\textbf{Confidence: $\star\star\star\bigcirc\bigcirc$ (moderate-high confidence)}

The current value of 60 days is well-supported by the March--July season with May--June peak reported in the literature. This represents a biologically realistic 4-month effective breeding season with a 2-month peak window.

\subsection{female\_max\_bouts: Maximum Female Spawning Bouts}

\paragraph{First Principles}
Each spawning event represents substantial energetic investment, with gonad development typically consuming 10--30\% of body mass in asteroids. Most species spawn once per season due to these energetic constraints, but \textit{Pycnopodia} as the largest known sea star (up to 5 kg) may have capacity for multiple smaller releases.

\paragraph{Literature Evidence}
\textbf{Energetic Constraints:} Research on \textit{Astropecten} species notes that ``resources stored in pyloric cecum seem to play an important role in the seasonal production of gonads,'' indicating tight energetic trade-offs in asteroid reproduction \citep{astropecten-helgoland-2016}.

\textbf{Size Advantage:} As the heaviest known sea star (approximately 5 kg, 80 cm diameter), \textit{Pycnopodia} may have greater energetic reserves for multiple spawning events compared to smaller asteroid species that typically spawn once annually.

\textbf{Data Gap:} No direct observations exist for \textit{Pycnopodia} spawning frequency. Most asteroid literature assumes single annual spawning, potentially reflecting study limitations or focus on smaller species.

\paragraph{Recommendation}
\textbf{Confidence: $\star\bigcirc\bigcirc\bigcirc\bigcirc$ (low confidence)}

Conservative estimate of 1--2 bouts per season based on energetic constraints, with recognition that exceptionally large individuals might support multiple smaller gamete releases. This parameter is a high priority for empirical validation through captive breeding programs and field observations.

\subsection{Research Priorities and Model Implications}

The spawning timing parameters represent a critical knowledge gap requiring targeted research efforts:

\begin{enumerate}
\item \textbf{Captive breeding data analysis:} Friday Harbor Laboratory observations \citep{hodin-2021} may contain quantitative spawning frequency and timing data requiring systematic analysis.

\item \textbf{Field validation:} Direct observation during the March--July breeding season could provide empirical constraints on spawning frequencies and environmental triggers.

\item \textbf{Energetic modeling:} Analysis of gonad development cycles relative to body size and nutritional status could constrain maximum spawning bout frequencies.
\end{enumerate}

These parameters directly influence disease transmission dynamics through post-spawning immunosuppression windows, Allee effect thresholds in low-density populations, and evolutionary selection on reproductive strategies. Accurate parameterization is essential for reliable conservation planning and captive breeding program design.