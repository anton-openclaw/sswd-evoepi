% abstract.tex — Publication paper abstract (~280 words)
% Written to accurately reflect content of all paper sections.

Sea star wasting disease (SSWD) caused a $>$90\% decline in the sunflower
sea star (\textit{Pycnopodia helianthoides}), a keystone predator in
northeastern Pacific kelp forest ecosystems, triggering cascading trophic
effects including sea urchin population explosions and extensive kelp
deforestation. Despite ongoing captive breeding and experimental
outplanting, no predictive framework integrates disease dynamics,
host genetics, and spatial structure to evaluate reintroduction
outcomes under continued disease pressure. Here we present
\modelname{}, the first coupled eco-evolutionary epidemiological
individual-based model for \textit{P.~helianthoides} and SSWD,
parameterized following confirmation of \textit{Vibrio pectenicida}
as a causative agent. The model tracks individual sea stars carrying
diploid genotypes across 51 loci encoding three heritable defense
traits---resistance (immune exclusion), tolerance (damage limitation),
and recovery (pathogen clearance)---within an 11-node stepping-stone
metapopulation spanning the species' range from Alaska to California.
Disease dynamics follow a stochastic SEIPD compartmental framework
with temperature-dependent progression rates calibrated to
experimental infection data, coupled with an environmental pathogen
reservoir driven by satellite-derived sea surface temperatures.
Comprehensive sensitivity analysis of all 47 model parameters using
Morris screening (960 runs) and Sobol variance decomposition
(25,088 runs) reveals that the base recovery rate dominates model
behavior yet has zero empirical basis. Validation simulations produce
$>$99\% population crashes across all configurations, and the
biologically correct reinfection dynamics (recovered individuals
return to the susceptible pool, reflecting the absence of adaptive
immunity in echinoderms) eliminate recovery trait evolution entirely,
shifting selection decisively toward resistance. Scaling population
size 20-fold does not improve outcomes. These results indicate that
natural evolutionary rescue is not viable on conservation-relevant
timescales, that captive breeding programs should prioritize genetic
screening for resistance alleles, and that determining whether
\textit{P.~helianthoides} can clear \textit{V.~pectenicida}
infections is the single highest-priority empirical question for
predicting reintroduction success.
