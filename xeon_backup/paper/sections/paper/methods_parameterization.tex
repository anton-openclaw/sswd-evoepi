% ══════════════════════════════════════════════════════════════════════
% METHODS — PARAMETERIZATION
% Fact-checked against: specs/parameter_justification/
%   MASTER_PARAMETER_TABLE.md, GAPS_AND_UNCERTAINTIES.md
% All parameter counts and tier assignments verified 2026-02-22
% ══════════════════════════════════════════════════════════════════════

\label{sec:parameterization}

The model requires 47~parameters spanning 11~functional groups: disease
progression~(3), pathogen shedding and dose--response~(5), environmental
pathogen dynamics~(4), recovery and immunity~(4), growth and life
history~(4), fecundity and recruitment~(3), genetic architecture~(8),
spawning timing~(4), spawning induction~(3), larval dispersal~(3), and
pathogen evolution~(6). We adopted a three-tier grounding strategy that
classifies each parameter by the strength of its empirical basis
(Table~\ref{tab:param_summary}; full justification report in
Supplementary Material~\ref{sec:supp_params}).

\paragraph{Tier~1: Literature-constrained ($\sim$12~parameters).}
These parameters are directly informed by species-specific or
closely related empirical data and were fixed at their published values
or narrow ranges. Disease progression rates were calibrated to
controlled infection experiments with \vp{} in asteroids
\citep{prentice2025kochs}, yielding a total disease course of
$\sim$11.6\,d at 13\textdegree C. Von~Bertalanffy growth parameters
draw on echinoderm life-history data \citep{hodin2021roadmap}, and the
Arrhenius reference temperature $T_{\mathrm{ref}} = 20$\textdegree C
reflects the thermal optimum of \vp{}
\citep{eisenlord2016ochre, bates2009sea}. The genetic
architecture---51 biallelic loci with exponentially distributed effect
sizes---is grounded in the GWAS of \citet{schiebelhut2018collapse}, who
identified $\sim$51 loci with significant allele frequency shifts in
SSWD survivors of a related asteroid species. SST forcing was derived
from NOAA OISST v2.1 satellite climatologies rather than fitted
(Section~\ref{sec:spatial}).

\paragraph{Tier~2: Informed priors ($\sim$22~parameters).}
These parameters lack direct \pycshort{}-specific measurements but are
constrained by data from related taxa, theoretical bounds, or
comparative scaling relationships. Examples include larval dispersal
scale ($D_L = 400$\,km, derived from \pycshort{} PLD estimates and
NE~Pacific current speeds), self-recruitment fractions (based on
estuarine retention physics), pathogen shedding ratios (informed by
marine \textit{Vibrio} literature and decomposition ecology), and
spawning phenology (constrained to the March--July season documented
for \pycshort{}). These parameters were assigned informative prior
distributions for sensitivity analysis and calibration.

\paragraph{Tier~3: Calibration targets ($\sim$13~parameters).}
These parameters are identifiable only through model fitting against
emergent population-level patterns. They include the recovery rate
scaling factor~$\rho_{\mathrm{rec}}$, maximum environmental pathogen
input~$P_{\mathrm{env,max}}$, settler survival probability~$s_0$,
and several pathogen evolution parameters (virulence--transmission
trade-off exponents). These are the primary targets for ABC-SMC
calibration (Section~\ref{sec:calibration}).

\paragraph{Key uncertainties.}
We acknowledge substantial uncertainty in several parameter domains.
Species-specific empirical data for \pycshort{} remain extremely
limited: no direct measurements exist for pathogen shedding rates,
virulence--transmission trade-offs, or individual growth trajectories.
The recovery rate is particularly uncertain---field observations
indicate $>$99\% mortality, but the genetic basis for resistance
\citep{pespeni2023insight} implies that rare recovery events are
biologically plausible. Larval dispersal scale depends on PLD estimates
that span a wide range (14--70\,d), and the effective dispersal distance
is typically 10--30\% of maximum transport distance due to eddies
and retention. We address these uncertainties through comprehensive
global sensitivity analysis (Section~\ref{sec:sensitivity}), which
identified the 10~most influential parameters and confirmed that all
47~parameters exhibit nonlinear effects ($\sigma / \mu^* > 1.0$),
precluding any simplification by parameter elimination. A complete parameter table with default values, sensitivity analysis
ranges, confidence ratings, and source literature for all 47~parameters
is provided in the Supplementary Material
(Table~\ref{tab:full_params}). The detailed parameter justification
report accompanying each functional group---including first-principles
reasoning, direct literature review (103~sources), and quantitative
interaction chain analysis---is available as Supplementary
Document~S1.
