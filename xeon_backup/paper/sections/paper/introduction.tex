% introduction.tex — Publication paper introduction

% Paragraph 1 — THE CRISIS
Sea star wasting disease (SSWD) caused one of the largest documented wildlife mass mortality events in marine ecosystems when it swept through populations of over 20 asteroid species along the northeastern Pacific coast beginning in 2013 \citep{Hewson2014,Montecino-Latorre2016,Harvell2019}. Characterized by arm twisting, loss of turgor, body wall lesions, ray autotomy, and rapid tissue degradation, the disease devastated populations from Baja California to the Gulf of Alaska within months \citep{Miner2018,Hewson2019}. Among the affected species, the sunflower sea star (\pyc{}) suffered the most catastrophic decline, losing an estimated 5.75 billion individuals---a 90.6\% range-wide population reduction---with declines exceeding 97\% along the outer coast from Washington to Baja California \citep{Gravem2021,Hamilton2021}. The species was assessed as Critically Endangered by the IUCN \citep{Gravem2021} and is under consideration for listing as Threatened under the U.S.\ Endangered Species Act \citep{Lowry2022}. As a large-bodied, mobile, generalist predator that consumes sea urchins at rates sufficient to structure entire subtidal communities, \pyc{} functions as a keystone species in northeastern Pacific kelp forest ecosystems \citep{Galloway2023,Burt2018}. Its precipitous decline has triggered cascading trophic effects, including sea urchin population explosions and extensive kelp forest deforestation---northern California lost 96\% of its kelp canopy following the 2014 marine heatwave \citep{Rogers-Bennett2019,Meunier2024}. The collapse of \pyc{} thus represents not only a single-species conservation crisis but a destabilization of an entire marine ecosystem \citep{Hamilton2021,Langendorf2025}.

% Paragraph 2 — THE PATHOGEN
For over a decade following the initial outbreak, the causative agent of SSWD remained contested. An early hypothesis implicating sea star associated densovirus \citep{Hewson2014} was subsequently undermined by failures to reproduce challenge experiments and the discovery that the virus is endemic in healthy echinoderm populations worldwide \citep{Hewson2018,Hewson2019,Hewson2025review}. The breakthrough came with \citet{Prentice2025}, who fulfilled Koch's postulates by demonstrating that \vp{} strain FHCF-3, a Gram-negative marine bacterium, is a causative agent of SSWD in \pyc{}. Through seven controlled exposure experiments using captive-bred, quarantined sea stars, the authors showed that injection of cultured \vpshort{} reliably produced disease signs and death within approximately two weeks, while heat-treated and filtered controls remained healthy. Critically, the pathogen was re-isolated from experimentally infected animals, completing the postulates. The identification of a specific bacterial pathogen with known temperature-dependent growth dynamics \citep{Lupo2020} provides a mechanistic basis for modeling disease transmission and environmental forcing. However, the etiological picture is not entirely resolved: \citet{Hewson2025autecology} demonstrated that \vpshort{} was not consistently detected in non-\pyc{} species during the 2013--2014 mass mortality, suggesting the pathogen may be specific to \pyc{} or may function opportunistically under different conditions. For \pycshort{}---the focus of this study---the evidence for \vpshort{} as the primary causative agent is robust.

% Paragraph 3 — CONSERVATION RESPONSE
The failure of \pyc{} populations to recover naturally in the decade following the initial epizootic---contrasting with partial recovery observed in some co-occurring asteroid species \citep{Gravem2025}---has motivated intensive conservation action. The species' long generation time ($\sim$30 years), broadcast spawning reproductive strategy, and vulnerability to Allee effects at low density \citep{Lundquist2004,Gascoigne2004} compound the challenge of natural recovery. Historical precedent is sobering: the Caribbean long-spined sea urchin \textit{Diadema antillarum}, which suffered a comparable 93--100\% mass mortality in 1983--1984, achieved only $\sim$12\% recovery after three decades \citep{Lessios2016}. In response, a coordinated multi-partner recovery effort has emerged. The Association of Zoos and Aquariums SAFE program maintains over 2,500 captive juveniles across 17 AZA institutions \citep{AZA2024}, and experimental outplanting has progressed through caged trials (2023), uncaged release (2024; \citealt{kuow2024seastar}), and the first California outplanting in December 2025, where 47 of 48 captive-bred juveniles survived after four weeks \citep{ssl2025outplanting}. These efforts raise urgent quantitative questions: How many captive-bred individuals should be released, where, and when? Can natural selection drive resistance evolution fast enough to matter on conservation timescales? How do ongoing disease, environmental change, and spatial structure interact to shape recovery trajectories?

% Paragraph 4 — WHY MODELING
Answering these questions demands a modeling framework that integrates disease dynamics with population genetics in an explicitly spatial context---yet existing models of SSWD address these components in isolation. \citet{Aalto2020} coupled an SIR-type model with ocean circulation to explain the rapid spread of SSWD but did not consider host evolution. \citet{Tolimieri2022} conducted a population viability analysis using stage-structured matrix models but omitted disease dynamics and genetics. \citet{Arroyo-Esquivel2025} recently modeled epidemiological consequences of managed reintroduction following disease-driven decline, but their framework lacks genetic evolution. None of these approaches captures the interplay between disease-driven selection, host genetic adaptation, and demographic recovery that is central to predicting conservation outcomes. Individual-based models (IBMs) are uniquely suited to this challenge because they can represent the stochasticity, genetic drift, and spatial heterogeneity that govern eco-evolutionary dynamics in depleted populations \citep{deangelis2005individual,grimm2005individual}. The closest methodological precedent is the eco-evolutionary IBM developed by \citet{Clement2024} for coevolution between Tasmanian devils (\textit{Sarcophilus harrisii}) and devil facial tumour disease (DFTD), which coupled an epidemiological framework with polygenic quantitative genetics and found a high probability of host persistence through coevolutionary dynamics. Our model extends this approach to a marine broadcast spawner---a system with fundamentally different reproductive biology, including sweepstakes reproductive success \citep{Hedgecock2011}, external fertilization subject to Allee effects, and a pelagic larval phase mediating spatial connectivity.

% Paragraph 5 — WHAT THIS PAPER DOES
We present \modelname{}, the first coupled eco-evolutionary epidemiological IBM for \pyc{} and SSWD. The model tracks individual sea stars as agents within a network of habitat nodes connected by larval dispersal and pathogen transport. Each agent carries a diploid genotype across 51 loci---informed by genome-wide association studies identifying loci with allele frequency shifts following the epizootic \citep{Schiebelhut2018,Schiebelhut2024genome}---governing three fitness-related traits: resistance (immune exclusion reducing infection probability), tolerance (damage limitation extending survival during late-stage infection), and recovery (pathogen clearance enabling transition from infected to susceptible states). Disease dynamics follow an SEIR-type compartmental structure with temperature-dependent progression rates calibrated to the experimental disease time course of \citet{Prentice2025}, coupled with an environmental pathogen reservoir driven by satellite-derived sea surface temperatures. We deploy the model on an 11-node stepping-stone metapopulation spanning the species' range from Sitka, Alaska to Monterey, California, and conduct comprehensive sensitivity analysis across 47 parameters using Morris screening and Sobol variance decomposition to identify the key drivers of epidemiological, demographic, and evolutionary outcomes. Validation against empirical patterns reveals $>$99\% population crashes, with evolutionary rescue via recovery not viable under realistic biology---resistance emerges as the primary trait under selection. These findings have direct implications for reintroduction strategy, suggesting that genetic management of resistance alleles, rather than reliance on natural recovery evolution, should be a priority for captive breeding programs.
