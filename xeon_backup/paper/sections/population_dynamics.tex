% ══════════════════════════════════════════════════════════════════════
% POPULATION DYNAMICS
% Fact-checked against: sswd_evoepi/model.py (daily_natural_mortality,
%   daily_growth_and_aging, von_bertalanffy, grow_individual,
%   annual_reproduction, settle_daily_cohorts),
%   sswd_evoepi/reproduction.py (fecundity, fertilization_success,
%   srs_reproductive_lottery, beverton_holt_recruitment,
%   pelagic_larval_duration, larval_survival, settlement_cue_modifier),
%   sswd_evoepi/spawning.py (spawning_step, cascade_induction,
%   immunosuppression)
% ══════════════════════════════════════════════════════════════════════

\section{Population Dynamics}
\label{sec:population_dynamics}

The population dynamics module governs the complete life history of
\pyc{}: growth, natural mortality, reproduction, larval dispersal,
and settlement. All demographic processes operate on a daily timestep,
integrated within the master simulation loop described in
Section~\ref{sec:model_architecture}. Disease-driven mortality is
handled by the disease module (Section~\ref{sec:disease_module});
coupling occurs through shared access to the agent array.


% ──────────────────────────────────────────────────────────────────────
\subsection{Life Stages}
\label{sec:stages}

Each individual progresses through five life stages defined by size
thresholds (Table~\ref{tab:stages_pop}). Stage transitions are
unidirectional: agents can only advance, never regress.

\begin{table}[H]
\centering
\caption{Life stages and transition thresholds for \pycshort{}.}
\label{tab:stages_pop}
\small
\begin{tabular}{llcl}
\toprule
\textbf{Stage} & \textbf{Size range} & \textbf{Transition at} & \textbf{Duration} \\
\midrule
Egg/Larva   & Planktonic      & Settlement event & 49--146 days PLD \\
Settler     & 0.5--10~mm      & $\geq 10$~mm     & $\sim$1 year \\
Juvenile    & 10--150~mm      & $\geq 150$~mm    & $\sim$1--5 years \\
Subadult    & 150--400~mm     & $\geq 400$~mm    & $\sim$5--10 years \\
Adult       & $>$400~mm       & ---               & 10--50+ years \\
\bottomrule
\end{tabular}
\end{table}


% ──────────────────────────────────────────────────────────────────────
\subsection{Growth}
\label{sec:growth}

Individual growth follows the von Bertalanffy (VB) growth model in
differential form, resolved daily:

\begin{equation}
  \label{eq:vb_growth}
  L(t + \Delta t) = L_\infty - (L_\infty - L(t)) \cdot
    \exp(-k_\text{growth} \cdot \Delta t)
\end{equation}

\noindent
where $L_\infty = 1000$~mm is the asymptotic arm-tip diameter,
$k_\text{growth} = 0.08$~yr$^{-1}$ is the Brody growth coefficient,
and $\Delta t = 1/365$~yr for the daily timestep. Individual growth
variation is introduced through a multiplicative log-normal noise
term applied to the daily increment:

\begin{equation}
  \Delta L_i = (L_\text{det} - L_i) \cdot
    \exp\!\bigl(\varepsilon_i\bigr), \quad
  \varepsilon_i \sim \mathcal{N}\!\left(0,\;
    \frac{\sigma_g}{\sqrt{365}}\right)
\end{equation}

\noindent
where $\sigma_g = 2.0$~mm is the annual growth noise scale and the
$\sqrt{365}$ scaling preserves the annual CV when integrated over
daily steps. Size is constrained to never decrease (no shrinking).
Stage transitions are evaluated after each growth step based on the
thresholds in Table~\ref{tab:stages_pop}.

Aging proceeds at $1/365$ years per day, producing fractional ages
that drive size-at-age trajectories and determine eligibility for
senescence mortality.


% ──────────────────────────────────────────────────────────────────────
\subsection{Natural Mortality}
\label{sec:mortality}

Natural mortality is resolved daily using continuous hazard rates
derived from stage-specific annual survival probabilities.
The daily death probability for individual $i$ is:

\begin{equation}
  \label{eq:daily_mort}
  p_{\text{death},i} = 1 -
    \bigl(1 - m_\text{annual}(s_i)\bigr)^{1/365}
\end{equation}

\noindent
where $m_\text{annual}(s) = 1 - S_\text{annual}(s)$ is the annual
mortality rate for stage $s$. The annual survival schedule
(Table~\ref{tab:survival}) produces a type III survivorship curve
with high settler/juvenile mortality and low adult mortality,
consistent with demographic estimates for long-lived asteroids.

\begin{table}[H]
\centering
\caption{Stage-specific annual survival rates.}
\label{tab:survival}
\small
\begin{tabular}{lcc}
\toprule
\textbf{Stage} & \textbf{Annual survival ($S$)} & \textbf{Annual mortality} \\
\midrule
Settler  & 0.001 & 0.999 \\
Juvenile & 0.03  & 0.97 \\
Subadult & 0.90  & 0.10 \\
Adult    & 0.95  & 0.05 \\
Senescent & 0.98 & 0.02 (base) \\
\bottomrule
\end{tabular}
\end{table}

\paragraph{Senescence.}
Individuals exceeding the senescence age ($a_\text{sen} = 50$~yr)
accumulate additional mortality linearly:

\begin{equation}
  m_\text{total}(s_i, a_i) = m_\text{annual}(s_i) +
    m_\text{sen} \cdot \frac{a_i - a_\text{sen}}{20}
\end{equation}

\noindent
where $m_\text{sen} = 0.10$ and the divisor of 20 scales the
senescence ramp such that a 70-year-old individual experiences an
additional 10\% annual mortality.

Daily mortality is applied via a single vectorized random draw across
all alive agents, converting stage-dependent annual rates to daily
hazard probabilities. This continuous approach avoids the artificial
synchronization artifacts of annual batch mortality and permits
realistic within-year population fluctuations.


% ──────────────────────────────────────────────────────────────────────
\subsection{Spawning System}
\label{sec:spawning}

\modelname{} implements a biologically detailed spawning system
reflecting the extended reproductive season and cascading spawning
behavior observed in \pycshort{}.

\subsubsection{Spawning Season and Phenology}
\label{sec:spawning_season}

The spawning season extends from day~305 ($\approx$November~1) through
day~196 ($\approx$July~15) of the following year, spanning
approximately 270 days and wrapping across the calendar year boundary.
Spawning intensity follows a Normal envelope centered on a
latitude-adjusted peak:

\begin{equation}
  \label{eq:spawning_envelope}
  P_\text{season}(d) = \exp\!\left(
    -\frac{(\Delta d)^2}{2\,\sigma_\text{peak}^2}
  \right)
\end{equation}

\noindent
where $\Delta d$ is the shortest circular distance between day $d$
and the peak day (accounting for year wrapping), and
$\sigma_\text{peak} = 60$~days is the standard deviation of the
seasonal peak. The peak day-of-year is latitude-dependent:

\begin{equation}
  d_\text{peak}(\phi) = d_\text{peak,base} +
    \lceil(\phi - 40^\circ\text{N}) \times 3\;\text{d/}^\circ\rceil
\end{equation}

\noindent
where $d_\text{peak,base} = 105$ ($\approx$April~15) is the reference
peak at $40^\circ$N, and higher-latitude populations spawn
approximately 3 days later per degree northward.


\subsubsection{Spontaneous Spawning}
\label{sec:spontaneous_spawning}

Each day during the spawning season, mature adults ($\geq$400~mm,
Susceptible or Recovered disease state) are first evaluated for
\emph{readiness}, a stochastic physiological state modulated by
the seasonal envelope $P_\text{season}(d)$. Once ready, individuals
attempt spontaneous spawning with sex-specific daily probabilities:

\begin{align}
  p_{\text{spawn},\text{female}} &= 0.012 \\
  p_{\text{spawn},\text{male}}   &= 0.0125
\end{align}

\noindent
These rates were calibrated to produce $\geq$80\% female spawning
participation per season and a mean of $\sim$2.2 male bouts per
season, consistent with the observation that males spawn more
frequently than females in broadcast-spawning asteroids.

\paragraph{Bout limits and refractory periods.}
Females are limited to a maximum of 2 spawning bouts per season;
males are limited to 3 bouts. Males enter a brief refractory period
between bouts (default 0 days, configurable) during which they
cannot spawn, reflecting the physiological recovery time for
spermatogenesis.


\subsubsection{Cascade Induction}
\label{sec:cascade}

Spawning by one individual can trigger spawning in nearby conspecifics
via waterborne chemical cues (spawning-induced spawning), producing
the synchronous mass spawning events observed in broadcast spawners.
Induction operates over a 3-day chemical cue persistence window and
is strongly sex-asymmetric:

\begin{align}
  \kappa_{\text{fm}} &= 0.80 \quad \text{(female $\to$ male induction)} \\
  \kappa_{\text{mf}} &= 0.60 \quad \text{(male $\to$ female induction)}
\end{align}

\noindent
where $\kappa_{\text{fm}}$ is the probability that a ready male spawns
when a female within the cascade radius (200~m) has spawned within
the cue window. The female-to-male asymmetry reflects the stronger
spawning trigger provided by egg-associated chemical signals.
Readiness induction also operates: individuals not yet
physiologically ready can be driven to readiness by nearby spawning
activity, with a daily probability of 0.5 when within a 300~m
detection radius.


\subsubsection{Post-Spawning Immunosuppression}
\label{sec:post_spawn_immunosuppression}

Spawning imposes a 28-day immunosuppression period during which the
individual's force of infection is multiplied by a susceptibility
factor of 2.0:

\begin{equation}
  \label{eq:post_spawn_immunosuppression}
  \lambda_i^\text{eff} = \lambda_i \times
    \begin{cases}
      \chi_\text{immuno} = 2.0 & \text{if immunosuppression timer} > 0 \\
      1.0 & \text{otherwise}
    \end{cases}
\end{equation}

\noindent
This reflects the metabolic cost of gamete production and the
documented increase in disease susceptibility following reproductive
investment in marine invertebrates. The immunosuppression timer is
reset each time an individual spawns, so multiple spawning bouts
within a season extend the vulnerability window. Immunosuppression
timers are decremented daily regardless of spawning season status.


% ──────────────────────────────────────────────────────────────────────
\subsection{Fecundity}
\label{sec:fecundity}

Female fecundity follows an allometric relationship with body size:

\begin{equation}
  \label{eq:fecundity}
  F_i = F_0 \cdot
    \left(\frac{L_i}{L_\text{ref}}\right)^b
\end{equation}

\noindent
where $F_0 = 10^7$ eggs is the reference fecundity at
$L_\text{ref} = 500$~mm and $b = 2.5$ is the allometric exponent.
Only females at or above the minimum reproductive size
$L_\text{min} = 400$~mm produce eggs. No cost-of-resistance penalty
is applied to fecundity (Section~\ref{sec:genetics_diagnostics}).


% ──────────────────────────────────────────────────────────────────────
\subsection{Fertilization Kinetics and the Allee Effect}
\label{sec:fertilization}

Broadcast spawners face a fertilization Allee effect: at low
population density, sperm limitation reduces the fraction of eggs
successfully fertilized \citep{Gascoigne2004, Lundquist2004}. We model
fertilization success using a mean-field approximation of the
\citet{Lundquist2004} broadcast fertilization model:

\begin{equation}
  \label{eq:fertilization}
  \mathcal{F}(\rho_m) = 1 -
    \exp(-\gamma_\text{fert} \cdot \rho_{m,\text{eff}})
\end{equation}

\noindent
where $\gamma_\text{fert} = 4.5$~m$^2$ is the sperm contact parameter
and $\rho_{m,\text{eff}}$ is the effective male density, potentially
enhanced by spawning aggregation behavior. The aggregation factor
increases effective local density within spawning clumps above the
spatially uniform average when adult count exceeds a threshold.

This produces a quadratic relationship between zygote production and
density at low density: $\text{zygotes} \propto \rho_f \times
\mathcal{F}(\rho_m) \propto \rho^2$ when $\rho \to 0$, creating a
strong demographic Allee effect. For high-fecundity broadcast spawners
like \pycshort{}, the deterministic Allee threshold is near zero
density; the practical Allee effect operates through stochastic
processes at low $N$.


% ──────────────────────────────────────────────────────────────────────
\subsection{Larval Phase}
\label{sec:larval}

Fertilized eggs enter a temperature-dependent pelagic larval duration
(PLD):

\begin{equation}
  \label{eq:pld}
  \text{PLD}(T) = \text{PLD}_\text{ref} \cdot
    \exp\!\bigl(-Q_\text{dev} \cdot (T - T_\text{ref})\bigr)
\end{equation}

\noindent
where $\text{PLD}_\text{ref} = 63$~days at $T_\text{ref} = 10.5$°C
\citep{Hodin2021}, and $Q_\text{dev} = 0.05$~°C$^{-1}$ produces
shorter PLD at warmer temperatures. PLD is clamped to $[30, 150]$
days.

Larval survival during the pelagic phase follows a constant daily
mortality model:

\begin{equation}
  \label{eq:larval_survival}
  S_\text{larval} = \exp(-\mu_\text{larva} \cdot \text{PLD})
\end{equation}

\noindent
with $\mu_\text{larva} = 0.05$~d$^{-1}$. At the reference PLD of
63 days, this yields $S_\text{larval} \approx 4.3\%$ --- high
mortality that is compensated by the enormous fecundity of
\pycshort{}.

Larval cohorts carry genotypes inherited via the SRS lottery
(Section~\ref{sec:srs}) and are tracked as discrete objects during
the pelagic phase. Upon completion of PLD, competent larvae are
available for settlement. In the spatial simulation
(Section~\ref{sec:spatial_module}), cohorts are dispersed between
nodes via the larval connectivity matrix $\mathbf{C}$ before
settlement.


% ──────────────────────────────────────────────────────────────────────
\subsection{Settlement and Recruitment}
\label{sec:settlement}

Competent larvae settle into the benthic population through a
three-stage process:

\paragraph{1. Settlement cue (Allee effect).}
Settlement success is modulated by the presence of conspecific adults
via a Michaelis--Menten function representing biofilm-mediated
settlement cues:

\begin{equation}
  \label{eq:settlement_cue}
  C_\text{settle}(N_\text{adults}) = 0.2 +
    \frac{0.8 \cdot N_\text{adults}}{5 + N_\text{adults}}
\end{equation}

\noindent
where the baseline of 0.2 represents settlement on coralline algae in
the absence of adults, and the additional 0.8 reflects enhanced
settlement induced by adult biofilm. The half-saturation constant of 5
adults means that even a small remnant population provides strong
settlement cues.

\paragraph{2. Density-dependent recruitment (Beverton--Holt).}
The number of recruits is governed by a standard Beverton--Holt
stock-recruitment relationship:

\begin{equation}
  \label{eq:bh}
  R = \frac{K \cdot s_0 \cdot S}{K + s_0 \cdot S}
\end{equation}

\noindent
where $S$ is the number of effective settlers (after cue modulation),
$K$ is the carrying capacity, and $s_0 = 0.03$ is the
density-independent per-settler survival rate. At low $S$,
$R \approx s_0 S$ (supply-limited); at high $S$, $R \to K$
(habitat-limited). For broadcast spawners with $S \gg K$, recruitment
is typically habitat-limited and population self-regulates.

\paragraph{3. Agent initialization.}
Recruited settlers are placed in dead agent slots, assigned size
0.5~mm, age 0, Settler stage, random sex (1:1 ratio), Susceptible
disease state, and random position within the node's habitat area.
Genotypes are copied from the SRS-selected settler genotypes, and all
three trait scores ($r_i$, $t_i$, $c_i$) are computed from the
inherited genotype.

\paragraph{Juvenile immunity.}
Newly settled individuals can optionally be granted a juvenile
immunity period (configurable, default 0 days) during which they
are not susceptible to infection. The settlement day is recorded
for each recruit to enable age-dependent susceptibility calculations.


% ──────────────────────────────────────────────────────────────────────
\subsection{Continuous Settlement}
\label{sec:continuous_settlement}

Rather than settling all larvae in an annual pulse, the model
tracks individual larval cohorts and settles them daily as their PLD
elapses. Cohorts generated by daily spawning events throughout the
extended spawning season (Section~\ref{sec:spawning_season}) are
stored in a per-node pending list sorted by settlement day. Each
simulation day, cohorts whose PLD has elapsed are popped from the
sorted list front (amortized $O(1)$) and passed through the
settlement pipeline. This continuous approach produces realistic
seasonal recruitment pulses that peak several months after the
spawning peak, consistent with the observed temporal offset between
spawning and juvenile recruitment in \pycshort{}.

At the annual boundary, any remaining unsettled cohorts from each
node are collected for spatial dispersal via the connectivity matrix
$\mathbf{C}$ (Section~\ref{sec:spatial_module}), then redistributed
to destination nodes where they continue to settle daily as PLD
elapses.


% ──────────────────────────────────────────────────────────────────────
\subsection{Demographic--Genetic--Epidemiological Coupling}
\label{sec:demo_coupling}

The population dynamics module is bidirectionally coupled to the
disease and genetics modules:

\begin{itemize}
  \item \textbf{Disease $\to$ demographics:} Disease kills
    individuals (I$_2 \to$ D), reducing population size and altering
    age/size structure. Post-spawning immunosuppression
    (Section~\ref{sec:post_spawn_immunosuppression}) increases disease risk for
    recent spawners, creating a temporal alignment between peak
    reproductive effort and peak epidemic severity during warm months.

  \item \textbf{Demographics $\to$ disease:} Reduced population
    density lowers contact rates and environmental pathogen
    concentration. The fertilization Allee effect
    (Section~\ref{sec:fertilization}) amplifies population collapse
    by reducing reproductive output at low density, potentially
    trapping populations in an extinction vortex.

  \item \textbf{Genetics $\to$ demographics:} The SRS lottery
    (Section~\ref{sec:srs}) produces extreme reproductive variance
    that amplifies genetic drift while accelerating the fixation of
    resistance, tolerance, and recovery alleles enriched by
    selection during epidemic episodes.

  \item \textbf{Demographics $\to$ genetics:} Population bottlenecks
    from disease reduce $N_e$ far below census $N$, compounded by
    SRS ($N_e/N \sim 10^{-3}$). The interaction of selection with
    small effective population size determines whether evolutionary
    rescue is fast enough to prevent extinction.
\end{itemize}
