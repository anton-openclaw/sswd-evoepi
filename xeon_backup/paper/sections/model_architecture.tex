% ══════════════════════════════════════════════════════════════════════
% MODEL ARCHITECTURE OVERVIEW
% Fact-checked against: sswd_evoepi/model.py, types.py, config.py
% ══════════════════════════════════════════════════════════════════════

\section{Model Architecture}
\label{sec:model_architecture}

\modelname{} is an individual-based model (IBM) that couples epidemiological,
demographic, genetic, and spatial dynamics to simulate the eco-evolutionary
consequences of sea star wasting disease in \pyc{}. Each agent represents
a single sea star tracked through its complete life history, carrying a
diploid genotype at 51 loci that determines three quantitative defense
traits against \vp{}. We chose an individual-based approach over
compartmental (ODE/PDE) models because SSWD dynamics depend critically on
individual-level heterogeneity in genetic resistance, body size, spatial
position, and disease stage---features that compartmental models cannot
represent without substantial loss of biological realism
\citep{deangelis2005individual, grimm2005individual}.

\subsection{Agent Representation}
\label{sec:agent_representation}

Each individual is represented as a record in a NumPy structured array
(\texttt{AGENT\_DTYPE}) comprising approximately 59 bytes per agent.
Table~\ref{tab:agent_fields} summarizes the principal state variables
grouped by functional module.

\begin{table}[H]
\centering
\caption{Agent state variables in \modelname{}.}
\label{tab:agent_fields}
\small
\begin{tabular}{llp{7.5cm}}
\toprule
\textbf{Module} & \textbf{Field} & \textbf{Description} \\
\midrule
\multirow{4}{*}{Spatial}
  & \texttt{x}, \texttt{y}       & Position within node habitat (m) \\
  & \texttt{heading}             & Movement heading (rad) \\
  & \texttt{speed}               & Instantaneous speed (m\,min$^{-1}$) \\
  & \texttt{node\_id}            & Home node index \\
\midrule
\multirow{4}{*}{Life history}
  & \texttt{size}                & Arm-tip diameter (mm) \\
  & \texttt{age}                 & Age (years, fractional) \\
  & \texttt{stage}               & Life stage (0--4; Table~\ref{tab:stages}) \\
  & \texttt{sex}                 & Sex (0 = female, 1 = male) \\
\midrule
\multirow{2}{*}{Disease}
  & \texttt{disease\_state}      & Compartment (S/E/I$_1$/I$_2$/D/R) \\
  & \texttt{disease\_timer}      & Days remaining in current disease stage \\
\midrule
\multirow{3}{*}{Genetics}
  & \texttt{resistance}          & Resistance score $r_i \in [0,1]$ \\
  & \texttt{tolerance}           & Tolerance score $t_i \in [0,1]$ \\
  & \texttt{recovery\_ability}   & Recovery/clearance score $c_i \in [0,1]$ \\
\midrule
\multirow{2}{*}{Spawning}
  & \texttt{has\_spawned}        & Bout count this season \\
  & \texttt{immunosuppression\_timer} & Post-spawning immunosuppression (days) \\
\midrule
\multirow{3}{*}{Administrative}
  & \texttt{alive}               & Active flag \\
  & \texttt{origin}              & Wild / captive-bred / AGF / wild-source \\
  & \texttt{pathogen\_virulence} & Virulence of infecting strain $v_i$ \\
\bottomrule
\end{tabular}
\end{table}

Genotypes are stored in a separate array of shape
$(N_\text{max}, 51, 2)$ with \texttt{int8} entries, where axis~1 indexes
loci and axis~2 indexes the two allele copies (diploid). This separation
from the agent record improves cache performance during non-genetic
operations (disease transmission, movement), which need not touch the
genotype array.

\begin{table}[H]
\centering
\caption{Life stages and size thresholds for \pyc{}.}
\label{tab:stages}
\small
\begin{tabular}{clcc}
\toprule
\textbf{Index} & \textbf{Stage} & \textbf{Size threshold (mm)} & \textbf{Reproductive} \\
\midrule
0 & Egg/Larva   & ---         & No  \\
1 & Settler     & Settlement  & No  \\
2 & Juvenile    & $\geq 10$   & No  \\
3 & Subadult    & $\geq 150$  & No  \\
4 & Adult       & $\geq 400$  & Yes \\
\bottomrule
\end{tabular}
\end{table}


\subsection{Node Structure}
\label{sec:node_structure}

The spatial domain is represented as a metapopulation network of $K$ discrete
habitat nodes. Each node encapsulates:
\begin{itemize}
  \item A population of agents (structured array + genotype array),
        initialized at local carrying capacity;
  \item Environmental state: sea surface temperature $T(t)$ (sinusoidal
        annual cycle with warming trend), salinity $S$, and tidal
        flushing rate $\phi_k$;
  \item A local Vibrio concentration $P_k(t)$ (bacteria\,mL$^{-1}$);
  \item Node metadata: latitude, habitat area, fjord classification.
\end{itemize}

Inter-node coupling occurs through two connectivity matrices:
\begin{enumerate}
  \item \textbf{Pathogen dispersal matrix} $\mathbf{D}$: governs daily
        exchange of waterborne \vp{} between nodes, parameterized with
        an exponential distance kernel (scale $D_P = 15$\,km);
  \item \textbf{Larval connectivity matrix} $\mathbf{C}$: governs annual
        dispersal of competent larvae among nodes, parameterized with a
        broader kernel (scale $D_L = 400$\,km) reflecting the extended
        pelagic larval duration of \pyc{}.
\end{enumerate}


\subsection{Simulation Loop}
\label{sec:simulation_loop}

The simulation advances in daily timesteps ($\Delta t = 1$\,day) nested
within an annual cycle. At each daily step, the following operations are
executed in sequence at every node (Figure~\ref{fig:loop_schematic}):

\begin{enumerate}
  \item \textbf{Environment update.}
        Compute $T_k(t)$ from a sinusoidal annual SST function with linear
        warming trend; update flushing rate $\phi_k$ (seasonally modulated
        for fjord nodes); salinity is constant per node.

  \item \textbf{Movement.}
        Agents execute a correlated random walk (CRW) with 24 hourly substeps
        per day. Movement speed is modulated by disease state
        ($\times 0.5$ for I$_1$, $\times 0.1$ for I$_2$, $\times 0$ for D).
        Elastic boundary reflection constrains agents within the habitat.

  \item \textbf{Disease dynamics.}
        Vibrio concentration is updated via an Euler step of the pathogen ODE.
        Susceptible agents are exposed to a force of infection that depends on
        local pathogen density, individual resistance, salinity, and body size.
        Infected agents progress through the SEIPD compartments with
        Erlang-distributed stage durations (Section~\ref{sec:disease_module}).

  \item \textbf{Pathogen dispersal.}
        Vibrio is exchanged between neighboring nodes via the $\mathbf{D}$
        matrix, representing waterborne transport.

  \item \textbf{Settlement.}
        Larval cohorts whose pelagic larval duration (PLD) has elapsed are
        settled into the local population via Beverton--Holt density-dependent
        recruitment, modulated by an adult-presence settlement cue (Allee
        effect).

  \item \textbf{Spawning.}
        During the spawning season (November--July), reproductively mature
        adults spawn stochastically with daily probability modulated by a
        seasonal Gaussian envelope centered on the peak spawning day. Female
        and male multi-bout spawning, sex-asymmetric cascade induction, and
        post-spawning immunosuppression are modeled explicitly.

  \item \textbf{Daily demographics.}
        Natural mortality is applied as a daily probability converted from
        stage-specific annual survival rates:
        \begin{equation}
          p_{\text{death,daily}} = 1 - S_{\text{annual}}^{1/365},
          \label{eq:daily_mortality}
        \end{equation}
        with a senescence overlay for individuals exceeding the senescence age
        ($\tau_\text{sen} = 50$\,yr). Growth follows the von~Bertalanffy
        differential form with daily-scaled stochastic noise; stage transitions
        are one-directional based on size thresholds (Table~\ref{tab:stages}).
\end{enumerate}

At the end of each simulated year, an annual step performs:
\begin{enumerate}
  \item \textbf{Larval dispersal} via the connectivity matrix $\mathbf{C}$:
        unsettled cohorts from all nodes are pooled, redistributed
        probabilistically among destination nodes, and settled at receiving
        nodes or retained in a pending queue for next-year daily settlement.
  \item \textbf{Disease introduction} (at the designated epidemic year):
        a fixed number of agents per node are seeded in the Exposed (E)
        compartment.
  \item \textbf{Genetic recording}: per-node allele frequencies, additive
        genetic variance $V_A$, and trait means are logged annually.
        Pre- and post-epidemic allele frequency snapshots are captured for
        calibration against genomic data.
\end{enumerate}


\subsection{Design Rationale}
\label{sec:design_rationale}

Several design choices distinguish \modelname{} from previous SSWD models:

\paragraph{Individual-based representation.}
SSWD mortality is strongly size-dependent
\citep[OR = 1.23 per 10\,mm;][]{eisenlord2016ochre}, genetically mediated
\citep{schiebelhut2018collapse}, and spatially heterogeneous. A compartmental
SIR/SEIR model would require aggregating these axes of variation into
homogeneous classes, losing the emergent eco-evolutionary dynamics that
arise from individual heterogeneity in resistance, tolerance, and recovery.
Following \citet{clement2024coevolution}, who demonstrated that individual-based
eco-evolutionary models are essential for predicting host--pathogen
coevolution in Tasmanian devil facial tumor disease, we track each
individual's genotype, phenotype, and infection history explicitly.

\paragraph{Continuous daily demographics.}
Rather than applying mortality, growth, and reproduction as annual pulses,
\modelname{} evaluates natural mortality and growth daily
(Eq.~\ref{eq:daily_mortality}), with spawning resolved to individual daily
events across a multi-month season. This avoids artificial synchronization
artifacts and allows disease--demography interactions (e.g., post-spawning
immunosuppression) to operate on their natural timescales.

\paragraph{Separated genotype storage.}
The 51-locus diploid genotype array (102 bytes per agent) is stored
separately from the agent state record. This ensures that the most
frequently accessed fields during daily disease and movement updates
(position, disease state, size) occupy contiguous memory, improving CPU
cache performance by a factor of $\sim$2--3$\times$ in profiled benchmarks.

\paragraph{Three-trait genetic architecture.}
The 51 loci are partitioned into three independently segregating trait
blocks of 17 loci each, controlling resistance (immune exclusion),
tolerance (damage limitation), and recovery (pathogen clearance). This
architecture captures the empirical observation that host defense against
infectious disease operates through mechanistically distinct pathways
that can evolve semi-independently \citep{raberg2009decomposing}.
