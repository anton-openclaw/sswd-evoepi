% ══════════════════════════════════════════════════════════════════════
% SENSITIVITY ANALYSIS
% Fact-checked against:
%   results/sensitivity/report/SENSITIVITY_ANALYSIS_REPORT.md  (R1-R2)
%   results/sensitivity_r3/MORRIS_ANALYSIS_v2.md               (R3)
%   results/sensitivity_r4/MORRIS_R4_ANALYSIS.md               (R4)
%   results/sensitivity_r4/morris_r4_ranking.json              (R4 data)
%   results/sensitivity_r4/morris_results.json                 (R4 raw)
% ══════════════════════════════════════════════════════════════════════

\section{Sensitivity Analysis}
\label{sec:sensitivity_analysis}

The \modelname{} model contains 47 uncertain parameters spanning six
modules: disease transmission and progression (16 parameters),
genetics and trait architecture (8), population dynamics (7),
spawning biology (7), pathogen virulence evolution (6), and spatial
connectivity (3). Most of these parameters have limited empirical
constraints (Section~\ref{sec:appendix_parameters}), necessitating a
systematic sensitivity analysis (SA) to identify which parameters
most influence model behavior and, critically, which parameter
\emph{interactions} dominate the system's dynamics. We conducted a
progressive, four-round SA campaign that tracked the model's growing
complexity from a single-trait, 3-node prototype through to the full
three-trait, 11-node eco-evolutionary framework.


% ──────────────────────────────────────────────────────────────────────
\subsection{Methods}
\label{sec:sa_methods}

\subsubsection{Morris Elementary Effects Screening}
\label{sec:morris_method}

Each SA round began with Morris elementary effects screening
\citep{Morris1991}, implemented via the SALib Python library
\citep{Herman2017}. The Morris method is a one-at-a-time (OAT)
design in which each parameter is perturbed along $r$ independent
trajectories through the $p$-dimensional input space. For parameter
$x_i$ in trajectory $j$, the elementary effect is
%
\begin{equation}
  d_{ij} = \frac{f(x_1, \ldots, x_i + \Delta_i, \ldots, x_p)
              - f(x_1, \ldots, x_i, \ldots, x_p)}{\Delta_i},
  \label{eq:elementary_effect}
\end{equation}
%
where $\Delta_i$ is the perturbation step. From these we compute
two summary statistics per parameter per metric
\citep{Campolongo2007}:
%
\begin{itemize}
  \item $\mu^*_i$: the mean of the \emph{absolute} elementary
    effects, measuring overall parameter importance regardless of
    sign;
  \item $\sigma_i$: the standard deviation of elementary effects,
    measuring interaction and nonlinearity strength.
\end{itemize}
%
When $\sigma_i / \mu^*_i > 1$, the parameter's influence on the
metric is dominated by interactions with other parameters rather
than by its direct (additive) effect \citep{Saltelli2008}. To
enable cross-metric comparison, we normalize $\mu^*$ by the
range of the metric across all trajectories, then rank parameters
by the mean normalized $\mu^*$ across all output metrics.

All rounds used $r = 20$ trajectories, yielding
$r \times (p + 1)$ total model evaluations per round (e.g.,
$20 \times 48 = 960$ runs for the 47-parameter Round~4).

\subsubsection{Sobol Variance Decomposition}
\label{sec:sobol_method}

Parameters surviving Morris screening advance to Sobol
variance-based global sensitivity analysis \citep{Sobol2001},
which decomposes the total output variance into contributions
from individual parameters and their interactions. Using the
Saltelli sampling scheme \citep{Saltelli2002}, $N(2p + 2)$
model evaluations produce two key indices for each parameter
$x_i$ and output metric $Y$:
%
\begin{itemize}
  \item $S_{1,i} = V_{x_i}[E_{x_{\sim i}}(Y | x_i)] \;/\;
    V(Y)$: the \emph{first-order} Sobol index, measuring the
    fraction of output variance attributable to $x_i$ alone;
  \item $S_{T,i} = 1 - V_{x_{\sim i}}[E_{x_i}(Y | x_{\sim i})]
    \;/\; V(Y)$: the \emph{total-order} index, capturing $x_i$'s
    contribution including all interactions with other parameters.
\end{itemize}
%
The gap $S_{T,i} - S_{1,i}$ quantifies the strength of parameter
interactions. When $S_{T,i} \gg S_{1,i}$, the parameter's
influence is mediated primarily through joint effects with other
parameters, implying that it cannot be calibrated independently.

\subsubsection{Output Metrics}
\label{sec:sa_metrics}

The SA tracks 23 output metrics capturing demographic, evolutionary,
spatial, and pathogen outcomes over 20-year simulations:
%
\begin{itemize}
  \item \textbf{Demographic:} population crash percentage,
    final population fraction, recovery (population returns to
    $>$50\% of initial), extinction (metapopulation collapse),
    peak single-year mortality, time to population nadir, total
    disease deaths, disease death fraction;
  \item \textbf{Evolutionary (host):} mean and maximum resistance
    shift ($\Delta \bar{r}$), tolerance shift ($\Delta \bar{t}$),
    recovery-trait shift ($\Delta \bar{c}$), additive variance
    retention ($V_A^{\text{post}}/V_A^{\text{pre}}$),
    evolutionary rescue index (composite of survival and
    resistance gain), total recovery events, recovery rate;
  \item \textbf{Spatial:} number of extinct nodes, north--south
    mortality gradient, fjord protection effect;
  \item \textbf{Pathogen:} mean final virulence, virulence shift
    ($\Delta \bar{v}$);
  \item \textbf{Spawning:} spawning participation rate, mean
    recruitment rate.
\end{itemize}


% ──────────────────────────────────────────────────────────────────────
\subsection{Progressive Sensitivity Analysis Design}
\label{sec:sa_rounds}

The SA was conducted in four rounds (Table~\ref{tab:sa_rounds}),
each corresponding to a major model extension. This progressive
design allows us to track how parameter importance shifts as model
complexity grows---a critical diagnostic for identifying emergent
behaviors introduced by new modules.

\begin{table}[htbp]
\centering
\caption{Summary of sensitivity analysis rounds. Each round
  incorporates all changes from prior rounds. ``New'' parameters
  are those added relative to the previous round.}
\label{tab:sa_rounds}
\small
\begin{tabular}{@{}lccccl@{}}
\toprule
\textbf{Round} & \textbf{Params} & \textbf{Metrics} &
  \textbf{Nodes} & \textbf{Runs} & \textbf{Key Changes} \\
\midrule
R1 (Morris) & 23 & 14 & 3 & 480 &
  Baseline: single resistance trait \\
R2 (Sobol) & 23 & 14 & 3 & 12{,}288 &
  Sobol decomposition of R1 params \\
R3 (Morris) & 43 & 20 & 3 & 880 &
  +20 params: pathogen evo, spawning, \\
  &&&&&
  continuous mortality, daily growth \\
R4 (Morris) & 47 & 23 & 11 & 960 &
  +4 params: three-trait genetics, \\
  &&&&&
  11-node stepping-stone network \\
\bottomrule
\end{tabular}
\end{table}

\paragraph{Rounds 1--2 (Pre--Three-Trait Baseline).}
The initial SA (Rounds~1--2) examined 23 parameters across disease
(13), population (7), genetics (1: $n_{\text{additive}}$), and
spawning (2) modules using a 3-node spatial network (Sitka, Howe
Sound, Monterey; $K = 5{,}000$ per node). The model at this stage
tracked a single resistance trait with $n_{\text{additive}}$
additive loci. Morris screening (480 runs) retained all 23
parameters for Sobol analysis (12{,}288 runs, $N = 256$).

The Sobol decomposition revealed that disease progression rate
$\mu_{\text{I2D,ref}}$ (I$_2 \to$ Death) was the single most
influential parameter (mean $S_T = 0.638$), followed by
$\text{susceptibility\_multiplier}$ ($S_T = 0.540$) and
$a_{\text{exposure}}$ ($S_T = 0.473$). A critical methodological
finding was that Morris and Sobol rankings \emph{disagreed}: Morris
identified \texttt{settler\_survival} and $\rho_{\text{rec}}$ as
the top drivers of population outcomes, while Sobol elevated
$\text{susceptibility\_multiplier}$ and $\mu_{\text{I2D,ref}}$.
This discrepancy arises because Morris measures marginal effects
from extreme-value perturbations, whereas Sobol captures
variance-weighted contributions including interactions. This
confirmed that Morris screening alone is insufficient for
identifying calibration priorities in this model.

\paragraph{Round 3 (Expanded Model, 3-Node).}
Round~3 added 20 parameters from four newly implemented modules:
pathogen virulence evolution (6 parameters: virulence--fitness
tradeoff exponents, mutation rate, initial virulence), expanded
spawning biology (4: male spontaneous spawning, readiness
induction, female bout limits, peak width), and additional disease
mechanics (immunosuppression duration, minimum susceptible age,
$\text{I}_1 \to \text{I}_2$ progression rate) and genetics
parameters ($\text{target\_mean\_r}$, Beta-distribution shape
parameters for initial allele frequencies). The network remained
at 3 nodes for comparability with R1--R2.

Morris screening (880 runs, 20 trajectories) revealed a dramatic
reshuffling: $\rho_{\text{rec}}$ (recovery rate) rose to \#1
($\mu^*_{\text{norm}} = 0.642$), displacing $\mu_{\text{I2D,ref}}$
from its R1--R2 dominance. This occurred because the transition
from discrete-stage to continuous daily mortality diluted the
I$_2 \to$ Death rate's marginal influence, while recovery rate's
role was amplified by its interaction with the new pathogen
evolution module (higher $\rho_{\text{rec}}$ imposes stronger
selection against virulent strains). All 43 parameters exceeded
the 5\% elimination threshold; zero were pruned.

\paragraph{Round 4 (Full Model, 11-Node).}
Round~4 represents the complete \modelname{} model with two
additions: (1) the three-trait genetic architecture (resistance,
tolerance, recovery; Section~\ref{sec:three_trait}), contributing
four new parameters ($\text{target\_mean\_c}$,
$\text{target\_mean\_t}$, $\tau_{\text{max}}$, $n_{\text{tolerance}}$);
and (2) an 11-node stepping-stone network spanning the latitudinal
range of \pyc{} habitat. The expanded spatial network was critical
for resolving spatial parameters that were undetectable at 3 nodes.
This round (960 runs, 48 cores on an Intel Xeon W-3365) provides
the most comprehensive screening of the model to date.


% ──────────────────────────────────────────────────────────────────────
\subsection{Round 4 Morris Results}
\label{sec:r4_results}

\subsubsection{Global Parameter Ranking}
\label{sec:r4_ranking}

Table~\ref{tab:r4_morris_full} presents the complete Round~4
Morris ranking for all 47 parameters, sorted by mean normalized
$\mu^*$ across 23 output metrics.
Figure~\ref{fig:morris_top20} shows the top 20 parameters
color-coded by module.

The top-10 parameters span four of six modules:
%
\begin{enumerate}
  \item $\rho_{\text{rec}}$ (recovery rate; $\mu^*_{\text{norm}} =
    0.889$) --- the rate at which infected individuals clear
    pathogen remains the single most influential parameter, as in
    R3. Its semi-additive behavior ($\sigma/\mu^* = 1.46$, the
    lowest interaction ratio of any parameter) reflects its direct
    mechanistic role: daily clearance probability
    $p_{\text{rec}} = \rho_{\text{rec}} \times c_i$ scales
    linearly with this rate regardless of other parameter values.
  \item $k_{\text{growth}}$ (von Bertalanffy growth rate;
    $\mu^*_{\text{norm}} = 0.633$) --- faster growth accelerates
    maturation and spawning eligibility, providing demographic
    compensation during epidemics. Rose from \#5 (R3) to \#2.
  \item $K_{\text{half}}$ (half-infective dose;
    $\mu^*_{\text{norm}} = 0.622$) --- the Michaelis--Menten
    saturation parameter controlling infection probability.
    Rose from \#8 to \#3.
  \item $P_{\text{env,max}}$ (environmental reservoir;
    $\mu^*_{\text{norm}} = 0.598$) --- background waterborne
    \vpshort{} input, independent of host shedding. Rose
    dramatically from \#11 to \#4, reflecting its interaction
    with the 11-node spatial network where environmental
    pathogen load varies with latitude and temperature.
  \item $n_{\text{resistance}}$ (number of resistance loci;
    $\mu^*_{\text{norm}} = 0.525$) --- genetic architecture
    of resistance. The largest rank gain of any parameter:
    \#19 $\to$ \#5 ($\Delta = +14$). The three-trait partition
    (17 loci per trait vs.\ the former 51 total) amplifies
    the sensitivity to how many loci underlie each defense
    mechanism.
  \item $s_0$ (settler survival;
    $\mu^*_{\text{norm}} = 0.509$) --- Beverton--Holt baseline
    recruitment. Dropped modestly from \#3 to \#6.
  \item $\sigma_{2,\text{eff}}$ (late-stage shedding rate;
    $\mu^*_{\text{norm}} = 0.431$).
  \item $\mu_{\text{I2D,ref}}$ (I$_2 \to$ Death rate;
    $\mu^*_{\text{norm}} = 0.419$) --- formerly the \#1 parameter
    in R1--R2 Sobol ($S_T = 0.638$), now \#8 in R4 Morris.
  \item $\sigma_{\text{spawn}}$ (spawning peak width;
    $\mu^*_{\text{norm}} = 0.392$) --- controls synchrony of
    the reproductive pulse; dropped from \#2 to \#9.
  \item $\text{target\_mean\_c}$ (initial mean recovery trait;
    $\mu^*_{\text{norm}} = 0.385$) --- a \emph{new} R4 parameter
    entering directly at \#10, confirming that the recovery trait
    ($c_i$) is the fastest-evolving defense in the model
    (Section~\ref{sec:three_trait}).
\end{enumerate}

\begin{figure}[htbp]
  \centering
  \includegraphics[width=\textwidth]{figures/morris_r4_top20.png}
  \caption{Top 20 parameters by mean normalized $\mu^*$ in
    Round~4 Morris screening (47 parameters, 23 metrics, 11-node
    network, 960 runs). Bars are color-coded by module. Error bars
    show 95\% bootstrap confidence intervals across 20 trajectories.}
  \label{fig:morris_top20}
\end{figure}

\subsubsection{Key Rank Shifts from Round 3}
\label{sec:r4_shifts}

The transition from R3 to R4 produced dramatic rank changes
(Figure~\ref{fig:morris_vs_r3}), driven by two structural
changes: the three-trait genetic architecture and the 11-node
spatial network.

\paragraph{Major rank gains.}
Six parameters gained $\geq$7 ranks (Table~\ref{tab:rank_shifts}):
%
\begin{itemize}
  \item $\sigma_{1,\text{eff}}$ (early shedding rate): \#43 $\to$
    \#16 ($\Delta = +27$). Early shedding now interacts with
    pathogen evolution: $\sigma_1$ shapes the initial epidemic
    wave that determines the selection regime on virulence.
  \item $\sigma_{v,\text{mut}}$ (virulence mutation step size):
    \#31 $\to$ \#14 ($\Delta = +17$). With 11 nodes providing
    diverse thermal and demographic environments, mutation rate
    controls how fast pathogen lineages adapt to local conditions.
  \item $T_{\text{ref}}$ (pathogen temperature optimum): \#34
    $\to$ \#18 ($\Delta = +16$). The latitudinal temperature
    gradient across 11 nodes (vs.\ 3) amplifies the importance
    of the thermal reference point.
  \item $n_{\text{resistance}}$: \#19 $\to$ \#5 ($\Delta = +14$),
    as discussed above.
  \item $\alpha_{\text{self,open}}$ (open-coast larval retention):
    \#39 $\to$ \#25 ($\Delta = +14$). Spatial retention was
    invisible at 3 nodes but becomes detectable with 11 nodes
    and realistic dispersal distances.
  \item $P_{\text{env,max}}$: \#11 $\to$ \#4 ($\Delta = +7$).
\end{itemize}

\paragraph{Major rank drops.}
Five parameters dropped $\geq$19 ranks:
%
\begin{itemize}
  \item $q_{\text{init},\beta_b}$ (Beta-distribution shape $b$):
    \#17 $\to$ \#46 ($\Delta = -29$). Initial allele-frequency
    shape is overwhelmed by the trait-specific mean parameters
    ($\text{target\_mean\_r/t/c}$).
  \item $F_0$ (reference fecundity): \#20 $\to$ \#47 ($\Delta =
    -27$). Diluted in the expanded 47-parameter space.
  \item Immunosuppression duration: \#15 $\to$ \#42 ($\Delta =
    -27$). Its effect is absorbed by spawning parameters and the
    recovery trait ($c_i$), which provides an alternative pathway
    through immunosuppressed periods.
  \item $\text{susceptibility\_multiplier}$: \#23 $\to$ \#44
    ($\Delta = -21$). This parameter was \#1 in the R1--R2 Sobol
    analysis ($S_T = 0.540$); its precipitous decline reflects
    absorption by the explicit resistance genetics---individual
    $r_i$ now captures susceptibility variation mechanistically,
    rendering the multiplicative modifier redundant.
  \item $p_{\text{spont},\female}$ (female spontaneous spawning):
    \#26 $\to$ \#45 ($\Delta = -19$).
\end{itemize}

\begin{figure}[htbp]
  \centering
  \includegraphics[width=\textwidth]{figures/morris_r4_vs_r3.png}
  \caption{Rank change from Round~3 to Round~4 for the 43 parameters
    common to both rounds. Positive values (rightward) indicate
    increased importance in R4; negative values (leftward) indicate
    decreased importance. Parameters are sorted by R4 rank. Four new
    R4 parameters (not shown) entered at ranks \#10, \#13, \#23,
    and~\#24.}
  \label{fig:morris_vs_r3}
\end{figure}

\begin{table}[htbp]
\centering
\caption{Largest rank shifts from R3 to R4 Morris screening. Positive
  $\Delta$ indicates increased importance.}
\label{tab:rank_shifts}
\small
\begin{tabular}{@{}llccl@{}}
\toprule
\textbf{Parameter} & \textbf{Module} &
  \textbf{R3 $\to$ R4} & $\boldsymbol{\Delta}$ &
  \textbf{Mechanism} \\
\midrule
$\sigma_{1,\text{eff}}$ & Disease &
  43 $\to$ 16 & $+27$ & Interacts with pathogen evolution \\
$\sigma_{v,\text{mut}}$ & Pathogen evo. &
  31 $\to$ 14 & $+17$ & Controls adaptation speed \\
$T_{\text{ref}}$ & Disease &
  34 $\to$ 18 & $+16$ & 11-node thermal gradient \\
$n_{\text{resistance}}$ & Genetics &
  19 $\to$ 5 & $+14$ & Three-trait partition \\
$\alpha_{\text{self,open}}$ & Spatial &
  39 $\to$ 25 & $+14$ & Resolvable at 11 nodes \\
\midrule
$q_{\text{init},\beta_b}$ & Genetics &
  17 $\to$ 46 & $-29$ & Absorbed by trait means \\
$F_0$ & Population &
  20 $\to$ 47 & $-27$ & Diluted in larger space \\
Immunosupp.\ duration & Disease &
  15 $\to$ 42 & $-27$ & Absorbed by recovery trait \\
Suscept.\ multiplier & Disease &
  23 $\to$ 44 & $-21$ & Absorbed by resistance genetics \\
\bottomrule
\end{tabular}
\end{table}

\subsubsection{New Three-Trait Parameters}
\label{sec:r4_new_params}

The four parameters introduced with the three-trait architecture
(Section~\ref{sec:three_trait}) immediately demonstrated meaningful
sensitivity:
%
\begin{itemize}
  \item $\text{target\_mean\_c}$ (initial mean recovery trait):
    rank \#10 ($\mu^*_{\text{norm}} = 0.385$). A top-10 entry
    confirms that recovery ($c_i$) is the dominant evolutionary
    pathway in the model, consistent with the validation finding
    that $\Delta \bar{c}$ exceeds $\Delta \bar{r}$ by $\sim$$7
    \times$ at all nodes (Section~\ref{sec:validation}).
  \item $\tau_{\text{max}}$ (maximum tolerance scaling): rank
    \#13 ($\mu^*_{\text{norm}} = 0.292$). The ceiling on how
    much tolerance extends I$_2$ survival matters because it sets
    the upper bound on the tolerance--recovery interaction.
  \item $\text{target\_mean\_t}$ (initial mean tolerance): rank
    \#23 ($\mu^*_{\text{norm}} = 0.197$). Mid-pack, reflecting
    the weaker selection signal on tolerance compared to recovery.
  \item $n_{\text{tolerance}}$ (number of tolerance loci): rank
    \#24 ($\mu^*_{\text{norm}} = 0.189$). Mid-pack, but notably
    the most interacting parameter in the entire model
    ($\sigma/\mu^* = 2.51$), suggesting tolerance's role is
    context-dependent.
\end{itemize}

\subsubsection{Universal Nonlinearity}
\label{sec:universal_nonlinearity}

A striking finding of the R4 Morris analysis is that \emph{every
one of the 47 parameters} has $\sigma/\mu^* > 1.0$
(Figure~\ref{fig:morris_interaction}). This means that no parameter
in the model acts additively---every parameter's effect on every
metric depends on the values of other parameters. The model is a
deeply coupled, nonlinear system.

The interaction ratio $\sigma/\mu^*$ ranges from 1.42
($s_0$, settler survival) to 2.52 ($\sigma_{v,\text{mut}}$,
virulence mutation rate). Two interaction tiers are apparent:
%
\begin{itemize}
  \item \textbf{Moderately interacting} ($\sigma/\mu^* < 1.5$;
    2 parameters): $\rho_{\text{rec}}$ (1.46) and $s_0$ (1.42).
    These parameters operate semi-additively---their effects are
    relatively stable across parameter space. For
    $\rho_{\text{rec}}$, this reflects its direct mechanistic role:
    daily clearance probability scales linearly with recovery rate
    regardless of context.
  \item \textbf{Strongly to extremely interacting}
    ($\sigma/\mu^* > 1.5$; 45 parameters): the remaining
    parameters exhibit moderate to extreme nonlinearity.
    The most interacting parameters are genetic/evolutionary:
    $\sigma_{v,\text{mut}}$ (2.52), $n_{\text{tolerance}}$
    (2.51), $q_{\text{init},\beta_a}$ (2.45), and
    $\alpha_{\text{SRS}}$ (2.34). These control \emph{adaptation
    rates} that feed back on disease dynamics, which feed back on
    selection pressures---creating cascading interaction loops.
\end{itemize}
%
This universal nonlinearity has profound implications for
calibration: no parameter can be tuned independently. Joint
calibration via approximate Bayesian computation (ABC) or Markov
chain Monte Carlo methods is essential.

\begin{figure}[htbp]
  \centering
  \includegraphics[width=\textwidth]{figures/morris_r4_interaction.png}
  \caption{Morris $\mu^*$ vs.\ $\sigma$ scatter for all 47
    parameters (R4). The dashed line shows $\sigma = \mu^*$ (unit
    interaction ratio). All parameters fall above this line,
    indicating universal nonlinearity. Symbol color indicates
    module; symbol size scales with mean normalized $\mu^*$.}
  \label{fig:morris_interaction}
\end{figure}

\subsubsection{Module-Level Sensitivity}
\label{sec:module_sa}

Figure~\ref{fig:morris_modules} summarizes sensitivity by module.
The disease module dominates in both parameter count (16) and mean
importance ($\overline{\mu^*_{\text{norm}}} = 0.332$), but genetics
punches above its weight: with only 8 parameters, it achieves the
second-highest mean importance ($\overline{\mu^*_{\text{norm}}} =
0.260$), and its top parameter ($n_{\text{resistance}}$) ranks \#5
globally. The pathogen evolution module, despite being entirely new
in R3--R4, achieves a mean $\mu^*_{\text{norm}} = 0.185$ with
$\sigma_{v,\text{mut}}$ at \#14---virulence evolution is not
negligible and must be retained in calibration.

Spatial parameters ($\overline{\mu^*_{\text{norm}}} = 0.171$) are
detectable for the first time at 11 nodes. At the 3-node
configuration of R1--R3, these parameters ranked \#39--\#42; at 11
nodes, they rise to \#25--\#33. This confirms that adequate spatial
resolution is necessary to capture dispersal and retention
dynamics.

\begin{figure}[htbp]
  \centering
  \includegraphics[width=0.85\textwidth]{figures/morris_r4_modules.png}
  \caption{Module-level sensitivity summary for R4 Morris screening.
    Bar height shows mean normalized $\mu^*$ for each module;
    whiskers show the range from minimum to maximum parameter within
    each module. Number of parameters per module shown in
    parentheses.}
  \label{fig:morris_modules}
\end{figure}


% ──────────────────────────────────────────────────────────────────────
\subsection{Cross-Round Parameter Trajectories}
\label{sec:cross_round}

Tracking individual parameters across all four rounds reveals
which parameters have stable importance versus those whose
influence is contingent on model structure
(Figure~\ref{fig:morris_heatmap}):

\paragraph{Consistently important.}
$\rho_{\text{rec}}$, $a_{\text{exposure}}$, and
$\sigma_{2,\text{eff}}$ remain in the top 12 across all rounds.
These are robust calibration targets regardless of model
configuration.

\paragraph{Structurally contingent.}
$\mu_{\text{I2D,ref}}$ was \#1 in R1--R2 Sobol but dropped to
\#7--\#8 in R3--R4 Morris after the switch to continuous daily
mortality. $\text{susceptibility\_multiplier}$ fell from \#1--\#2
(R1--R2) to \#44 (R4) as explicit resistance genetics absorbed its
role. These shifts demonstrate that parameter importance can be an
\emph{artifact of model structure}, not a property of the
underlying biology, underscoring the need for structural
sensitivity analysis alongside parametric SA.

\paragraph{Emergent with complexity.}
$P_{\text{env,max}}$, $n_{\text{resistance}}$, and all pathogen
evolution parameters only revealed their importance at $\geq$11
nodes or $\geq$43 parameters. Simple model configurations
systematically underestimate the importance of spatial and
evolutionary parameters.

\begin{figure}[htbp]
  \centering
  \includegraphics[width=\textwidth]{figures/morris_r4_heatmap.png}
  \caption{Parameter--metric sensitivity heatmap (R4 Morris).
    Cell color indicates normalized $\mu^*$ for each
    parameter--metric pair. Parameters (rows) are sorted by global
    rank; metrics (columns) are grouped by category. White cells
    indicate $\mu^*_{\text{norm}} < 0.05$.}
  \label{fig:morris_heatmap}
\end{figure}


% ──────────────────────────────────────────────────────────────────────
\subsection{Sobol Variance Decomposition: Rounds 1--2 and Ongoing}
\label{sec:sobol_results}

\subsubsection{R1--R2 Sobol Results}

The Round~1--2 Sobol analysis (23 parameters, $N = 256$, 12{,}288
runs) revealed massive parameter interactions across the model.
For most metrics, total-order indices $S_T$ far exceeded
first-order indices $S_1$, meaning that parameter combinations
dominate behavior over individual effects. Notable interaction
signatures include:
%
\begin{itemize}
  \item \textbf{Extinction:} $\sigma_{2,\text{eff}}$ had
    $S_T = 1.51$ but $S_1 \approx 0$---extinction risk is
    \emph{entirely} driven by interactions between shedding rate
    and other parameters.
  \item \textbf{Fjord protection:} $a_{\text{exposure}}$ had
    $S_T = 0.96$ but $S_1 = -0.12$---a negative first-order index
    means the parameter's effect \emph{reverses sign} depending on
    the values of other parameters.
  \item \textbf{Recovery:} $\text{susceptibility\_multiplier}$ had
    $S_T = 0.96$ but $S_1 = 0.38$---60\% of its influence arises
    through interactions.
\end{itemize}

\subsubsection{Round 4 Sobol (In Progress)}
\label{sec:r4_sobol}

A Round~4 Sobol analysis is currently running on a 48-core Intel
Xeon W-3365 server. With 47 parameters and $N = 512$, the Saltelli
design requires $N(2p + 2) = 49{,}152$ model evaluations at
$\sim$25~s each. At 12 parallel workers, the estimated wall time
is approximately 7~days. This analysis will provide the first
variance decomposition of the full three-trait, 11-node model and
will enable direct comparison with the R1--R2 Sobol indices to
quantify how the three-trait architecture redistributes variance
among parameters.

Based on the R4 Morris results, we prioritize convergence
monitoring for the top-10 parameters and anticipate particularly
informative second-order ($S_2$) indices for the following
parameter pairs:
%
\begin{itemize}
  \item $\rho_{\text{rec}} \times \text{target\_mean\_c}$:
    recovery rate $\times$ recovery genetics (both affect pathogen
    clearance);
  \item $P_{\text{env,max}} \times a_{\text{exposure}}$:
    environmental reservoir $\times$ transmission rate (dual
    exposure pathways);
  \item $n_{\text{resistance}} \times \sigma_{v,\text{mut}}$:
    host genetic architecture $\times$ pathogen adaptation rate
    (coevolutionary arms race);
  \item $k_{\text{growth}} \times s_0$:
    growth rate $\times$ recruitment (demographic compensation).
\end{itemize}


% ──────────────────────────────────────────────────────────────────────
\subsection{Summary and Implications}
\label{sec:sa_summary}

The four-round sensitivity analysis yields five principal findings:

\begin{enumerate}
  \item \textbf{Recovery dominates.} The base recovery rate
    $\rho_{\text{rec}}$ is consistently the most influential
    parameter across rounds and model configurations, yet has zero
    empirical basis. Determining whether \pyc{} can clear
    \vpshort{} infections---and at what rate---is the single
    highest-priority empirical question for model calibration.

  \item \textbf{Genetic architecture is a structural choice with
    major consequences.} The number of resistance loci
    ($n_{\text{resistance}}$) ranks \#5 globally and cannot be
    calibrated from data without high-resolution GWAS. The
    three-trait partition amplifies this sensitivity: 17 loci per
    trait behave very differently from 51 loci in a single trait.

  \item \textbf{Parameter importance is model-contingent.}
    $\text{susceptibility\_multiplier}$ fell from \#1 (R1--R2
    Sobol) to \#44 (R4 Morris) as explicit genetics absorbed its
    role; $\mu_{\text{I2D,ref}}$ fell from \#1 to \#8 with
    continuous mortality. SA results from simpler model
    configurations cannot be extrapolated to the full model.

  \item \textbf{Universal nonlinearity demands joint calibration.}
    All 47 parameters interact ($\sigma/\mu^* > 1.0$). No
    parameter can be tuned independently. Approximate Bayesian
    computation with sequential Monte Carlo sampling (ABC-SMC) is
    the appropriate calibration framework.

  \item \textbf{Spatial resolution matters.} Spatial and
    environmental parameters only emerge as important at $\geq$11
    nodes. The planned 150-node full-coastline simulation will
    likely reveal additional spatially contingent sensitivities.
\end{enumerate}


% ──────────────────────────────────────────────────────────────────────
% FULL PARAMETER TABLE
% ──────────────────────────────────────────────────────────────────────

\begin{landscape}
\begin{longtable}{@{}rllS[table-format=1.3]S[table-format=1.2]rr@{}}
\caption{Complete Round~4 Morris parameter ranking (47 parameters,
  23 metrics, 11-node network, 960 runs). Mean normalized $\mu^*$
  is averaged across all metrics. The $\sigma/\mu^*$ ratio indicates
  interaction strength ($> 1$: interaction-dominated). R3 Rank
  column shows the parameter's position in the 43-parameter R3
  analysis; ``---'' indicates a new R4 parameter.}
\label{tab:r4_morris_full} \\

\toprule
\textbf{Rank} & \textbf{Parameter} & \textbf{Module} &
  {$\overline{\mu^*_{\text{norm}}}$} &
  {$\sigma/\mu^*$} &
  \textbf{R3} & $\boldsymbol{\Delta}$ \\
\midrule
\endfirsthead

\multicolumn{7}{c}%
  {\tablename\ \thetable\ (continued)} \\
\toprule
\textbf{Rank} & \textbf{Parameter} & \textbf{Module} &
  {$\overline{\mu^*_{\text{norm}}}$} &
  {$\sigma/\mu^*$} &
  \textbf{R3} & $\boldsymbol{\Delta}$ \\
\midrule
\endhead

\midrule
\multicolumn{7}{r}{\textit{Continued on next page}} \\
\endfoot

\bottomrule
\endlastfoot

1  & $\rho_{\text{rec}}$              & Disease    & 0.889 & 1.46 & 1  & {---}    \\
2  & $k_{\text{growth}}$              & Population & 0.633 & 1.63 & 5  & {$\uparrow$3}  \\
3  & $K_{\text{half}}$                & Disease    & 0.622 & 1.84 & 8  & {$\uparrow$5}  \\
4  & $P_{\text{env,max}}$             & Disease    & 0.598 & 1.92 & 11 & {$\uparrow$7}  \\
5  & $n_{\text{resistance}}$          & Genetics   & 0.525 & 1.78 & 19 & {$\uparrow$14} \\
6  & $s_0$ (settler survival)         & Population & 0.509 & 1.42 & 3  & {$\downarrow$3} \\
7  & $\sigma_{2,\text{eff}}$          & Disease    & 0.431 & 1.95 & 10 & {$\uparrow$3}  \\
8  & $\mu_{\text{I2D,ref}}$           & Disease    & 0.419 & 1.98 & 7  & {$\downarrow$1} \\
9  & $\sigma_{\text{spawn}}$ (peak width) & Spawning & 0.392 & 2.03 & 2  & {$\downarrow$7} \\
10 & $\text{target\_mean\_c}$         & Genetics   & 0.385 & 2.08 & {---} & {---}     \\
11 & $a_{\text{exposure}}$            & Disease    & 0.379 & 2.20 & 6  & {$\downarrow$5} \\
12 & $T_{\text{VBNC}}$                & Disease    & 0.355 & 2.07 & 9  & {$\downarrow$3} \\
13 & $\tau_{\text{max}}$              & Genetics   & 0.292 & 2.05 & {---} & {---}     \\
14 & $\sigma_{v,\text{mut}}$          & Path.\ evo.& 0.259 & 2.52 & 31 & {$\uparrow$17} \\
15 & $\alpha_{\text{kill}}$           & Path.\ evo.& 0.254 & 2.25 & 22 & {$\uparrow$7}  \\
16 & $\sigma_{1,\text{eff}}$          & Disease    & 0.245 & 2.24 & 43 & {$\uparrow$27} \\
17 & $\text{target\_mean\_r}$         & Genetics   & 0.236 & 1.86 & 4  & {$\downarrow$13} \\
18 & $T_{\text{ref}}$                 & Disease    & 0.229 & 1.94 & 34 & {$\uparrow$16} \\
19 & min.\ susceptible age            & Disease    & 0.229 & 2.04 & 13 & {$\downarrow$6} \\
20 & $\sigma_D$                       & Disease    & 0.211 & 1.96 & 29 & {$\uparrow$9}  \\
21 & female max bouts                 & Spawning   & 0.206 & 1.95 & 32 & {$\uparrow$11} \\
22 & readiness induction prob.        & Spawning   & 0.204 & 2.26 & 33 & {$\uparrow$11} \\
23 & $\text{target\_mean\_t}$         & Genetics   & 0.197 & 2.05 & {---} & {---}     \\
24 & $n_{\text{tolerance}}$           & Genetics   & 0.189 & 2.51 & {---} & {---}     \\
25 & $\alpha_{\text{self,open}}$      & Spatial    & 0.187 & 2.07 & 39 & {$\uparrow$14} \\
26 & $D_L$                            & Spatial    & 0.178 & 2.29 & 18 & {$\downarrow$8} \\
27 & $\kappa_{\text{mf}}$ (M$\to$F induction) & Spawning & 0.176 & 2.07 & 16 & {$\downarrow$11} \\
28 & $s_{\text{min}}$                 & Disease    & 0.175 & 1.84 & 36 & {$\uparrow$8}  \\
29 & $v_{\text{init}}$                & Path.\ evo.& 0.173 & 2.13 & 12 & {$\downarrow$17} \\
30 & $p_{\text{spont,m}}$              & Spawning   & 0.169 & 2.11 & 28 & {$\downarrow$2} \\
31 & $\mu_{\text{I1I2,ref}}$          & Disease    & 0.156 & 1.97 & 14 & {$\downarrow$17} \\
32 & $q_{\text{init},\beta_a}$        & Genetics   & 0.150 & 2.45 & 40 & {$\uparrow$8}  \\
33 & $\alpha_{\text{self,fjord}}$     & Spatial    & 0.149 & 2.00 & 42 & {$\uparrow$9}  \\
34 & senescence age                   & Population & 0.148 & 1.66 & 21 & {$\downarrow$13} \\
35 & $\gamma_{\text{early}}$          & Path.\ evo.& 0.148 & 2.03 & 30 & {$\downarrow$5} \\
36 & $\alpha_{\text{SRS}}$            & Population & 0.146 & 2.34 & 35 & {$\downarrow$1} \\
37 & $\alpha_{\text{prog}}$           & Path.\ evo.& 0.143 & 2.09 & 38 & {$\uparrow$1}  \\
38 & $\mu_{\text{EI1,ref}}$           & Disease    & 0.141 & 2.19 & 27 & {$\downarrow$11} \\
39 & $L_{\text{min,repro}}$           & Population & 0.139 & 2.06 & 25 & {$\downarrow$14} \\
40 & $\alpha_{\text{shed}}$           & Path.\ evo.& 0.136 & 2.12 & 41 & {$\uparrow$1}  \\
41 & $\kappa_{\text{fm}}$ (F$\to$M induction) & Spawning & 0.130 & 1.79 & 24 & {$\downarrow$17} \\
42 & immunosupp.\ duration            & Disease    & 0.127 & 2.07 & 15 & {$\downarrow$27} \\
43 & $\gamma_{\text{fert}}$           & Population & 0.122 & 2.21 & 37 & {$\downarrow$6} \\
44 & suscept.\ multiplier             & Disease    & 0.111 & 2.03 & 23 & {$\downarrow$21} \\
45 & $p_{\text{spont,f}}$              & Spawning   & 0.110 & 1.67 & 26 & {$\downarrow$19} \\
46 & $q_{\text{init},\beta_b}$        & Genetics   & 0.104 & 2.20 & 17 & {$\downarrow$29} \\
47 & $F_0$                            & Population & 0.102 & 1.83 & 20 & {$\downarrow$27} \\

\end{longtable}
\end{landscape}
