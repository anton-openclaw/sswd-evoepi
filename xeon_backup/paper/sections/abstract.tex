% Abstract
Sea star wasting disease (SSWD) caused one of the largest wildlife mass mortality events in marine ecosystems, driving the sunflower sea star (\pyc{}) to a 90.6\% range-wide decline and IUCN Critically Endangered status. The recent identification of \vp{} strain FHCF-3 as a causative agent, combined with active captive breeding and the first experimental outplanting of captive-bred juveniles, creates an urgent need for quantitative tools to guide recovery. We present \modelname{}, an individual-based, spatially explicit eco-evolutionary epidemiological model coupling \vpshort{} transmission dynamics with polygenic host evolution under sweepstakes reproductive success. Each agent carries a diploid genotype across 51 loci governing three fitness-related traits --- resistance (immune exclusion), tolerance (damage limitation), and recovery (pathogen clearance) --- that evolve in response to disease-driven selection. Disease dynamics follow an SEIR-type compartmental structure with an environmental pathogen reservoir, pathogen evolution along a virulence--transmission tradeoff, temperature-dependent forcing, and recovery returning individuals to the susceptible pool (reflecting the absence of adaptive immunity in echinoderms). Reproduction implements sweepstakes reproductive success with $N_e/N \sim 10^{-3}$, sex-asymmetric spawning induction, and post-spawning immunosuppression. Four rounds of global sensitivity analysis (Morris screening and Sobol variance decomposition) across up to 47 parameters reveal that model behavior is dominated by nonlinear interactions among disease mortality rate, host susceptibility, environmental pathogen pressure, and genetic architecture, with resistance replacing recovery as the dominant adaptive response when reinfection is permitted. The model provides a framework for evaluating captive-bred release strategies, assisted gene flow, and the feasibility of evolutionary rescue on conservation-relevant timescales.
