\section{\pyc{}-Specific Considerations}
\label{sec:biology}

The general theory above must be grounded in the specific 
biology of \pyc{} to generate realistic predictions. This 
section collects the biological parameters and constraints 
relevant to breeding program design.

\subsection{Reproductive biology}

\subsubsection{Sexual reproduction}
\pyc{} is gonochoristic (separate sexes) with external 
fertilization via broadcast spawning. Key parameters:
\begin{itemize}
    \item \textbf{Sexual maturity}: $\sim$2 years (estimated 
    from growth rates; not precisely known in captivity)
    \item \textbf{Spawning}: Annual, triggered by temperature 
    cues (spring--summer)
    \item \textbf{Fecundity}: Females release millions of eggs 
    per spawning event. Fertilization success depends on 
    proximity and synchrony.
    \item \textbf{Larval duration}: 6--10 weeks as a 
    planktotrophic bipinnaria/brachiolaria larva
    \item \textbf{Settlement}: Larvae settle onto hard substrate 
    and metamorphose
\end{itemize}

\subsubsection{Implications for breeding}
\begin{enumerate}
    \item \textbf{Generation time $\sim$2 years}: 8 generations 
    of selective breeding = $\sim$16 years. This is long but 
    not unprecedented for conservation breeding programs 
    (cf.\ California condor, black-footed ferret).
    
    \item \textbf{Very high fecundity}: Not a bottleneck. 
    A single cross can produce thousands of juveniles for 
    screening. This makes within-family selection 
    (\Cref{subsec:family_selection}) highly effective.
    
    \item \textbf{Equal sex ratio assumed}: No sex-linked 
    resistance known. Equal allocation to both sexes.
    
    \item \textbf{No clonal reproduction}: Unlike some 
    echinoderms, \pyc{} does not reproduce asexually. 
    Every generation requires sexual crossing.
\end{enumerate}

\subsection{Immune system}

Echinoderms possess only innate immunity — no adaptive immune 
system, no immunological memory, no antibodies.

\subsubsection{Implications for the model}
\begin{enumerate}
    \item \textbf{No acquired immunity}: Recovered individuals 
    return to the susceptible state (R $\to$ S in our model). 
    They can be reinfected.
    
    \item \textbf{Resistance is genetic, not learned}: There is 
    no vaccination or immunization strategy. Resistance must 
    come from heritable genetic variation — exactly what a 
    breeding program provides.
    
    \item \textbf{No maternal antibody transfer}: Offspring do 
    not inherit any immunological protection from parents 
    beyond genetic resistance alleles.
\end{enumerate}

\subsection{Disease ecology relevant to breeding}

\subsubsection{Vibrio pectenicida as the causative agent}
\citet{prentice2025koch} established Koch's postulates for 
\textit{Vibrio pectenicida} in \pyc{} SSWD. Key findings:
\begin{itemize}
    \item 92\% attack rate in experimental challenge (46/50)
    \item Exposure to death: $11.6 \pm 3.3$ days at $\sim$13°C
    \item Temperature-dependent virulence (Arrhenius-scaled)
\end{itemize}

\subsubsection{Implications for breeding}
\begin{enumerate}
    \item \textbf{High attack rate} means most wild \pyc{} 
    have been exposed. Survivors represent the tail of the 
    resistance distribution — already screened by nature.
    
    \item \textbf{Temperature dependence} means different 
    latitudes experience different selection pressures. 
    Southern populations (warmer water) face stronger 
    disease pressure and therefore stronger selection for 
    resistance.
    
    \item \textbf{Challenge assays are possible}: The Prentice 
    protocol provides a standardized method for phenotyping 
    disease resistance in captive individuals. This could 
    be used to validate genotype-based predictions from the 
    model.
\end{enumerate}

\subsection{Population status}
\label{subsec:pop_status}

\subsubsection{Wild populations}
\begin{itemize}
    \item Pre-epidemic (pre-2013): Abundant throughout range, 
    though not precisely censused at most sites
    \item Post-epidemic: 90--99\% decline across the range 
    \citep{montecino2020sunflower}. IUCN Critically Endangered 
    (2020).
    \item Recovery signs: Sporadic observations of juveniles 
    at some sites, but recurrent wasting events prevent 
    sustained recovery
    \item Genetic bottleneck: Small surviving populations 
    have reduced $N_e$, increasing drift and inbreeding
\end{itemize}

\subsubsection{Captive populations}
\begin{itemize}
    \item Multiple facilities maintaining broodstock 
    (Friday Harbor Labs, Birch Aquarium, others)
    \item First releases: FHL 2023 (caged), FHL 2024 
    (20 stars, open release), California 2025 (47 stars, 
    46/47 survived 1 month)
    \item Genetic composition of broodstock: not well 
    characterized for resistance loci
\end{itemize}

\subsection{Practical constraints}
\label{subsec:practical}

\begin{enumerate}
    \item \textbf{No resistance assay yet}: The Schiebelhut 
    GWAS loci have not been validated as markers for 
    resistance phenotype. Until they are, ``screening for 
    resistance'' means phenotypic challenge assays (slow, 
    lethal to non-resistant individuals) rather than 
    genotyping (fast, non-lethal).
    
    \item \textbf{Generation time}: 2 years minimum in 
    captivity, possibly longer under suboptimal conditions.
    
    \item \textbf{Captive space}: Sea star aquaculture is 
    space-intensive. Maintaining hundreds of adults through 
    multiple breeding generations requires significant 
    facility capacity.
    
    \item \textbf{Pedigree tracking}: Difficult in broadcast 
    spawners. May require genetic parentage assignment 
    (microsatellites or SNP panels) rather than physical 
    tracking.
    
    \item \textbf{Regulatory constraints}: Releases of 
    captive-bred individuals may require permits and 
    environmental review, especially across state/provincial 
    boundaries.
    
    \item \textbf{Disease management in captivity}: Captive 
    populations can experience SSWD outbreaks. Biosecurity 
    protocols are essential to protect broodstock.
\end{enumerate}

\subsection{What the model can and cannot predict}

\begin{center}
\begin{tabular}{p{0.45\textwidth}p{0.45\textwidth}}
\toprule
\textbf{The model CAN predict} & 
\textbf{The model CANNOT predict} \\
\midrule
Expected trait distributions of 
wild survivors at each site & 
Actual genetic composition of 
specific living individuals \\
\addlinespace
Relative effectiveness of breeding 
strategies & 
Absolute generation counts (depends 
on real $h^2$) \\
\addlinespace
Optimal release sizes and locations 
for genetic impact & 
Captive husbandry success rates \\
\addlinespace
Allele frequency trajectories under 
different scenarios & 
Inbreeding depression magnitude 
(not modeled) \\
\addlinespace
Cost-benefit trade-offs between 
breeding program designs & 
Regulatory or political feasibility \\
\bottomrule
\end{tabular}
\end{center}
