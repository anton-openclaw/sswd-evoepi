\section{Inbreeding and Genetic Diversity}
\label{sec:inbreeding}

Breeding programs operate on small populations. Small populations 
lose genetic diversity through drift and accumulate inbreeding. 
This section develops the theory for tracking and managing both.

\subsection{Inbreeding coefficient}

The inbreeding coefficient $F_i$ of individual $i$ is the 
probability that the two alleles at a randomly chosen locus 
are identical by descent (IBD):
\begin{equation}
    F_i = \Prob(\text{two alleles at a locus are IBD})
    \label{eq:f_definition}
\end{equation}

In our model, we can estimate $F$ directly from genotype data 
as excess homozygosity relative to Hardy--Weinberg expectation:
\begin{equation}
    \hat{F}_i = 1 - \frac{H_{\text{obs},i}}{H_{\text{exp}}}
    = 1 - \frac{\frac{1}{L}\sum_\ell 
    \mathbf{1}[a_{i,\ell,1} \neq a_{i,\ell,2}]}
    {\frac{1}{L}\sum_\ell 2 q_\ell (1 - q_\ell)}
    \label{eq:f_genomic}
\end{equation}
where $L$ is the number of loci, $H_{\text{obs},i}$ is individual 
$i$'s observed heterozygosity, and $H_{\text{exp}}$ is the 
Hardy--Weinberg expected heterozygosity.

\subsection{Rate of inbreeding}

In a population of effective size $N_e$, inbreeding accumulates 
at rate:
\begin{equation}
    \Delta F = \frac{1}{2 N_e}
    \label{eq:delta_f}
\end{equation}
per generation. After $g$ generations:
\begin{equation}
    F_g = 1 - (1 - \Delta F)^g \approx 1 - e^{-g/(2N_e)}
    \label{eq:f_after_g}
\end{equation}

\subsubsection{Effective population size under selection}

When only $N_s$ of $N$ individuals are selected as parents:
\begin{equation}
    N_e = \frac{4 N_m N_f}{N_m + N_f}
    \label{eq:ne_unequal_sex}
\end{equation}
where $N_m$ and $N_f$ are the numbers of male and female 
parents. In \pyc{}, sexes are separate (gonochoristic), so 
equal sex allocation ($N_m = N_f = N_s/2$) gives $N_e = N_s$.

With unequal family sizes (variance in reproductive 
contribution $\sigma_k^2$):
\begin{equation}
    N_e = \frac{4N - 4}{2 + \sigma_k^2}
    \label{eq:ne_family_size}
\end{equation}
Equal family sizes ($\sigma_k^2 = 0$) double $N_e$ compared 
to random variation ($\sigma_k^2 = 2$ under Poisson).

\subsection{The 50/500 rule and its application}
\label{subsec:50_500}

The classic conservation genetics guidelines 
\citep{franklin1980evolutionary, jamieson2012applicability}:

\begin{itemize}
    \item $N_e \geq 50$: Avoids severe inbreeding depression 
    in the short term ($\Delta F \leq 1\%$ per generation)
    \item $N_e \geq 500$: Maintains sufficient genetic variance 
    for long-term evolutionary response to selection 
    ($V_A$ lost at $\sim$0.1\%/generation)
\end{itemize}

For a captive breeding program with discrete generations:

\begin{center}
\begin{tabular}{cccc}
\toprule
$N_e$ & $\Delta F$/gen & $F$ after 5 gen & $F$ after 10 gen \\
\midrule
25 & 2.0\% & 9.6\% & 18.3\% \\
50 & 1.0\% & 4.9\% & 9.6\% \\
100 & 0.5\% & 2.5\% & 4.9\% \\
200 & 0.25\% & 1.2\% & 2.5\% \\
500 & 0.10\% & 0.5\% & 1.0\% \\
\bottomrule
\end{tabular}
\end{center}

\begin{remark}
A realistic captive program might maintain 50--100 breeding 
adults. With $N_e \approx 50$--$100$ and a generation time 
of $\sim$2 years, reaching $F = 10\%$ (a commonly used 
threshold for significant inbreeding depression) takes 
5--10 generations (10--20 years). This is a real constraint 
for multi-generation selective breeding.
\end{remark}

\subsection{Inbreeding depression}
\label{subsec:inbreeding_depression}

Inbreeding depression arises from increased homozygosity of 
deleterious recessive alleles. The expected decline in a fitness 
trait:
\begin{equation}
    \bar{w}(F) = \bar{w}(0) \cdot e^{-BF}
    \label{eq:inbreeding_depression}
\end{equation}
where $B$ is the number of lethal equivalents per diploid 
genome. For marine invertebrates, $B$ typically ranges 
from 2--12 \citep{obrien1994genetic, hedrick2002inbreeding}.

\begin{remark}
Our model \textbf{does not currently implement inbreeding 
depression}. This is a known gap. The 51 modeled loci control 
disease traits only; we do not track deleterious alleles at 
other loci. Adding inbreeding depression would require either:
\begin{enumerate}
    \item Explicit deleterious loci (adds many parameters)
    \item A phenotypic penalty proportional to genomic $F$ 
    (simpler, empirically calibratable)
\end{enumerate}
We flag this as a \textbf{priority model extension} for 
conservation applications.
\end{remark}

\subsection{Diversity metrics}

\subsubsection{Expected heterozygosity}
\begin{equation}
    H_e = \frac{1}{L} \sum_{\ell=1}^{L} 2 q_\ell (1 - q_\ell)
    \label{eq:he}
\end{equation}
Decreases monotonically as alleles fix (either direction).

\subsubsection{Allelic richness}
For biallelic loci, allelic richness is simply the number of 
loci that are polymorphic ($0 < q_\ell < 1$). A locus is 
``lost'' when either allele fixes. Under drift:
\begin{equation}
    \Prob(\text{allele lost by generation } g) \approx 
    1 - e^{-g/(2N_e)} \quad \text{(for rare alleles)}
    \label{eq:allele_loss}
\end{equation}

\subsubsection{Additive genetic variance}
\begin{equation}
    V_A^{(g)} = \sum_\ell 2 q_\ell^{(g)} 
    (1 - q_\ell^{(g)}) \alpha_\ell^2
    \label{eq:va_over_time}
\end{equation}
This is the ``fuel'' for future selection response. Once 
$V_A \to 0$, no further genetic gain is possible through 
selection alone.

\subsection{Managing the gain--diversity trade-off}
\label{subsec:gain_diversity}

The fundamental trade-off: stronger selection increases 
short-term genetic gain but accelerates diversity loss.

\subsubsection{Constrained optimization approach}

The optimal contribution selection framework 
(\Cref{eq:ocs}) solves this formally. In practice, for 
our discrete-locus model, we can implement a simpler 
version:

\begin{enumerate}
    \item Rank all candidates by breeding value (resistance 
    score, or selection index)
    \item Starting from the top, add candidates to the 
    breeding pool
    \item For each candidate, compute the marginal change 
    in $N_e$ if they are included
    \item Stop when either: (a) the target pool size is 
    reached, or (b) including the next candidate would 
    push $\Delta F$ above the threshold
\end{enumerate}

\subsubsection{Practical guideline}

For a breeding program targeting resistance while maintaining 
diversity:
\begin{equation}
    N_{\text{breeding}} \geq 
    \max\!\left(N_{\min}^{(\Delta F)}, \; 
    N_{\min}^{(\text{alleles})}\right)
    \label{eq:breeding_pool_size}
\end{equation}
where $N_{\min}^{(\Delta F)}$ ensures $\Delta F \leq$ target, 
and $N_{\min}^{(\text{alleles})}$ ensures retention of rare 
alleles. For 51 biallelic loci with minimum allele frequency 
$q_{\min} \approx 0.01$:
\begin{equation}
    N_{\min}^{(\text{alleles})} \approx 
    \frac{1}{q_{\min}} = 100
    \label{eq:nmin_alleles}
\end{equation}
(need $\sim$100 individuals to expect $\geq 1$ copy of a 
1\% frequency allele).
