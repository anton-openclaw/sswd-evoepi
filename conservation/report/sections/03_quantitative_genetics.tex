\section{Quantitative Genetics of Selection Response}
\label{sec:quant_gen}

Given the genetic architecture defined in \Cref{sec:architecture}, 
we now derive the expected response to selection — the central 
quantity for breeding program design.

\subsection{The breeder's equation}

The fundamental prediction tool in quantitative genetics is the 
breeder's equation \citep{lush1937animal, falconer1996introduction}:
\begin{equation}
    R = h^2 \cdot S
    \label{eq:breeders}
\end{equation}
where:
\begin{itemize}
    \item $R = \bar{\tau}_{t+1} - \bar{\tau}_t$ is the 
    response to selection (change in population mean trait 
    value per generation)
    \item $h^2 = V_A / V_P$ is the narrow-sense heritability
    \item $S = \bar{\tau}_{\text{selected}} - \bar{\tau}_t$ 
    is the selection differential (difference between the mean 
    of selected parents and the population mean)
\end{itemize}

\subsection{Heritability in our model}

In the SSWD-EvoEpi model, traits are purely genetically determined 
with no environmental variance:
\begin{equation}
    V_P = V_A + \underbrace{V_D}_{= 0} + \underbrace{V_E}_{= 0} = V_A
    \label{eq:heritability_model}
\end{equation}

Under strict additivity (no dominance), $V_D = 0$. We impose no 
environmental stochasticity on trait expression, so $V_E = 0$. 
Therefore:
\begin{equation}
    h^2 = 1.0 \quad \text{(in the model)}
    \label{eq:h2_model}
\end{equation}

\begin{remark}
This is an idealization. In real \pyc{}, heritability of disease 
resistance is unknown but certainly $< 1$. Environmental factors 
(nutritional state, temperature stress, prior infections, microbiome 
composition) would add $V_E > 0$, reducing $h^2$. However, for the 
purpose of \emph{relative} comparisons between breeding strategies 
within the model, $h^2 = 1$ simplifies the analysis without 
affecting ranking of strategies. When interpreting absolute 
generation counts, we should apply a correction factor 
$h^2_{\text{real}} / 1.0$ to account for this.
\end{remark}

\subsection{Selection differential from truncation selection}

In a breeding program, truncation selection retains the top 
fraction $p$ of individuals. The selection differential is:
\begin{equation}
    S = i \cdot \sigma_P
    \label{eq:selection_differential}
\end{equation}
where $i$ is the \textbf{selection intensity}, a function of $p$ alone.

For a normal distribution (approximately valid for polygenic traits 
with many loci):
\begin{equation}
    i = \frac{\phi(z_p)}{1 - \Phi(z_p)} = \frac{\phi(z_p)}{p}
    \label{eq:selection_intensity}
\end{equation}
where $\Phi$ is the standard normal CDF, $\phi$ is the PDF, and 
$z_p = \Phi^{-1}(1 - p)$ is the truncation point.

\begin{center}
\begin{tabular}{ccc}
\toprule
Fraction selected ($p$) & Selection intensity ($i$) & Practical meaning \\
\midrule
0.50 & 0.80 & Keep top half \\
0.20 & 1.40 & Keep top fifth \\
0.10 & 1.76 & Keep top tenth \\
0.05 & 2.06 & Intensive selection \\
0.01 & 2.67 & Extreme selection \\
\bottomrule
\end{tabular}
\end{center}

\subsection{Predicted gain per generation}

Combining \Cref{eq:breeders,eq:selection_differential} with 
$h^2 = 1$:
\begin{equation}
    R = i \cdot \sigma_A = i \cdot \sqrt{V_A}
    \label{eq:gain_per_gen}
\end{equation}

For resistance with $n_R = 17$ loci, the additive standard deviation is:
\begin{equation}
    \sigma_A = \sqrt{\sum_{\ell=1}^{17} 2 q_\ell (1 - q_\ell) 
    \alpha_\ell^2}
    \label{eq:sigma_a_resistance}
\end{equation}

\begin{remark}
$V_A$ is \textbf{not constant} across generations. As selection 
pushes allele frequencies toward fixation ($q \to 1$), the term 
$q(1-q) \to 0$ and $V_A$ decreases. This means the breeder's 
equation predicts diminishing returns over generations — the 
response slows as the population approaches fixation.
\end{remark}

\subsection{Multi-generation prediction}

For generation $g$, the allele frequency at locus $\ell$ after 
selection is:
\begin{equation}
    q_\ell^{(g+1)} = q_\ell^{(g)} + \Delta q_\ell^{(g)}
    \label{eq:allele_freq_update}
\end{equation}

Under truncation selection on an additive trait, the single-locus 
allele frequency change is approximately \citep{barton2000multilocus}:
\begin{equation}
    \Delta q_\ell \approx i \cdot 
    \frac{\alpha_\ell \cdot q_\ell(1-q_\ell)}{\sigma_P}
    \label{eq:delta_q}
\end{equation}

This gives us a system of coupled difference equations — one per 
locus — that we can iterate forward numerically.

\begin{definition}[Fixation]
Locus $\ell$ is \textbf{fixed} for the protective allele when 
$q_\ell = 1$ (all individuals homozygous derived). The trait 
reaches its maximum when all loci are fixed.
\end{definition}

\begin{proposition}[Time to fixation scales with initial frequency]
\label{prop:fixation_time}
Loci with initially high $q_\ell$ fix first. Loci with very low 
$q_\ell$ and small $\alpha_\ell$ take the longest — they contribute 
little to the trait and experience weak selection. The ``last loci'' 
to fix determine the total generations needed to reach $\tau \approx 1$.
\end{proposition}

\subsection{The selection--variance trade-off}
\label{subsec:sel_var_tradeoff}

Intensive selection (small $p$) increases $S$ but reduces the 
effective population size of the breeding pool:
\begin{equation}
    N_e \approx \frac{4 N_{\text{selected}} \cdot N_{\text{total}}}
    {N_{\text{selected}} + N_{\text{total}}}
    \label{eq:ne_selection}
\end{equation}

For $N_{\text{selected}} \ll N_{\text{total}}$:
$N_e \approx 4 N_{\text{selected}}$.

Small $N_e$ means:
\begin{enumerate}
    \item Faster genetic drift (random loss of alleles)
    \item Faster inbreeding ($\Delta F = 1/(2N_e)$ per generation)
    \item Risk of losing rare but valuable alleles at small-effect loci
\end{enumerate}

This creates a fundamental tension in breeding program design: 
\textbf{selecting harder gives faster genetic gain but erodes 
the genetic diversity needed for long-term adaptation.} The 
optimal strategy balances these forces 
(see \Cref{sec:inbreeding}).

\subsection{Response in non-normal distributions}

Our trait distributions are \emph{not} normal — they are 
right-skewed (see \Cref{sec:architecture}, the allele 
frequencies are Beta-distributed and many loci have low $q$). 
The standard breeder's equation assumes normality.

For non-normal distributions, the selection differential must 
be computed directly:
\begin{equation}
    S = \E[\tau_i \mid \tau_i \geq \tau^*] - \E[\tau_i]
    \label{eq:s_nonnormal}
\end{equation}
where $\tau^*$ is the truncation threshold. This requires the 
full trait distribution, which we compute numerically from the 
allele frequency vector and effect sizes.

\subsection{Multi-trait selection}
\label{subsec:multi_trait}

When selecting on multiple traits simultaneously, the Smith--Hazel 
selection index \citep{smith1936discriminant, hazel1943genetic} 
provides the optimal linear combination:
\begin{equation}
    I_i = \mathbf{b}^T \boldsymbol{\tau}_i, \qquad
    \mathbf{b} = \mathbf{G}^{-1} \mathbf{w}
    \label{eq:selection_index}
\end{equation}
where $\mathbf{G}$ is the genetic variance--covariance matrix and 
$\mathbf{w}$ is the vector of economic (or fitness) weights.

In our model, because traits are genetically independent 
(\Cref{prop:independence}), $\mathbf{G}$ is diagonal:
\begin{equation}
    \mathbf{G} = \begin{pmatrix}
        V_A^{(r)} & 0 & 0 \\
        0 & V_A^{(t)} & 0 \\
        0 & 0 & V_A^{(c)}
    \end{pmatrix}
    \label{eq:g_matrix}
\end{equation}

and the index simplifies to:
\begin{equation}
    I_i = \frac{w_r}{V_A^{(r)}} r_i + 
    \frac{w_t}{V_A^{(t)}} t_i + 
    \frac{w_c}{V_A^{(c)}} c_i
    \label{eq:index_simplified}
\end{equation}

The fitness weights $\mathbf{w}$ should reflect the marginal 
fitness contribution of each trait. From \Cref{eq:infection_hazard,%
eq:tolerance,eq:recovery}, resistance has by far the largest effect 
on survival per exposure event. We derive appropriate weights in 
\Cref{sec:breeding}.
