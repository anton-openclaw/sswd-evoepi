\section{Introduction}
\label{sec:introduction}

Sea star wasting disease (SSWD) has caused one of the largest documented 
mass mortality events in a marine invertebrate, reducing \pyc{} populations 
by an estimated 90--99\% across its range from Alaska to Baja California 
\citep{harvell2019disease, montecino2020sunflower}. As a keystone predator 
of urchins, the loss of \pyc{} has triggered trophic cascades leading to 
kelp forest collapse in multiple regions \citep{schultz2016urchin}.

Conservation efforts now center on captive breeding and reintroduction 
\citep{seastarlab2024}. The first caged outplanting trials were conducted 
at Friday Harbor Laboratories in 2023, followed by the first open release 
of 20 captive-bred juveniles in July 2024, and a California release of 
47 juveniles in December 2025 (with 46/47 surviving the first month). 
These programs face a fundamental question: \textbf{how do we design a 
breeding program that maximizes disease resistance in released stock while 
maintaining sufficient genetic diversity for long-term population viability?}

\subsection{What this module provides}

The SSWD-EvoEpi model tracks individual genotypes at 51 biallelic loci 
controlling three disease-related traits (resistance, tolerance, recovery). 
This individual-level genetic resolution enables analyses that 
population-level models cannot:

\begin{enumerate}[leftmargin=*]
    \item \textbf{Predicted genetic state.} What do the trait distributions 
    of surviving wild populations look like at each site, right now?
    
    \item \textbf{Screening effort.} How many wild individuals must be 
    sampled to find founders with desired resistance levels?
    
    \item \textbf{Breeding optimization.} Which crossing strategy 
    (random, assortative, complementary) maximizes resistance gain per 
    generation, and at what cost to genetic diversity?
    
    \item \textbf{Reintroduction design.} How many captive-bred 
    individuals, released where and when, shift the evolutionary 
    trajectory of wild populations?
\end{enumerate}

\subsection{Approach}

Each section of this document follows the same structure:
\begin{itemize}[leftmargin=*]
    \item \textbf{First-principles derivation} of the relevant 
    quantitative genetics or population genetics theory
    \item \textbf{Mapping to our model} — how the general theory 
    specializes to 51 loci, 3 traits, and biallelic architecture
    \item \textbf{Computable predictions} — equations that take model 
    parameters as input and produce conservation-relevant output
    \item \textbf{Implementation notes} — pointers to the code that 
    implements each result
\end{itemize}

All derivations assume the calibrated model is available. Until then, 
we use validation-run parameters as placeholders and note where results 
will change.

\subsection{Scope and limitations}

This module addresses the \emph{genetics} of conservation. It does not 
address habitat restoration, water quality management, or disease 
treatment — only the question of which individuals to breed, how to 
cross them, and where to release their offspring. The model does not 
currently include inbreeding depression (reduced fitness from 
homozygosity of deleterious recessives), which we flag as a known gap 
throughout.
