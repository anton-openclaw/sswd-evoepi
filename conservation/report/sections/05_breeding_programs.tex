\section{Breeding Program Design}
\label{sec:breeding}

Given a set of founders, how should we cross them to maximize 
resistance in offspring destined for release? This section 
derives the theory for several crossing strategies and compares 
their expected performance.

\subsection{Fitness weights for multi-trait selection}

From the disease model, we can derive the relative importance 
of each trait for individual survival. Consider a single 
exposure event. The probability of surviving it is:
\begin{equation}
    w(r, t, c) = \underbrace{r}_{\text{avoid infection}} + 
    \underbrace{(1 - r)}_{\text{get infected}} \cdot 
    \underbrace{s(t, c)}_{\text{survive infection}}
    \label{eq:fitness_function}
\end{equation}
where $s(t, c)$ is the probability of recovering given infection:
\begin{equation}
    s(t, c) = p_{\text{rec,I}_1}(c) + 
    [1 - p_{\text{rec,I}_1}(c)] \cdot 
    p_{\text{rec,I}_2}(t, c)
    \label{eq:survival_given_infection}
\end{equation}

The I$_2$ recovery probability over the (extended) I$_2$ period:
\begin{equation}
    p_{\text{rec,I}_2}(t, c) = 
    1 - (1 - \rho_{\text{rec}} \cdot c)^{D_{I_2}(t)}
    \label{eq:p_rec_i2}
\end{equation}
where the mean I$_2$ duration in days is:
\begin{equation}
    D_{I_2}(t) = \frac{1}{\mu_{I_2D} \cdot 
    \max(1 - t \cdot \tau_{\max}, \; 0.05)}
    \label{eq:i2_duration}
\end{equation}

\subsubsection{Marginal fitness effects}

The partial derivatives reveal relative trait importance:
\begin{align}
    \frac{\partial w}{\partial r} &= 1 - s(t, c) 
    \label{eq:dw_dr} \\
    \frac{\partial w}{\partial t} &= (1 - r) \cdot 
    \frac{\partial s}{\partial t} 
    \label{eq:dw_dt} \\
    \frac{\partial w}{\partial c} &= (1 - r) \cdot 
    \frac{\partial s}{\partial c}
    \label{eq:dw_dc}
\end{align}

At population-mean trait values ($r = 0.15$, $t = 0.10$, 
$c = 0.02$), where $s \approx 0.002$:
\begin{align}
    \frac{\partial w}{\partial r} &\approx 0.998 \\
    \frac{\partial w}{\partial t} &\approx 0.85 \cdot 
    (1 - 0.15) \cdot 0.001 \approx 0.001 \\
    \frac{\partial w}{\partial c} &\approx (1 - 0.15) \cdot 
    0.05 \cdot 1.9 \approx 0.08
\end{align}

\begin{remark}
Resistance is \textbf{$\sim$1000$\times$ more important} than 
tolerance and \textbf{$\sim$12$\times$ more important} than 
recovery at population-mean trait values. This ordering persists 
across all biologically plausible parameter ranges. 
\textbf{Breeding programs should weight resistance heavily.}
\end{remark}

However, this ranking changes at high resistance:
at $r = 0.5$, tolerance and recovery matter more per marginal 
unit because the remaining infection events are rarer but the 
stakes of each are the same.

\subsection{Crossing strategies}

We formalize four crossing strategies and derive their 
expected offspring distributions.

\subsubsection{Strategy 1: Random mating}

Parents paired uniformly at random from the selected pool. 
Expected offspring trait mean:
\begin{equation}
    \E[\tau_{\text{offspring}}] = \bar{\tau}_{\text{parents}}
    \label{eq:random_mating_mean}
\end{equation}
(the midparent value, exactly, under additivity).

Offspring variance comes from Mendelian segregation:
\begin{equation}
    V_{\text{offspring}} = \frac{1}{2} V_{\text{within-parents}} + 
    V_{\text{segregation}}
    \label{eq:offspring_variance}
\end{equation}
where segregation variance is:
\begin{equation}
    V_{\text{seg}} = \sum_{\ell} \frac{\alpha_\ell^2}{4} 
    \cdot h_\ell^{(p_1)} \cdot h_\ell^{(p_2)}
    \label{eq:segregation_var}
\end{equation}
with $h_\ell^{(p)} = \mathbf{1}[\text{parent } p 
\text{ is heterozygous at locus } \ell]$.

\subsubsection{Strategy 2: Assortative mating}

Pair the highest-resistance individuals together: rank parents 
by $r$, pair 1st with 2nd, 3rd with 4th, etc. This maximizes 
the mean resistance of offspring but does \emph{not} maximize 
the maximum. Two parents both homozygous-derived at the same 
loci produce offspring identical to themselves at those loci — 
no gain.

\subsubsection{Strategy 3: Complementary mating}
\label{subsec:complementary}

Pair parents that cover different loci. For parent genotypes 
$G_i$ and $G_j$, define the expected offspring resistance as:
\begin{equation}
    \E[r_{\text{offspring}}(i,j)] = \sum_{\ell} \alpha_\ell 
    \cdot \bar{q}_\ell^{(i,j)}
    \label{eq:complementary_expected}
\end{equation}
where $\bar{q}_\ell^{(i,j)}$ is the expected frequency of the 
protective allele in offspring from parents $i$ and $j$ at 
locus $\ell$:
\begin{equation}
    \bar{q}_\ell^{(i,j)} = \frac{1}{2}\left(
    \frac{a_{i,\ell,1} + a_{i,\ell,2}}{2} + 
    \frac{a_{j,\ell,1} + a_{j,\ell,2}}{2}\right)
    = \frac{g_{i,\ell} + g_{j,\ell}}{4}
    \label{eq:offspring_freq}
\end{equation}
where $g_{i,\ell} = a_{i,\ell,1} + a_{i,\ell,2} \in \{0, 1, 2\}$ 
is the count of protective alleles.

The key insight: if parent 1 is homozygous-derived at locus $\ell$ 
($g = 2$) and parent 2 is homozygous-ancestral ($g = 0$), all 
offspring are heterozygous ($g = 1$) at that locus, contributing 
$\alpha_\ell/2$. But if parent 1 has $g = 2$ at locus $\ell$ 
while parent 2 has $g = 2$ at locus $\ell'$ (a \emph{different} 
locus), offspring get $\alpha_\ell/2 + \alpha_{\ell'}/2$ from 
those two loci — more than either parent contributed from a 
single locus.

\begin{proposition}[Complementary $>$ assortative for max offspring]
Under additive genetics with multiple loci, complementary mating 
produces offspring with higher maximum trait values than 
assortative mating, because complementary pairs combine 
protective alleles from different loci rather than duplicating 
the same alleles.
\end{proposition}

\subsubsection{Strategy 4: Optimal contribution selection}

The gold standard for balancing genetic gain and diversity 
\citep{meuwissen1997maximizing, woolliams2015genetic}. 
Maximize:
\begin{equation}
    \text{max}_{\mathbf{c}} \; \mathbf{c}^T \boldsymbol{\tau}
    \quad \text{subject to} \quad 
    \mathbf{c}^T \mathbf{A} \mathbf{c} \leq 
    \frac{1}{2N_e^*}
    \label{eq:ocs}
\end{equation}
where $\mathbf{c}$ is the vector of parental contributions 
(fraction of next generation sired by each individual), 
$\boldsymbol{\tau}$ is the trait vector, $\mathbf{A}$ is the 
additive relationship matrix, and $N_e^*$ is the target 
effective population size.

The constraint ensures that the rate of inbreeding does not 
exceed $\Delta F = 1/(2N_e^*)$ per generation.

\begin{remark}
OCS requires computing the relationship matrix $\mathbf{A}$, 
which in our model can be derived from the genotype matrix: 
$A_{ij} = (2/L) \sum_\ell \sum_k a_{i,\ell,k} a_{j,\ell,k}$ 
(genomic relationship).
\end{remark}

\subsection{Expected generations to resistance targets}

Given a strategy with per-generation gain $R_g$ (which 
decreases as $V_A$ erodes), the number of generations to 
reach target $\tau^*$ from initial mean $\bar{\tau}_0$ is:
\begin{equation}
    G(\tau^*) = \min\left\{g : \bar{\tau}_0 + 
    \sum_{k=0}^{g-1} R_k \geq \tau^*\right\}
    \label{eq:gens_to_target}
\end{equation}

This must be computed iteratively because $R_g$ depends on $V_A^{(g)}$, 
which depends on the allele frequencies after $g$ rounds of selection.

\subsection{Family structure and within-family selection}
\label{subsec:family_selection}

In practice, \pyc{} can produce very large families (millions 
of larvae from a single spawning). This creates an opportunity 
for \textbf{within-family selection}: from a cross of parents 
$i \times j$, select the best offspring. The variance within a 
family is the segregation variance (\Cref{eq:segregation_var}), 
which is maximized when parents are heterozygous at many loci.

\begin{proposition}[Within-family gain]
The expected best-of-$m$ offspring from a cross has resistance:
\begin{equation}
    \E[r_{(m)}^{(i \times j)}] = 
    \E[r_{\text{offspring}}^{(i \times j)}] + 
    \sigma_{\text{seg}}^{(i \times j)} \cdot 
    \E[Z_{(m)}]
    \label{eq:within_family_best}
\end{equation}
where $\E[Z_{(m)}]$ is the expected maximum of $m$ standard 
normal draws, approximately $\sqrt{2 \ln m}$ for large $m$.
\end{proposition}

With $m = 100$ offspring per cross: $\E[Z_{(100)}] \approx 2.51$.
With $m = 1000$: $\E[Z_{(1000)}] \approx 3.09$.

This is powerful: within-family selection leverages the high 
fecundity of sea stars to get extra genetic gain without 
reducing the number of families (and thus without increasing 
inbreeding rate).
