\section{Screening Theory: Finding Good Founders}
\label{sec:screening}

Before a breeding program begins, we must find founders from the 
wild. This section derives the statistical theory of founder 
screening — how many individuals must be sampled to find the 
desired genetic quality.

\subsection{The basic screening problem}

Let $F(\tau)$ be the CDF of a trait in the wild population. We 
want to find at least one individual with $\tau \geq \tau^*$ 
(some target threshold). If we sample $n$ individuals independently:

\begin{equation}
    \Prob(\text{at least one} \geq \tau^*) = 
    1 - [F(\tau^*)]^n = 1 - (1 - p)^n
    \label{eq:screening_basic}
\end{equation}
where $p = 1 - F(\tau^*) = \Prob(\tau \geq \tau^*)$ is the 
exceedance probability.

\begin{definition}[Required sample size]
For confidence level $\gamma$ (e.g., 0.95):
\begin{equation}
    n(\gamma, p) = \left\lceil 
    \frac{\ln(1 - \gamma)}{\ln(1 - p)} 
    \right\rceil
    \label{eq:required_n}
\end{equation}
\end{definition}

For small $p$, this simplifies to $n \approx -\ln(1-\gamma) / p$. 
At 95\% confidence: $n \approx 3/p$. At 50\% confidence: 
$n \approx 0.7/p$.

\subsection{Trait distribution from allele frequencies}

The exceedance probability $p$ requires the trait distribution, 
which depends on allele frequencies. For $n_R$ biallelic loci 
with frequencies $q_1, \ldots, q_{n_R}$ and effects 
$\alpha_1, \ldots, \alpha_{n_R}$:

The trait value is $\tau = \sum_\ell \alpha_\ell x_\ell$ where 
$x_\ell \sim (1/2) \text{Binomial}(2, q_\ell)$. The exact 
distribution is a convolution of scaled binomials — tractable 
numerically via characteristic functions or direct convolution, 
but unwieldy analytically.

\begin{proposition}[Normal approximation]
By the Lyapunov CLT, for sufficiently many loci with no single 
dominant effect:
\begin{equation}
    \tau \overset{d}{\approx} \mathcal{N}(\mu_\tau, \sigma_\tau^2)
    \label{eq:normal_approx}
\end{equation}
where:
\begin{align}
    \mu_\tau &= \sum_\ell \alpha_\ell q_\ell 
    \label{eq:trait_mean} \\
    \sigma_\tau^2 &= \sum_\ell \frac{\alpha_\ell^2}{2} 
    q_\ell (1 - q_\ell)
    \label{eq:trait_var}
\end{align}
The factor of $1/2$ in the variance arises because $x_\ell = 
(a_1 + a_2)/2$ with $\Var[a] = q(1-q)$.
\end{proposition}

\begin{remark}
The normal approximation is decent for the bulk of the distribution 
but underestimates the tail probabilities. Our trait distributions 
are right-skewed (many loci with low $q$), so the normal 
approximation \textbf{underestimates} the probability of finding 
high-trait individuals. For screening calculations, we should use 
the exact (simulated) distribution rather than the normal 
approximation.
\end{remark}

\subsection{Expected best individual from a sample}

When screening $n$ individuals, we care about the \emph{maximum} 
trait value observed. The expected value of the maximum (the first 
order statistic of the upper tail) is:

\begin{equation}
    \E[\tau_{(n)}] = \int_{-\infty}^{\infty} 
    \tau \cdot n \cdot f(\tau) \cdot [F(\tau)]^{n-1} \, d\tau
    \label{eq:expected_max}
\end{equation}

For a normal distribution, this is approximately 
\citep{david2003order}:
\begin{equation}
    \E[\tau_{(n)}] \approx \mu_\tau + \sigma_\tau 
    \cdot \left(\Phi^{-1}\!\left(\frac{n}{n+1}\right)\right)
    \label{eq:expected_max_normal}
\end{equation}

For large $n$, this grows as $\sigma_\tau \sqrt{2 \ln n}$ — 
logarithmically slow. This is the mathematical basis for the 
\textbf{diminishing returns} of screening: doubling the sample 
size does not double the best individual found.

\subsection{Multi-site screening}

If populations at different sites have different trait 
distributions $F_k(\tau)$ with different means $\mu_k$ (due to 
different selection histories), the optimal screening allocation 
across $K$ sites with budget $N = \sum_k n_k$ maximizes:
\begin{equation}
    \E\left[\max_{k} \tau_{(n_k)}^{(k)}\right]
    \label{eq:multisite_max}
\end{equation}

This is a constrained optimization problem. Intuitively:
\begin{itemize}
    \item Sites with higher mean resistance yield better 
    individuals per sample
    \item Sites with higher variance (more genetic diversity) 
    have heavier tails — rare but valuable outliers
    \item Sites with very small surviving populations may not 
    be worth sampling (population size limits $n_k$)
\end{itemize}

The optimal allocation depends on the site-specific trait 
distributions, which the calibrated model provides.

\subsection{Screening for complementarity}
\label{subsec:screening_complementarity}

For breeding purposes, we don't just want the single best 
individual — we want a \emph{set of founders} with 
complementary genotypes. Two individuals are complementary if 
they carry protective alleles at different loci.

\begin{definition}[Locus union]
For individuals $i$ and $j$, the \textbf{locus union} is the 
number of resistance loci at which at least one parent carries 
at least one protective allele:
\begin{equation}
    U(i, j) = \sum_{\ell=1}^{n_R} 
    \mathbf{1}\!\left[(a_{i,\ell,1} + a_{i,\ell,2}) > 0 
    \;\lor\; (a_{j,\ell,1} + a_{j,\ell,2}) > 0\right]
    \label{eq:locus_union}
\end{equation}
Maximum value: $n_R$ (every locus covered).
\end{definition}

\begin{definition}[Complementarity score]
\begin{equation}
    C(i, j) = U(i, j) - O(i, j)
    \label{eq:complementarity}
\end{equation}
where $O(i,j)$ is the overlap (loci where both parents have 
protective alleles). High $C$ means the parents cover different 
loci — their offspring can inherit protective alleles from both 
and achieve higher resistance than either parent.
\end{definition}

\begin{proposition}[Expected union of two random individuals]
If locus $\ell$ has protective allele frequency $q_\ell$, the 
probability that at least one of two random individuals carries 
$\geq 1$ protective allele at this locus is:
\begin{equation}
    \Prob(\text{locus } \ell \text{ covered}) = 
    1 - (1 - q_\ell)^4
    \label{eq:locus_coverage_prob}
\end{equation}
(since each individual has 2 independent allele draws, so 4 total).
The expected union is:
\begin{equation}
    \E[U] = \sum_{\ell=1}^{n_R} 
    \left[1 - (1 - q_\ell)^4\right]
    \label{eq:expected_union}
\end{equation}
\end{proposition}

\subsection{Screening cost model}
\label{subsec:screening_cost}

In practice, screening has costs: collection effort, genetic 
assays, and holding facilities. A simple cost model:
\begin{equation}
    C_{\text{total}} = c_{\text{collect}} \cdot n + 
    c_{\text{assay}} \cdot n + 
    c_{\text{hold}} \cdot n_{\text{keep}} \cdot T_{\text{hold}}
    \label{eq:screening_cost}
\end{equation}
where $n$ is total screened, $n_{\text{keep}}$ is the number 
retained as founders, and $T_{\text{hold}}$ is the holding 
duration. This module provides the genetic analysis; the cost 
parameters must be supplied by conservation practitioners.
