\subsection{Phenotyping constraints: the challenge assay}
\label{subsec:binary_phenotyping}

The screening and breeding theory above assumes we can measure 
resistance as a continuous value. In practice, we cannot. There 
is currently no validated genetic marker panel for SSWD resistance 
in \pyc{}. The primary phenotyping method available is the 
\textbf{challenge assay}: expose an individual to 
\textit{Vibrio pectenicida} and observe the outcome 
\citep{prentice2025koch}. This section develops a framework 
for understanding what a challenge assay can and cannot tell us 
about the underlying genetics, and how this constrains breeding 
program design.

\note{All numerical examples in this section use pre-calibration 
parameter values as illustrations. The specific ratios and 
probabilities will change after model calibration. What persists 
is the \emph{structure} of the problem: the relationships between 
traits, observables, assay design, and selection response.}

\subsubsection{What does a challenge assay actually measure?}

A well-monitored challenge assay produces more than a binary 
outcome. If individuals are observed regularly (e.g., daily 
health assessment), the following are observable:

\begin{enumerate}
    \item \textbf{Whether the individual shows symptoms.} In 
    the model, visible disease corresponds to reaching the 
    I$_1$ (early symptomatic) stage. The \emph{only} way to 
    avoid showing symptoms is to avoid infection entirely, 
    which depends on resistance $r_i$ and stochastic luck 
    (the Bernoulli draw in infection probability).
    
    \item \textbf{Whether a symptomatic individual recovers 
    or dies.} Recovery requires pathogen clearance via the 
    recovery trait $c_i$ (and potentially the extended 
    survival window from tolerance $t_i$).
    
    \item \textbf{Time from symptom onset to death} (for 
    those who die). In the model, this duration is influenced 
    by tolerance: high-$t$ individuals have extended I$_2$ 
    timers (\Cref{eq:tolerance}).
    
    \item \textbf{Time from exposure to symptom onset} 
    (incubation period). In the model, this is governed by 
    the E $\to$ I$_1$ transition rate, which is 
    \textbf{not trait-dependent} --- all infected individuals 
    progress through incubation at the same rate (modulated 
    only by temperature). This observable is therefore 
    \textbf{uninformative} for genetic selection.
\end{enumerate}

\subsubsection{Three survival pathways}

Challenge assay survivors fall into two distinct categories, 
each reflecting different underlying genetics:

\begin{description}
    \item[Pathway A --- Resistant (never infected):] Individual 
    was exposed to the pathogen but never became infected. 
    This is gated primarily by resistance $r_i$ 
    (\Cref{eq:infection_hazard}). These individuals are 
    enriched for resistance alleles.
    
    \item[Pathway B --- Tolerant/Recoverer (infected but 
    survived):] Individual was infected, progressed through 
    disease stages, but cleared the pathogen. This requires 
    sufficient recovery ability $c_i$ 
    (\Cref{eq:recovery,eq:early_recovery}), potentially 
    aided by tolerance $t_i$ extending the recovery window 
    (\Cref{eq:tolerance}). These individuals are enriched 
    for tolerance and recovery alleles.
\end{description}

Without additional diagnostics (e.g., PCR for pathogen load), 
these two categories may be indistinguishable in practice: 
both simply ``survived.'' However, if monitoring reveals that 
some survivors showed symptoms and others did not, this 
provides a partial decomposition.

\begin{remark}
The relative sizes of Pathway A and Pathway B depend on the 
model parameters and will change with calibration. What 
persists is the \emph{structure}: resistance acts before 
infection, while tolerance and recovery act after. The 
challenge assay conflates these into a single ``survived'' 
outcome unless symptom status is tracked.
\end{remark}

\subsubsection{Dose as a design parameter}
\label{subsubsec:dose}

The pathogen dose in a challenge assay is not given --- it is 
a \textbf{design choice} that determines what information the 
assay provides.

The infection hazard (\Cref{eq:infection_hazard}) includes a 
dose-response term $P_k / (K_{1/2} + P_k)$. The fraction of 
individuals who avoid infection depends on both dose and 
individual resistance:
\begin{equation}
    p_{\text{avoid}}(r_i, P) = \exp\!\left(-a \cdot 
    \frac{P}{K_{1/2} + P} \cdot (1 - r_i) \cdot S_{\text{sal}} 
    \cdot f_{\text{size}}\right)
    \label{eq:avoid_infection}
\end{equation}

This creates a trade-off:

\begin{center}
\begin{tabular}{p{0.28\textwidth}p{0.30\textwidth}p{0.30\textwidth}}
\toprule
\textbf{Dose regime} & \textbf{Advantage} & \textbf{Disadvantage} \\
\midrule
High dose \newline (high attack rate) & 
Strong discrimination: survivors are almost certainly 
genetically resistant & 
High mortality: kills most stock, including many with 
moderate resistance \\
\addlinespace
Moderate dose & 
Balanced: reasonable survival with moderate signal & 
Mixed signal: luck and genetics both contribute to survival \\
\addlinespace
Low dose \newline (low attack rate) & 
Preserves stock: most survive & 
Weak discrimination: most survival is stochastic, 
not genetic \\
\bottomrule
\end{tabular}
\end{center}

The optimal dose depends on how many individuals are available, 
how many founders are needed, and whether the goal is to 
identify the most resistant few (high dose) or broadly enrich 
the population (moderate dose). The calibrated model can 
simulate the outcome of each dose choice, providing 
quantitative guidance.

\subsubsection{Heritability of the binary phenotype}

The binary outcome (survived/died) can be analyzed using the 
\textbf{threshold model} of quantitative genetics 
\citep{falconer1996introduction}. The binary phenotype is 
determined by an underlying continuous ``liability'' (here, 
$p_{\text{surv}}(r_i, t_i, c_i)$). The heritability on the 
observed binary scale relates to the heritability on the 
underlying scale:
\begin{equation}
    h^2_{\text{obs}} = h^2_{\text{liability}} \cdot 
    \frac{z^2}{P(1 - P)}
    \label{eq:h2_observed}
\end{equation}
where $P$ is the population survival rate and 
$z = \phi(\Phi^{-1}(P))$ is the standard normal ordinate.

This relationship means:
\begin{itemize}
    \item At very low $P$ (stringent challenge): $h^2_{\text{obs}}$ 
    is moderate. Most variation is between ``always die'' and 
    ``sometimes survive.''
    \item At moderate $P$: $h^2_{\text{obs}}$ reaches a minimum 
    near $P = 0.5$. The binary outcome is maximally noisy.
    \item At high $P$ (weak challenge): $h^2_{\text{obs}}$ rises 
    again, but few die, limiting the selection differential.
\end{itemize}

The key implication: \textbf{the information content of a 
challenge assay is not fixed} --- it depends on dose, 
population genetics, and environmental conditions. As a 
breeding program advances (shifting $P$ upward), the assay 
must be recalibrated to maintain discrimination.

\subsubsection{Repeated exposures}

Re-challenging survivors amplifies discrimination because 
survival across $k$ independent exposures requires surviving 
each one:
\begin{equation}
    p_{\text{surv}}^{(k)} = \left(p_{\text{surv},i}\right)^k
    \label{eq:repeated_exposure}
\end{equation}

This exponentially amplifies the fitness difference between 
resistant and susceptible individuals. Two exposures roughly 
square the survival probability ratio between phenotypic 
classes.

The biology of \pyc{} facilitates this: because echinoderms 
lack adaptive immunity, recovered individuals return to the 
susceptible state (R $\to$ S in the model). Survivors of a 
first challenge can be re-challenged without waiting for a 
new generation, enabling sequential screening within a single 
cohort.

However, repeated exposure is costly --- each round kills a 
fraction of the stock, including genetically valuable 
individuals who were unlucky. The optimal number of rounds 
is a cost--benefit calculation that depends on the value of 
genetic information versus the cost of losing individuals.

\subsubsection{Family-based selection}
\label{subsubsec:family_selection}

The high fecundity of \pyc{} (millions of larvae per spawn) 
enables a fundamentally different approach: rather than 
phenotyping individual founders, challenge \textbf{offspring 
groups} from controlled crosses and compare \textbf{family 
survival rates}.

If family $j$ (from cross $i_1 \times i_2$) has $n_j$ 
offspring challenged and $k_j$ survive, the family survival 
rate $\hat{p}_j = k_j / n_j$ is an estimator of the parental 
breeding value. The precision scales with family size:
\begin{equation}
    \text{SE}(\hat{p}_j) = \sqrt{\frac{\hat{p}_j(1 - \hat{p}_j)}
    {n_j}}
    \label{eq:family_se}
\end{equation}

With $n_j = 100$ offspring per family, a family with true 
survival probability 0.30 has SE $\approx 0.046$, providing 
reasonable discrimination from the population mean. This 
converts the noisy individual binary outcome into a 
\textbf{precise family-level continuous phenotype}.

The heritability of the family mean is:
\begin{equation}
    h^2_{\text{family}} = \frac{h^2}{1 + (n - 1) r_{ICC}}
    \cdot n \cdot r_{ICC}
    \label{eq:h2_family}
\end{equation}
where $r_{ICC}$ is the intraclass correlation (proportion 
of variance due to between-family differences) and $n$ is 
family size. For large families, $h^2_{\text{family}}$ 
approaches 1 --- the family mean becomes a near-perfect 
predictor of parental breeding value.

\subsubsection{Strategic crossing of survival categories}

If Pathway A (never infected) and Pathway B (infected but 
recovered) survivors can be distinguished, they carry 
different genetic information:

\begin{center}
\begin{tabular}{lcc}
\toprule
& \textbf{Pathway A} & \textbf{Pathway B} \\
\midrule
Enriched for & Resistance ($r$) & Tolerance ($t$) + Recovery ($c$) \\
Best cross with & Other Pathway A & Pathway A (complementary traits) \\
Offspring advantage & Stacked resistance & Resistance + recovery \\
\bottomrule
\end{tabular}
\end{center}

Crossing Pathway A $\times$ Pathway A maximizes resistance in 
offspring. Crossing Pathway A $\times$ Pathway B creates 
offspring that may inherit both resistance alleles and 
tolerance/recovery alleles --- a hedge against imperfect 
resistance.

\subsubsection{Genomic markers: the transformative alternative}

If the loci identified by \citet{schiebelhut2018} can be 
validated as markers for resistance phenotype, non-lethal 
genotyping could replace challenge assays entirely. This 
would:
\begin{itemize}
    \item Eliminate false negatives (no genetically resistant 
    stars lost to stochastic disease)
    \item Provide continuous trait measurement (full 
    resolution, not binary)
    \item Enable screening without killing
    \item Allow within-family selection without sacrificing 
    offspring
    \item Make the theoretical framework in 
    \Cref{sec:screening} directly applicable
\end{itemize}

Until markers are validated, the challenge assay framework 
developed here is the operationally relevant one. 
\textbf{Validating resistance markers should be a high 
priority for the conservation program.}

\subsubsection{What the model provides}

The calibrated model can simulate the full challenge assay 
process:
\begin{enumerate}
    \item Generate populations with known genetic structure
    \item Simulate pathogen exposure at a specified dose
    \item Track individual outcomes (never infected, 
    recovered, died) with timing
    \item Compute the trait distribution of each outcome 
    category
    \item Evaluate the selection differential achieved by 
    different assay designs (dose, repeated exposure, 
    family vs.\ individual selection)
    \item Predict multi-generation breeding trajectories 
    under realistic phenotyping constraints
\end{enumerate}

This makes the model a \textbf{virtual laboratory} for 
optimizing breeding program design before committing real 
animals. The framework developed in this section defines the 
questions; the calibrated model answers them quantitatively.
